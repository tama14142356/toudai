% \documentclass[dvipdfmx, a4paper, 11pt]{jsarticle}%A4用紙縦、明朝(デフォルト)11ポイント
\documentclass[dvipdfmx,titlepage, 11pt, a4paper]{jsarticle}%A4用紙縦、明朝(デフォルト)11ポイント
\usepackage[top=18truemm,bottom=18truemm,left=18truemm,right=18truemm]{geometry}%余白調整
% \setlength{\textheight}{45\baselineskip}
% \setlength{\textwidth}{46zw}% 46文字/行

\usepackage{template}

%%%====================================================================================================================================
\renewcommand{\thesection}{第\arabic{section}問}
\renewcommand{\thesubsection}{\thesection}
\titleformat*{\section}{\LARGE\mcfamily}%章のタイトルの文字の大きさを通常サイズに設定(明朝体で)
\titlespacing*{\section}{0pt}{*0}{0pt}%章番号の後の空白行の削除
\titleformat*{\subsection}{\Large\mcfamily}%節のタイトルの文字の大きさを通常サイズに設定(明朝体で)
\titlespacing*{\subsection}{0pt}{*0}{0pt}%節番号の後の空白行の削除
\titleformat*{\subsubsection}{\large\mcfamily}%小節のタイトルの文字の大きさを通常サイズに設定(明朝体で)
\titlespacing*{\subsubsection}{0pt}{*0}{0pt}%小節番号の後の空白行の削除

\makeatletter
%章番号付きコード番号
\AtBeginDocument{
  \renewcommand*{\thelstlisting}{\arabic{section}.\arabic{lstlisting}}%
  \@addtoreset{lstlisting}{section}
}

%章番号付き表番号
\renewcommand{\thetable}{%
    \arabic{section}.\arabic{table}%
}
\@addtoreset{table}{section}%

%章番号付き図番号
\renewcommand{\thefigure}{%
	\arabic{section}.\arabic{figure}%
}%
\@addtoreset{figure}{section}%

%章番号付き式番号
\renewcommand{\theequation}{%
	\arabic{section}.\arabic{equation}%
}%
\@addtoreset{equation}{section}%

\renewcommand{\p@enumiii}{}% 箇条書きの参照時の番号の変更(2(親の番号)a(子の番号)→aだけに)
\renewcommand{\p@enumii}{}% 箇条書きの参照時の番号の変更(2a→aだけに)
\makeatother
%%%====================================================================================================================================

%%%====================================================================================================================================
%%ページのレイアウト設定
\pagestyle{fancy}
\renewcommand{\sectionmark}[1]{\markboth{}{\thesection\ #1}}%                   %\rightmarkにセクション名を格納
\renewcommand{\subsectionmark}[1]{\markboth{#1}{\rightmark}}%   %\leftmarkにサブセクション名を格納
%[]は省略可で省略すると{}で指定された内容が偶数ページ奇数のどちらにも適用される。
% \renewcommand{\headrulewidth}{0pt} %ヘッダの罫線を消す 
\fancyfoot{}%                        %clear all footer fields
\lhead{\leftmark}%                   %左側ヘッダの定義[<偶数ページ>]{<奇数ページ>}
\chead{}%                            %中央ヘッダの定義[<偶数ページ>]{<奇数ページ>}
\rhead{\rightmark}%                  %右側ヘッダの定義[<偶数ページ>]{<奇数ページ>}
\lfoot{東大2019年度数学解答例}%       %左側フッターの定義[<偶数ページ>]{<奇数ページ>}
\cfoot{\thepage}%                    %中央フッターの定義[<偶数ページ>]{<奇数ページ>}
\rfoot{文殊の知恵}%                  %右側フッターの定義[<偶数ページ>]{<奇数ページ>}
\renewcommand{\headrulewidth}{0.1pt}%%ヘッダの線の太さ 
\renewcommand{\footrulewidth}{0.1pt}%%フッターの線の太さ
%%%====================================================================================================================================

\makeindex%索引用

\renewcommand{\refname}{}%参考文献の文字を非表示にする
\title{\Huge 東大2019年度数学解答例\\[10mm]}
\author{{\LARGE 文殊の知恵}\\[1mm]\LARGE 高橋那弥}
\date{}

\begin{document}
\maketitle
\tableofcontents % 目次
\pagenumbering{roman}%目次のページ番号のスタイルをローマ数字にする
\newpage
\setcounter{tocdepth}{3}%章節の深さを3にするsubsubsectionまで
\pagenumbering{arabic}%他のページ番号は通常の数字にする。
\section{}%第1問
\subsection{問題文}
複素正方行列$X$は$XX^{\ast}=I$を満たすとき、ユニタリ行列であるという。但し、$X^{\ast}$は行列$X$の共役転置行列(もしくは随伴行列)を
表し、$I$は単位行列とする。また、iは虚数単位とする。以下の問いに答えよ。
\begin{enumerate}[(1)]
    \setlength{\itemsep}{10pt}
    \item $n$を正の整数とし、$A, B$を$n$次ユニタリ行列とする。行列$AB$もユニタリ行列であることを示せ。
    \item $n$を正の整数とし、$C, D$を$n$次実正方行列とする。行列$F$を$F = C + \imag D$と定義し、行列$G$を
    \begin{align*}
        G = \spalignmat{
            C {-D};
            D {C}
        }
    \end{align*}
    と定義する。行列$F$がユニタリ行列であることと行列$G$が直交行列であることは同値であることを示せ。
    \item 次の行列の固有値を求めよ。
    \begin{align*}
        \frac{1}{2}\spalignmat{
            1 1 1 1;
            1 {\imag} -1 {-\imag};
            1 -1 1 -1;
            1 {-\imag} -1 {\imag}
        }
    \end{align*}
    \item $n$を正の整数とし、$n$次正方行列$Q$の$(j, k)$成分$q_{jk}$を
    \begin{equation*}
        q_{jk} = \cfrac{1}{\sqrt{n}}\exp \left(\cfrac{2\pi\imag (j - 1)(k - 1)}{n}\right)
    \end{equation*}
    とする。行列$Q$はユニタリ行列であることを示せ。
    \item 行列式が1である2次のユニタリ行列は次の一形式を持つことを示せ。但し、$\theta, \psi$は実数であるとする。
    \begin{equation*}
        H = \spalignmat{
            {\exp(\imag \psi)\cos\theta} {\exp(\imag \psi)\sin\theta};
            {-\exp(-\imag \psi)\sin\theta} {\exp(-\imag \psi)\cos\theta}
        }
    \end{equation*}
    \item 2次のユニタリ行列の一般形を求めよ
\end{enumerate}

\newpage

\subsection{解答例}
\begin{enumerate}[(1)]
    \setlength{\itemsep}{10pt}
    \item $A, B$ともにユニタリ行列より以下が成り立つ。
    \begin{align}
        \left\{
            \begin{array}{lcl}
            AA^{\ast} &=& I\\ 
            BB^{\ast} &=& I\\ 
            \end{array}
        \right.\label{eq:abunitary}
    \end{align}
    よって、式\eqref{eq:abunitary}より、以下が成り立つ。
    \begin{align*}
        (AB)(AB)^{\ast} & = ABB^{\ast}A^{\ast}\\
                        & = AIA^{\ast}\\
                        & = AA^{\ast}\\
                        & = I
    \end{align*}
    よって、行列$AB$についてもユニタリ行列であることが示された。
    \item 題意は以下のように同値変形できる。
    \begin{align}
        \mbox{行列$F$がユニタリ行列であること}&\mbox{と行列$G$が直交行列であることは同値である}\nonumber\\
        \Longleftrightarrow \mbox{行列$F$がユニタリ行列である} &\Leftrightarrow \mbox{行列$G$が直交行列である} \nonumber\\
        \Longleftrightarrow FF^{\ast} = I &\Leftrightarrow GG^{\top} = I\label{eq:hodai1_2}
    \end{align}
    よって、式\eqref{eq:hodai1_2}が成り立つことを示せばよい。

    まず以下の式\eqref{eq:hodai1_2_1}が成り立つことを示す。
    \begin{equation}
        FF^{\ast} = I \Rightarrow GG^{\top} = I \label{eq:hodai1_2_1}
    \end{equation}
    題意より、$F = C + \imag D$より、$F^{\ast} = C^{\top} - \imag D^{\top}$であり、
    式\eqref{eq:hodai1_2_1}の仮定条件$FF^{\ast} = I$から以下が成り立つ。
    \begin{align*}
        FF^{\ast} & = \left(C + \imag D\right)\left(C^{\top} - \imag D^{\top}\right)\\
                & = CC^{\top} + DD^{\top} + \imag \left(DC^{\top} - CD^{\top}\right)\\
                & = I\\
        \Longleftrightarrow
        I & = CC^{\top} + DD^{\top} + \imag \left(DC^{\top} - CD^{\top}\right)\\
        \Longrightarrow I&\mbox{は実正方行列より} \mbox{\boldmath $0$} = DC^{\top} - CD^{\top}\\
        \Longrightarrow I & = CC^{\top} + DD^{\top}\\
        \therefore GG^{\top} & = 
        \spalignmat{
            C {-D};
            D {C}
        }
        \spalignmat{
            {C^{\top}} {D^{\top}};
            {-D^{\top}} {C^{\top}}
        }\\
        & = 
        \spalignmat{
            {CC^{\top} + DD^{\top}} {CD^{\top} - DC^{\top}};
            {DC^{\top} - CD^{\top}} {DD^{\top} + CC^{\top}}
        }\\
        & = 
        \spalignmat{
            {I} {\mbox{\boldmath $0$}};
            {\mbox{\boldmath $0$}} {I}
        }\\
        & = I
    \end{align*}
    よって、式\eqref{eq:hodai1_2_1}が成り立つことは示された。

    次に以下の式\eqref{eq:hodai1_2_2}が成り立つを示す。
    \begin{equation}
        GG^{\top} = I \Rightarrow FF^{\ast} = I \label{eq:hodai1_2_2}
    \end{equation}
    題意と式\eqref{eq:hodai1_2_2}の仮定条件$GG^{\top} = I$から以下が成り立つ。
    \begin{align*}
        GG^{\top} & = 
        \spalignmat{
            C {-D};
            D {C}
        }
        \spalignmat{
            {C^{\top}} {D^{\top}};
            {-D^{\top}} {C^{\top}}
        }\\
        & = 
        \spalignmat{
            {CC^{\top} + DD^{\top}} {CD^{\top} - DC^{\top}};
            {DC^{\top} - CD^{\top}} {DD^{\top} + CC^{\top}}
        }\\
        & = I\\
        \Longleftrightarrow &
        \begin{cases}
            \mbox{\boldmath $0$} = CD^{\top} - DC^{\top}\\
            I = CC^{\top} + DD^{\top}
        \end{cases}
    \end{align*}
    \begin{align*}
        \therefore FF^{\ast} & = \left(C + \imag D\right)\left(C^{\top} - \imag D^{\top}\right)\\
            & = CC^{\top} + DD^{\top} + \imag \left(DC^{\top} - CD^{\top}\right)\\
            & = I + \mbox{\boldmath $0$}\\
            & = I
    \end{align*}
    よって、式\eqref{eq:hodai1_2_2}が成り立つことが示された。

    従って、式\eqref{eq:hodai1_2_1}, \eqref{eq:hodai1_2_2}が成り立つことが示されたので、式\eqref{eq:hodai1_2}
    が成り立つことが示された。よって、題意は示された。\\[1cm]
    (中田解)
    \begin{eqnarray*}
    行列Fがユニタリ行列である&\Longleftrightarrow&FF^{*}=I\\
                            &\Longleftrightarrow&(C+{\rm i}D)(C+{\rm i}D)^{*}=I\\
                            &\Longleftrightarrow&(C+{\rm i}D)(C^{\sf T}-{\rm i}D^{\sf T})=I\\
                            &\Longleftrightarrow&(CC^{\sf T}+DC^{\sf T})+{\rm i}(DC^{\sf T}+CD^{\sf -T})=I\\
                            &\Longleftrightarrow&\left\{\begin{array}{l}CC^{\sf T}+DD^{\sf T}=I\\DC^{\sf T}+CD^{\sf T}=\bm{0}\end{array}\right.\\
                            &\Longleftrightarrow&\begin{pmatrix}CC^{\sf T}+DD^{\sf T}&\bm{0}\\\bm{0}&CC^{\sf T}+DD^{\sf T}\end{pmatrix}=I\\
                            &\Longleftrightarrow&\begin{pmatrix}C&-D\\D&C\end{pmatrix}\begin{pmatrix}C^{\sf T}&D^{\sf T}\\-D^{\sf T}&C^{\sf T}\end{pmatrix}=I\\
                            &\Longleftrightarrow&GG^{\sf T}=I
    \end{eqnarray*}
    よって, 題意は示された.
    \item 題意の4次正方行列を$A$とおき、$\mid A \mid$は行列$A$の行列式を表すとすると、
    固有値$\lambda$は以下を満たす。
    \begin{equation*}
        \mid \lambda I - A \mid = 0
    \end{equation*}
    よって、この方程式を解くと以下のようになる。
    \begin{align*}
        \mid \lambda I - A \mid 
        & = \left(\frac{1}{2}\right)^{4} \mid 2\lambda I - 2A \mid\\
        & = \frac{1}{16}
        \spaligndelims\vert\vert \spalignmat{
            {2\lambda - 1} -1 -1 -1;
            -1 {2\lambda - \imag} 1 {\imag};
            -1 1 {2\lambda - 1} 1;
            -1 {\imag} 1 {2\lambda - \imag}
        }\\
        & = \frac{1}{16}
        \spaligndelims\vert\vert \spalignmat{
            0 {(2\lambda - 1)(2\lambda - \imag) - 1} {2\lambda - 2} {\imag(2\lambda - 1) - 1};
            -1 {2\lambda - \imag} 1 {\imag};
            0 {1 + \imag - 2\lambda} {2\lambda - 2} {1 - \imag};
            0 {2\imag - 2\lambda} 0 {2\lambda - 2\imag}
        }\\
        & = \frac{1}{16}\times (-1)^{2 + 1}\times (-1)
        \spaligndelims\vert\vert \spalignmat{
            {(2\lambda - 1)(2\lambda - \imag) - 1} {2\lambda - 2} {\imag(2\lambda - 1) - 1};
            {1 + \imag - 2\lambda} {2\lambda - 2} {1 - \imag};
            {2\imag - 2\lambda} 0 {2\lambda - 2\imag}
        }\\
        & = \frac{1}{16}
        \spaligndelims\vert\vert \spalignmat{
            {(2\lambda - 1)(2\lambda - \imag) - 1} {2\lambda - 2} {2\lambda(2\lambda - 1) - 2};
            {1 + \imag - 2\lambda} {2\lambda - 2} {2 - 2\lambda};
            {2\imag - 2\lambda} 0 0
        }\\
        & = \frac{1}{16}\times (-1)^{3 + 1} \times (2\imag - 2\lambda)
        \spaligndelims\vert\vert \spalignmat{
            {2\lambda - 2} {2\lambda(2\lambda - 1) - 2};
            {2\lambda - 2} {2 - 2\lambda}
        }\\
        & = \frac{1}{16}(2\imag - 2\lambda)(2\lambda - 2)\left[2 - 2\lambda - \left\{2\lambda(2\lambda - 1) - 2\right\}\right]\\
        & = \frac{1}{2}(\imag - \lambda)(\lambda - 1)(2 - \lambda - 2\lambda^2 + \lambda)\\
        & = (\imag - \lambda)(\lambda - 1)(1 - \lambda^2)\\
        & = (\lambda - \imag)(\lambda - 1)^2(\lambda + 1)\\
        \therefore\, \lambda & = \pm 1, \imag
    \end{align*}
    よって、固有値は$\pm 1 \imag$である。\\[1cm]
    (中田解)\\
    この行列を$A$とし, この行列$A$に対する固有値を$\lambda$, 固有ベクトルを$\bm{x}$とおくと
    \begin{eqnarray*}
        A\bm{x} = \lambda\bm{x} &\Longleftrightarrow& (\lambda I -A)\bm{x} = \bm{0}\\
                                &\Longleftrightarrow& {\rm det}|\lambda I-A|=0
    \end{eqnarray*}
    が成り立つ. ゆえに求める固有値$\lambda$は
    \begin{eqnarray*}
        && \mathrm{det}\left\lvert
        \begin{pmatrix}
            \lambda & 0 & 0 & 0\\
            0 & \lambda & 0 & 0\\
            0 & 0 & \lambda & 0\\
            0 & 0 & 0 & \lambda 
        \end{pmatrix}
        -\frac{1}{2}
        \begin{pmatrix}
            1 & 1 & 1 & 1\\
            1 & \imag & -1 & -\imag\\
            1 & -1 & 1 & -1\\
            1 & -\imag & -1 & \imag
        \end{pmatrix}
        \right\rvert = 0\\
        \Longleftrightarrow\ && \mathrm{det}\left\lvert
        \frac{1}{2}
        \begin{pmatrix}
            2\lambda - 1 & -1 & -1 & -1\\
            -1 & 2\lambda - \imag & 1 & \imag\\
            -1 & 1 & 2\lambda - 1 & 1\\
            -1 & \imag & 1 & 2\lambda - \imag
        \end{pmatrix}\right\rvert = 0\\
        \Longleftrightarrow\ && 
        \begin{vmatrix}
            2\lambda - 1 & -1 & -1 & -1\\
            -1 & 2\lambda - \imag & 1 & \imag\\
            -1 & 1 & 2\lambda - 1 & 1\\
            -1 & \imag & 1 & 2\lambda - \imag
        \end{vmatrix} = 0\\
        \Longleftrightarrow\ && 
        \begin{vmatrix}
            0 & -1 + (2\lambda - 1)(2\lambda - \imag) & 2\lambda - 2 & -1 + \imag(2\lambda - 1)\\
            -1 & 2\lambda - \imag & 1 & \imag\\
            0 & -2\lambda + \imag + 1 & 2\lambda - 2 & 1 - \imag\\
            0 & -2\lambda + 2\imag & 0 & 2\lambda - 2\imag
        \end{vmatrix} = 0\\
        \Longleftrightarrow\ && 
        \begin{vmatrix}
            -1 + (2\lambda - 1)(2\lambda - \imag) & 2\lambda - 2 & -1 + \imag(2\lambda - 1)\\
            -2\lambda + \imag + 1 & 2\lambda - 2 & 1 - \imag\\
            -2\lambda + 2\imag & 0 & 2\lambda - 2\imag
        \end{vmatrix} = 0\\
        \Longleftrightarrow\ && 
        \begin{vmatrix}
            -1 + (2\lambda - 1)(2\lambda - \imag) & 2\lambda - 2 & -2 + 2\lambda(2\lambda - 1)\\
            -2\lambda + \imag + 1 & 2\lambda - 2 & -2\lambda + 2\\
            -2\lambda + 2\imag & 0 & 0
        \end{vmatrix} = 0\\
        \Longleftrightarrow\ && (-2\lambda + 2\imag)
        \begin{vmatrix}
            2\lambda - 2 & -2 + 2\lambda(2\lambda - 1)\\
            2\lambda - 2& -2\lambda + 2
        \end{vmatrix} = 0\\
        \Longleftrightarrow\ && (\lambda - \imag)(\lambda - 1)
        \begin{vmatrix}
            1 & -2 + 2\lambda(2\lambda - 1)\\
            1 & -2\lambda + 2
        \end{vmatrix} = 0\\
        \Longleftrightarrow\ && (\lambda - \imag)(\lambda - 1)\bigl\{-2\lambda + 2 + 2 - 2\lambda(2\lambda - 1)\bigr\} = 0\\
        \Longleftrightarrow\ && (\lambda - \imag)(\lambda - 1)(-4\lambda^{2} + 4) = 0\\
        \Longleftrightarrow\ && (\lambda - \imag)(\lambda - 1)(\lambda^{2} - 1) = 0\\
        \Longleftrightarrow\ && (\lambda - \imag)(\lambda - 1)^{2}(\lambda + 1) = 0\\
        \Longleftrightarrow\ && \lambda = \pm 1,\imag
    \end{eqnarray*}
    よって, 固有値は$\pm 1,\ \imag$である.\\
    \underline{中田別方針}\\
    問題の行列を$A$とすると
    \begin{eqnarray*}
        AA^{\ast}=I
    \end{eqnarray*}
    より,$A$はユニタリ行列である. この$A$の固有値を$\lambda$, 固有ベクトルを$\bm{x}$とすると
    \begin{eqnarray*}
        A\bm{x}=\lambda \bm{x}
    \end{eqnarray*}
    が成り立ち, 複素内積と随伴行列の間に
    \begin{eqnarray*}
        \langle \bm{x},A\bm{y}\rangle = \langle A^{\ast}\bm{x},\bm{y}\rangle
    \end{eqnarray*}
    の関係があることから
    \begin{eqnarray*}
        \langle A\bm{x},A\bm{x}\rangle &=& \langle A^{\ast}A\bm{x},\bm{x}\rangle\\
                                       &=&\langle \bm{x},\bm{x}\rangle\\
                                       &=&\|\bm{x}\|^{2}
    \end{eqnarray*}
    となり, $A\bm{x}$同士の内積は$\bm{x}$のノルムの2乗に等しくなる.\\
    一方で, 複素内積の性質で
    \begin{eqnarray*}
        \langle \bm{x},\alpha \bm{y}\rangle &=& \alpha \langle \bm{x},\bm{y}\rangle \\
        \langle \alpha \bm{x},\bm{y}\rangle &=& \alpha^{\ast}\langle \bm{x},\bm{y}\rangle
    \end{eqnarray*}
    となるので,
    \begin{eqnarray*}
        \langle A\bm{x},A\bm{x}\rangle &=& \langle \lambda \bm{x},\lambda \bm{x}\rangle \\
                                       &=&\lambda^{\ast}\langle \bm{x},\lambda \bm{x}\rangle\\
                                       &=& \lambda^{\ast}\lambda \langle \bm{x},\bm{x}\rangle\\
                                       &=&|\lambda|^{2}\|\bm{x}\|^{2}
    \end{eqnarray*}
    したがって,
    \begin{eqnarray*}
      \|\bm{x}\|^{2} = |\lambda|^{2}\|\bm{x}\|^{2}
    \end{eqnarray*}
    ここで, $\bm{x}\neq \bm{0}$から$\|\bm{x}\|^{2}\neq 0$であるので,
    \begin{eqnarray*}
      |\lambda|^{2} = 1\Longleftrightarrow |\lambda| = 1
    \end{eqnarray*}
    となる. 4次のユニタリ行列であるので, $\lambda =\pm 1,\pm \imag$が候補に上がる. これから固有ベクトルを求めて一致するかどうかを確認する.
    \item 題意より複素数$z$に対する共役な複素数を$\overline{z}$と表すとき、
    $n$次正方行列$Q$の共役転置行列$Q^{\ast}$の$(j, k)$成分$q^{\ast}_{jk}$は以下のようになる。
    \begin{align}
        q^{\ast}_{jk} & = \overline{q_{kj}}\nonumber\\
        & = \cfrac{1}{\sqrt{n}}\exp \left(\cfrac{-2\pi\imag (k - 1)(j - 1)}{n}\right)\label{eq:kyoyaku}
    \end{align}
    よって、式\eqref{eq:kyoyaku}から$QQ^{\ast}$の$(j, k)$成分$Q_{jk}$は以下のようになる。
    \begin{align*}
        Q_{jk} & = \sum_{l = 1}^{n} \left(q_{jl}\times q^{\ast}_{lk}\right)\\
               & = \sum_{l = 1}^{n}
               \left\{
               \cfrac{1}{\sqrt{n}}\exp \left(\cfrac{2\pi\imag (j - 1)(l - 1)}{n}\right)
               \times 
               \cfrac{1}{\sqrt{n}}\exp \left(\cfrac{-2\pi\imag (k - 1)(l - 1)}{n}\right)
               \right\}\\
               & = \cfrac{1}{n} \sum_{l = 1}^{n}
               \exp\left(\cfrac{2\pi\imag (j - 1)(l - 1)}{n} + \cfrac{-2\pi\imag (k - 1)(l - 1)}{n}\right)\\
               & = \cfrac{1}{n} \sum_{l = 0}^{n - 1}
               \exp\left(\cfrac{2\pi\imag (j - k)l}{n}\right)\\
               & = 
               \begin{cases}
                \cfrac{1}{n}\sum\limits_{l = 0}^{n - 1}\exp(0) & j = k\\
                &\\
                \cfrac{1}{n}\cfrac{\exp\left(\cfrac{2\pi\imag (j - k)n}{n}\right) - \exp(0)}{\exp\left(\cfrac{2\pi\imag (j - k)}{n}\right) - 1}& j \neq k
               \end{cases}\\
        \mbox{オイラーの公式から}&j, k\mbox{は整数より}\\
        Q_{jk} & = 
        \begin{cases}
            1 & j = k\\
            \cfrac{1}{n}\cfrac{\cos\{2\pi(j - k)\} + \imag \sin\{2\pi(j - k)\} - 1}{\exp\left(\cfrac{2\pi\imag (j - k)}{n}\right) - 1} = 0 & j \neq k
        \end{cases}
    \end{align*}
    従って、対角成分のみ1となり、他の成分は全て0となるので、$QQ^{\ast}$は単位行列となる。従って、$Q$はユニタリ行列であることが示された。
    \item 題意の2次正方行列$H$についてユニタリ行列であることを示す。
    \begin{align*}
        HH^{\ast} & = 
        \spalignmat{
            {\exp(\imag \psi)\cos\theta} {\exp(\imag \psi)\sin\theta};
            {-\exp(-\imag \psi)\sin\theta} {\exp(-\imag \psi)\cos\theta}
        }
        \spalignmat{
            {\exp(-\imag \psi)\cos\theta} {-\exp(\imag \psi)\sin\theta};
            {\exp(-\imag \psi)\sin\theta} {\exp(\imag \psi)\cos\theta}
        }\\
        & = 
        \spalignmat{
            {\exp(\imag \psi - \imag \psi)(\cos^{2}\theta + \sin^{2}\theta)} 
            {\exp(2\imag \psi)(-\cos\theta\sin\theta + \sin\theta\cos\theta)};
            {\exp(-2\imag \psi)(-\sin\theta\cos\theta + \cos\theta\sin\theta)} 
            {\exp(-\imag \psi + \imag \psi)(\sin^{2}\theta + \cos^{2}\theta)}
        }\\
        & = 
        \spalignmat{
            {1} {0};
            {0} {1}
        }\\
        & = I
    \end{align*}
    よって、行列$H$はユニタリ行列である。また、行列$H$の行列式は以下のようになる。
    \begin{align*}
        \mid H \mid & = \exp(\imag \psi)\cos\theta\times\exp(-\imag \psi)\cos\theta - 
        (-\exp(-\imag \psi)\sin\theta)\times\exp(\imag \psi)\sin\theta\\
        & = \cos^{2}\theta + \sin^{2}\theta = 1
    \end{align*}
    よって、この2次正方行列$H$は行列式が1でユニタリ行列であるので、行列式が1で2次のユニタリ行列の一形式となる
    ことが示された。
    \item 解けなかったので後述。
\end{enumerate}
\newpage
\section{}%第2問
\subsection{問題文}
実数値関数$u(x, t)$が$0\, \leq \, x\, \leq\, 1, \; t\, \geq \, 0$で定義されている。
ここで$x$と$t$は互いに独立である。偏微分方程式
\begin{equation*}
    \pdiff{u}{t} = \pdiff[2]{u}{x} \qquad (*)
\end{equation*}
の解を次の条件
\begin{align*}
    \mbox{境界条件:} & \quad u(0, t) = u(1, t) = 0\\
    \mbox{初期条件:} & \quad u(x, 0) = x - x^{2}
\end{align*}
のもとで求める。但し、定数関数$u(x, t) = 0$は明らかに解であるから、それ以外の解を考える。
以下の問いに答えよ。
\begin{enumerate}[(1)]
    \item 次の式を計算せよ。ここで, $n, m$はともに正の整数とする。
        \begin{equation*}
            \dint{0}{1}{\sin (n\pi x)\; \sin (m\pi x)}
        \end{equation*}\label{subsec:prom2:subprom1}
    \item $x$のみの関数$\xi (x)$及び$t$のみの関数$\tau (t)$を用いて、$u(x, t) = \xi (x)\tau (t)$と置けるとする。
        任意の定数$C$を用いて、$\xi$および$\tau$が満たす常微分方程式をそれそれ表せ。関数$f(x)$と関数$g(t)$が任意の
        $x$と$t$について$f(x) = g(t)$を満たす場合は、$f(x)$と$g(t)$が定数関数となることを用いてもよい。\label{subsec:prom2:subprom2}
    \item 設問\eqref{subsec:prom2:subprom2}の常微分方程式を解け。次に、境界条件を満たす偏微分方程式$(*)$の解の一つが次の式で表される$u_n(x, t)$
        で与えられることを示し、$\alpha, \beta$を正の整数$n$を用いて表せ。
        \begin{equation*}
            u_n (x, t) = e^{\alpha t}\sin (\beta x)
        \end{equation*}\label{subsec:prom2:subprom3}
    \item 境界条件と初期条件を満たす偏微分方程式$(*)$の解は$u_n (x, t)$の線形結合として次の式で表される。$c_n$を求めよ。設問\eqref{subsec:prom2:subprom1}の結果を
        用いてもよい。
        \begin{equation*}
            u(x, t) = \sum\limits_{n = 1}^{\infty} c_n u_n (x, t)
        \end{equation*}\label{subsec:prom2:subprom4}
\end{enumerate}
\newpage
\subsection{解答}
\begin{enumerate}[(1)]
    \item 題意の式より以下が成り立つ。
        \begin{align*}
            \mbox{(与式)} 
            & = \frac{1}{2}\dint{0}{1}{\cos (n\pi x - m\pi x) - \cos (n\pi x + m\pi x)}\\
            & = 
            \begin{cases}
                \frac{1}{2}\dint{0}{1}{1 - \cos (2n\pi x)} & n = m\\
                \frac{1}{2}\left[\frac{\sin ((n - m)\pi x)}{(n - m)\pi}\right]_{0}^{1} - \frac{1}{2}\left[\frac{\sin ((n + m)\pi x)}{(n + m)\pi}\right]_{0}^{1} & n \neq m\\
            \end{cases}\\
            & = 
            \begin{cases}
                \frac{1}{2} - \frac{1}{2}\left[\frac{\sin (2n\pi x)}{2n\pi}\right]_{0}^{1} & n = m\\
                0 & n \neq m\\
            \end{cases}\\
            & = 
            \begin{cases}
                \frac{1}{2} & n = m\\
                0 & n \neq m\\
            \end{cases}
        \end{align*}
    \item 偏微分方程式$(*)$より$u(x, t) = \xi (x)\tau (t)$と表せるとすると以下が成り立つ。
        \begin{align*}
            (*) \Longleftrightarrow
            \pdiff{\{\xi (x)\tau (t)\}}{t} & = \pdiff[2]{\{\xi (x)\tau (t)\}}{x}\\
            \Longleftrightarrow
            \xi (x)\diff{\tau (t)}{t} & = \tau (t)\diff[2]{\xi (x)}{x}
        \end{align*}
        ここで、$\xi (x)\tau(t) \neq 0$より$\xi (x) \neq 0, \tau (t) \neq 0$よって、以下のようになる。
        \begin{align*}
            \frac{\diff[2]{\xi (x)}{x}}{\xi (x)} &= \frac{\diff{\tau (t)}{t}}{\tau (t)}\\
            \therefore &
            \begin{cases}
                \frac{\diff[2]{\xi (x)}{x}}{\xi (x)} = C\\
                \frac{\diff{\tau (t)}{t}}{\tau (t)} = C\\
            \end{cases}\\
            \Longleftrightarrow &
            \begin{cases}
                C\xi (x) = \diff[2]{\xi (x)}{x}\\
                C\tau (t) = \diff{\tau (t)}{t}\\
            \end{cases}\\
        \end{align*}
    \item $(2)$よりそれぞれ解くと以下のようになる。\\
    $\tau (t)$に関する常微分方程式を解く。$\tau (t) \neq 0$より以下が成り立つ。
        \begin{align*}
            C &= \frac{1}{\tau (t)}\diff{\tau (t)}{t}
        \end{align*}
        よって両辺$t$で積分して、
        \begin{align*}
            \dint[t]{}{}{C} &= \dint[t]{}{}{\frac{1}{\tau (t)}\diff{\tau (t)}{t}}\\
            \dint[t]{}{}{C} &= \dint[\tau]{}{}{\frac{1}{\tau}} \qquad(分かりやすくするため(t)を省略)\\
            Ct + C_3 &= \log |\tau (t)|\\
            |\tau (t)| &= e^{Ct + C_3}\\
            \tau (t) &= e^{Ct + C_3}
        \end{align*}
        よって、一般解は以下のようになる。
        \begin{equation*}
            \tau (t) = e^{Ct + C_3}
        \end{equation*}
        ここで常微分方程式から$t\to \infty$の時、$\tau (t)$は発散しないが、$C > 0$とすると、
        この一般解の式から$\tau (t)$は発散するので矛盾してしまう。よって、$C < 0$となるので、正の実数$k$を用いて、$C = -k^2$とおく。\\
    次に$\xi (x) = C_1 e^{\lambda x}$とおくと、$(2)$の常微分方程式と、$C = -k^2$より
        \begin{align*}
            -k^2 C_1 e^{\lambda x} &= \lambda^2 C_1 e^{\lambda x}\\
            -k^2 &= \lambda^2 \\
            \therefore \lambda &= \pm \imag k
        \end{align*}
    よって、斉次の微分方程式より独立な2つの解の和も解となるので、$\xi(x)$の一般解は以下のようになる。
        \begin{align*}
            \xi (x) &= C_1 e^{\imag kx} + C_2 e^{-\imag kx}\\
            \xi (x) &= (C_1 + C_2)\cos (kx) + \imag(C_1 - C_2)\sin (kx)
        \end{align*}
        よって、$C_1 + C_2 = A \in \mathbb{R}, \imag(C_1 - C_2) = B\in\mathbb{R}$とおくと以下のようになる。
        \begin{align*}
            \xi (x) &= A\cos (kx) + B\sin (kx)
        \end{align*}
        次に境界条件を満たす偏微分方程式$(*)$の解の一つに題意の式$u_n(x, t)$が存在することを示す。\\
        $(2)$より$u(x, t) = \xi (x)\tau (t) = e^{-k^2 t + C_3}(A\cos (kx) + B\sin (kx))$となる。\\
        よって、この時初期条件より
        \begin{align*}
            \tau (0) &= 1\\
            e^{C_3} &= 1\\
            \therefore C_3 &= 0\\
            \therefore \tau (t) &= e^{-k^2t}
        \end{align*}
        境界条件より、
        \begin{align*}
            &\begin{cases}
                \xi(0) = 0\\
                \xi(1) = 0\\
            \end{cases}\\
            &\begin{cases}
                A\cos (0) + B\sin (0) = 0\\
                A\cos (k) + B\sin (k) = 0\\
            \end{cases}\\
            &\begin{cases}
                A = 0\\
                B\sin (k) = 0\\
            \end{cases}\\
        \end{align*}
        よって、$\xi (x) \neq 0$より$B \neq 0$より、
        \begin{align*}
            \sin (k) &= 0\\
            \therefore k = n\pi
        \end{align*}
        \begin{equation*}
            \xi (x) = B\sin (n\pi x)
        \end{equation*}
        となる。よって、この時、$(2)$より、$u(x, t) = \xi (x)\tau (t)$は偏微分方程式$(*)$を満たすので、以下の式はこの偏微分方程式の解の一つである。
        \begin{equation*}
            u(x, t) = Be^{-n^2\pi^2t}\sin(n\pi x)
        \end{equation*}
        よって、この式の$B = 1, n\pi = \beta, -n^2\pi^2 = \alpha$とおくと、
        \begin{align*}
            e^{\alpha t}\sin(\beta x) = u_n(x, t)
        \end{align*}
        となるので、題意は示された。また、$\alpha = -n^2\pi^2, \beta = n\pi$となる。
    \item 題意のように偏微分方程式$(*)$は線形であるため、その解は$u_n(x, t)$の線形結合で表される。よって、$(3)$より以下のようになる。
    \begin{align*}
        初期条件から、u(x, 0) &= x - x^2 = \sum_{n = 1}^{\infty}c_n\sin(n\pi x)\\
        \dint{0}{1}{(x - x^2)\sin(m\pi x)} &= \dint{0}{1}{\left(\sum_{n = 1}^{\infty}c_n\sin(n\pi x)\right)\sin (m\pi x)}
    \end{align*}
    よって、それぞれ計算する。
    \begin{align*}
        m = 0の時、
        (左辺) = 0&\\
        m \neq 0の時、\qquad
        \dint{0}{1}{x\sin(m\pi x)} &= \left[x\frac{-\cos(m\pi x)}{m\pi}\right]_{0}^{1} + \frac{1}{m\pi}\dint{0}{1}{\cos(m\pi x)}\\
        &= \frac{(-1)^{m + 1}}{m\pi} + \frac{1}{m^2\pi^2}\bigl[\sin (m\pi x)\bigr]_{0}^{1} = \frac{(-1)^{m + 1}}{m\pi}\\
        \dint{0}{1}{x\cos(m\pi x)} &= \left[x\frac{\sin(m\pi x)}{m\pi}\right]_{0}^{1} - \frac{1}{m\pi}\dint{0}{1}{\sin(m\pi x)}\\
        &= -\frac{1}{m^2\pi^2}\bigl[-\cos (m\pi x)\bigr]_{0}^{1} = \frac{(-1)^{m} - 1}{m^2\pi^2}\\
        \dint{0}{1}{x^2\sin(m\pi x)} &= \left[x^2\frac{-\cos(m\pi x)}{m\pi}\right]_{0}^{1} + \frac{2}{m\pi}\dint{0}{1}{x\cos(m\pi x)}\\
        &= \frac{(-1)^{m + 1}}{m\pi} + \frac{2}{m\pi}\frac{(-1)^{m} - 1}{m^2\pi^2}\\
        (左辺) &= \dint{0}{1}{x\sin(m\pi x)} - \dint{0}{1}{x^2\sin(m\pi x)}\\
        &= \frac{(-1)^{m + 1}}{m\pi} - \left(\frac{(-1)^{m + 1}}{m\pi} + \frac{2}{m\pi}\frac{(-1)^{m} - 1}{m^2\pi^2}\right)\\
        &= \frac{2 - 2(-1)^{m}}{m^3\pi^3}
    \end{align*}
    $(1)$より右辺に関しては以下のようになる。
    \begin{align*}
        (右辺) &= \sum_{n = 1}^{\infty}c_n \dint{0}{1}{\sin(n\pi x)\sin(m\pi x)}\\
        &= \frac{1}{2}c_m
    \end{align*}
    従って、以下が成り立つ。
    \begin{align*}
        \frac{2 - 2(-1)^{m}}{m^3\pi^3} &= \frac{1}{2}c_m\\
        c_m &= \frac{4\left\{1 - (-1)^{m}\right\}}{m^3\pi^3}\\
    \end{align*}
    従って求める解答は以下のようになる。
    \begin{equation*}
        c_n = \frac{4\left\{1 - (-1)^{n}\right\}}{n^3\pi^3}
    \end{equation*}
\end{enumerate}

\newpage
\section{}%第3問
\subsection{問題文}
以下に示す情報システムに関する8項目から4項目を選択し、各項目を4$\thicksim$8行程度で説明せよ。
必要に応じて例や図を用いてよい。

\begin{enumerate}[(1)]
    \item 逆運動学
    \item 隠れマルコフモデル
    \item MinMax法
    \item NP完全問題
    \item レイトレーシング
    \item SIMD(Single Instruction Multiple data)
    \item Call by value(値渡し) and call by reference(参照渡し)
    \item 公開鍵暗号
\end{enumerate}

\newpage

\subsection{解答例}
\begin{enumerate}[(1)]
    \item ロボットのアームの位置座標からロボットのアームの長さとアームの基準面からの角度を求める
    もののことである。つまり、以下の図のように位置座標$(x, y)$から$L, \theta$を求める学問のことである。
    \begin{center}
        \begin{tikzpicture}[>=stealth]
            \draw (0, 0) -- (3, 0);
            \draw (0, 0) -- (0, 3);
            \coordinate (A) at (3, 0);
            \coordinate (B) at (0.3, 0);
            \coordinate (D) at ($(B)!3cm!30:(A)$) node at (D) [above right] {$(x, y)$};
            \coordinate (C) at ($(B)!0.5!(D)$);
            \coordinate (E) at ($(B)!.55!(D)!2.5mm!120:(D)$);
            \draw (E) node [rectangle, minimum height=3cm, minimum width = 0.5cm, rotate = -60, draw] {};
            \draw (0, 0) circle [radius = 3mm];
            \draw (A) -- (B)  pic["$\theta$", <->, very thick, angle eccentricity=1.2, angle radius=1cm,draw=orange] {angle=A--B--C};
            \draw (B).. controls ($(B)!.2!(D)!10pt!90:(D)$) and ($(B)!.8!(D)!10pt!90:(D)$) .. (D) node [midway, sloped, fill=white] {$L$};
        \end{tikzpicture}
    \end{center}    
    \item まずN重マルコフモデルとは前のN個の出力のみに依存する離散確率過程のことを表すモデルである。
    $N=1$の時は単純マルコフモデルと呼ばれる。つまり、単純マルコフモデルは現在の状態のみで次の状態が確率的に決定されるモデルのことである。
    但し、単純マルコフモデルは出力が一つしか認められておらず、複雑な出力は表現できないモデルである。そのため、出力を確率的な表現で表すことにより、
    複数の出力を許すようにしたモデルが隠れマルコフモデルである。
    \item ゲーム理論で使われるアルゴリズムであり、2人零和ゲームであり、有限の手数で終了することが保証されているのであれば、必ず解くことができる
    アルゴリズムである。このアルゴリズムは相手の最善手は自分の最悪手であるという考えのもと、自分は最善の手をだし、相手も最善の手を出すと考えた場
    合に自分の手の場合は最大の評価値を選択し、相手の手の場合は最小の評価値を選択するアルゴリズムである。
    \begin{center}
        \begin{tikzpicture}[>=stealth]
            \draw (0, 0) coordinate (O) rectangle (2, 2) coordinate (A);
            \draw ($(O -| A)!.5!(O)$) coordinate (B) -- ($(O |- A)!.5!(A)$) coordinate (C);
            \draw ($(O -| A)!.5!(A)$) coordinate (D) -- ($(O |- A)!.5!(O)$) coordinate (E);
            \draw (O) node [above right] {$a_{21}$};
            \draw (A) node [below left] {$b_{12}$};
            \draw (B) node [above right] {$a_{22}$};
            \draw (C) node [below left] {$b_{11}$};
            \draw (D) node [below left] {$b_{22}$};
            \draw (E) node [above right] {$a_{11}$};
            \draw ($(B)!.5!(C)$) coordinate (F) node [above right] {$a_{12}$};
            \draw (F) node [below left] {$b_{21}$};
            \draw ($(O |- A)!.5!(C)$) coordinate (G) node [above] {$S_1$};
            \draw ($(A)!.5!(C)$) coordinate (H) node [above] {$S_2$};
            \draw ($(O |- A)!.5!(E)$) coordinate (I) node [left] {$S_1$};
            \draw ($(O)!.5!(E)$) coordinate (J) node [left] {$S_2$};
            \draw ($(I)!.5!(J)$) coordinate (K) node [left = 20] {$P_1$};
            \draw ($(G)!.5!(H)$) coordinate (L) node [above = 20] {$P_2$};
        \end{tikzpicture}
    \end{center}
    よって、上図のような利得表を考えた場合は、2人零和ゲームであるので$a_{ij} = -b_{ij}$であり、$a_{ij}$の値だけ評価すればよく、自分の
    その戦略を選んだ際の評価値の最小値を考えることにより、相手の最善手を考え、その最小値の中で最大となるものの戦略を選択すれば自分の最善手を
    選択することになるので、評価対象$P_1$が取るべき戦略は$S_{\max\limits_{i}\min\limits_{j}a_{ij}}$となるアルゴリズムである。
    \item 
\end{enumerate}
\newpage
\renewcommand{\thesection}{自信のない解法or解けたとこまで}
\renewcommand{\thesubsection}{第\arabic{subsection}問}
\section{}
\subsection{}
\markboth{\thesubsection}{\thesection}
\begin{itemize}
    \item[(6)]解法その1\\
    \setlength{\itemsep}{10pt} 
    行列式が1で2次のユニタリ行列の一般形$P$は設問(5)より、以下のように表せる。
    \begin{align*}
        P = 
        \spalignmat{
            {p_{11}\exp(\imag \psi + \alpha_{11})\cos(\theta + \beta_{11})} {p_{12}\exp(\imag \psi + \alpha_{12})\sin(\theta + \beta_{12})};
            {-p_{21}\exp(-\imag \psi + \alpha_{21})\sin(\theta + \beta_{12})} {p_{22}\exp(-\imag \psi + \alpha_{22})\cos(\theta + \beta_{22})}
        }
    \end{align*}
    よって、この時以下が成り立つ。
    \begin{align*}
        &\mid P \mid = 1\\
        &\Longleftrightarrow p_{11}\exp(\imag \psi + \alpha_{11})\cos(\theta + \beta_{11})\times p_{22}\exp(-\imag \psi + \alpha_{22})\cos(\theta + \beta_{22})\\
        &-(-p_{21}\exp(-\imag \psi + \alpha_{21})\sin(\theta + \beta_{21}))\times p_{12}\exp(\imag \psi + \alpha_{12})\sin(\theta + \beta_{12}) = 1\\
        &\Longleftrightarrow p_{11}p_{22}\exp(\alpha_{11} + \alpha_{22})\cos(\theta + \beta_{11})\cos(\theta + \beta_{22})\\
        &+ p_{21}p_{12}\exp(\alpha_{21} + \alpha_{12})\sin(\theta + \beta_{21}))\sin(\theta + \beta_{12}) = 1
    \end{align*}
    \begin{align*}
        PP^{\ast} = I
    \end{align*}
    以下挫折。。。
    \item[(6)] 解法その2\\
    行列$A$の$(j, k)$成分が$a_{jk}$の時、$A=\{ a_{jk}\}$と表すとき、行列式が1である2次ユニタリ行列
    $A = \{ a_{jk}\} = \{ c_{jk} + \imag d_{jk}\}$を考えると以下の条件を行列$A$は満たす。
    共役転置行列を$A^{\ast} = \{ a^{\ast}_{jk}\} = \{\overline{a_{kj}}\}$とし、それと行列$A$との積を
    $AA^{\ast} = \{ A_{jk}\}$とおく。
    \begin{align*}
        & \begin{cases}
            AA^{\ast} = I\\
            \mid A\mid = 1
        \end{cases}\\
        \Longleftrightarrow & 
        \begin{cases}
            A_{jk} = \sum\limits_{l = 1}^{2} a_{jl}\overline{a_{kl}}
            = 
            \begin{cases}
                0 & j = k\\
                1 & j \neq k  
            \end{cases}\\
            a_{11}a_{22} - a_{12}a_{21} = 1
        \end{cases}\\
        \Longleftrightarrow & 
        \begin{cases}
            A_{11} = a_{11}\overline{a_{11}} + a_{12}\overline{a_{12}} = 1\\
            A_{12} = a_{11}\overline{a_{21}} + a_{12}\overline{a_{22}} = 0\\
            A_{21} = a_{21}\overline{a_{11}} + a_{22}\overline{a_{12}} = 0\\
            A_{22} = a_{21}\overline{a_{21}} + a_{22}\overline{a_{22}} = 1\\
            a_{11}a_{22} - a_{12}a_{21} = 1
        \end{cases}\\
        \Longleftrightarrow & 
        \begin{cases}
            A_{11} = c_{11}^{2} + d_{11}^{2} + c_{12}^{2} + d_{12}^{2} = 1\\
            A_{12} = (c_{11} + \imag d_{11})(c_{21} - \imag d_{21}) + (c_{12} + \imag d_{12})(c_{22} - \imag d_{22}) = 0\\
            A_{21} = (c_{21} + \imag d_{21})(c_{11} - \imag d_{11}) + (c_{22} + \imag d_{22})(c_{12} - \imag d_{12}) = 0\\
            A_{22} = c_{21}^{2} + d_{21}^{2} + c_{22}^{2} + d_{22}^{2} = 1\\
            (c_{11} + \imag d_{11})(c_{22} + \imag d_{22}) - (c_{12} + \imag d_{12})(c_{21} + \imag d_{21}) = 1
        \end{cases}\\
        \Longleftrightarrow & 
        \begin{cases}
            A_{11} = c_{11}^{2} + d_{11}^{2} + c_{12}^{2} + d_{12}^{2} = 1\\
            A_{12} = (c_{11}c_{21} + d_{11}d_{21} + c_{12}c_{22} + d_{12}d_{22}) + \imag (d_{11}c_{21} -  c_{11}d_{21} + d_{12}c_{22} - c_{12}d_{22}) = 0\\
            A_{21} = (c_{21}c_{11} + d_{21}d_{11} + c_{22}c_{12} + d_{22}d_{12}) + \imag(d_{21}c_{11} - c_{21}d_{11} + d_{22}c_{12} - c_{22}d_{12}) = 0\\
            A_{22} = c_{21}^{2} + d_{21}^{2} + c_{22}^{2} + d_{22}^{2} = 1\\
            (c_{11}c_{22} + d_{11}d_{22} - c_{12}c_{21} - d_{12}d_{21}) + \imag (d_{11}c_{22} + c_{11}d_{22} - d_{12}c_{21} - c_{12}d_{21}) = 1
        \end{cases}
    \end{align*}
    よってここで、$\forall k \in \mathbb{Z}$についてオイラーの定理より以下が成り立つ。
    \begin{align*}
        \begin{cases}
            \exp(2\pi\ell\imag) = 1 & \ell = k\\
            \exp(\pi\ell\imag) = -1 & \ell = 2k + 1\\
            \exp\left(\frac{\pi\ell}{2}\imag\right)  = \imag & \ell = 4k + 1\\
            \exp\left(\frac{\pi\ell}{2}\imag\right)  = -\imag & \ell = 4k + 3\\
        \end{cases}
    \end{align*}
    以下挫折。。。
\end{itemize}
\index{ティック@tikz}
% \printindex
\end{document}