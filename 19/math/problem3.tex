\subsection{問題文}
下図のように、平面上に三角形$\mathrm{ABC}$が与えられており、各頂点の座標は
$\mathrm{A(1, 0), B(0, 1), C(-1, -1)}$とする。原点$(0, 0)$を端点とする半直線$\ell$をランダムに選ぶ。
すなわち、$\Theta$を区間$[0, 2\pi)$上の一様分布に従う確率変数として、
\begin{equation*}
    \ell = \left\{(r\cos\Theta, r\sin\Theta) \mid r \geq 0\right\}
\end{equation*}
とおく。この半直線$\ell$と三角形$\mathrm{ABC}$の周との交点を$\mathrm{Q}$とおく。また、$\mathrm{Q}$の座標を
$(X, Y)$とおく。ただし、$X, Y$は確率変数である。以下の問いに答えよ。
\begin{enumerate}[(1)]
    \setlength{\itemsep}{10pt}
    \item 点$\mathrm{Q}$が辺$\mathrm{AB}$上にある確率を求めよ。\label{subsec3:prom1}
    \item 点$\mathrm{Q}$が辺$\mathrm{AB}$上にあるという条件の下での$X$の期待値は$1/2$であることを示せ。但し、
    三角形$\mathrm{ABC}$が直線$y = x$に関して対称であることを利用してもよい。\label{subsec3:prom2}
    \item 点$\mathrm{Q}$が辺$\mathrm{BC}$にあるという条件のもとでの$X$の確率密度関数を、変数変換の公式
    \begin{equation*}
        f(x) = g(h(x))\left\lvert \diff{h}{x}(x) \right\rvert
    \end{equation*}
    を使って求めよ。ただし、$x$は任意の実数とし、$f$と$g$はそれぞれ$X$と$\Theta$の確率密度関数を表し、$h$は$\Theta = h(X)$を満たす関数とする。\label{subsec3:prom3}
    \item 点$\mathrm{Q}$が辺$\mathrm{BC}$にあるという条件のもとでの$X$の期待値を$\alpha$とおく。設問\ref{subsec3:prom3}の結果を用いて$\alpha$
    を求めよ。\label{subsec3:prom4}
    \item $X$の期待値$\mu$を求めよ。
\end{enumerate}
\begin{figure}[htbp]
    \centering
    \begin{tikzpicture}[>=stealth, scale = 1.5]
        \coordinate (O) at (0, 0) node at (O) [below left] {$\mathrm{O}$};
        \coordinate (A) at (2, 0) node at (A) [below] {$\mathrm{A(1, 0)}$};
        \coordinate (B) at (0, 2) node at (B) [right] {$\mathrm{B(0, 1)}$};
        \coordinate (C) at (-2, -2) node at (C) [below] {$\mathrm{C(-1, -1)}$};
        \draw (A) -- (B) -- (C) -- cycle;
        \draw [name path = AB] (A) -- (B);
        \draw [->] ($(O) + (-2.8, 0)$) -- ($(O) + (2.8, 0)$);%x軸
        \draw [->] ($(O) + (0, -2.8)$) -- ($(O) + (0, 2.8)$);%y軸
        \draw [domain = 0:2.5, name path = l] plot(\x, 1/2*\x) node [above] {$\ell$};
        \draw [fill = black, name intersections = {of = l and AB}]
		(intersection-1) circle [radius = 0.5mm] node [right = 0.2cm] {$\mathrm{Q}(X, Y)$};
    \end{tikzpicture}
\end{figure}

\newpage

\subsection{解答例}
\begin{enumerate}[(1)]
    \setlength{\itemsep}{10pt}
    \item 点$\mathrm{Q}$が辺$\mathrm{AB}$にある時、確率変数$\Theta$は以下の範囲に存在する。
    \begin{equation}
        0\, \leq\, \Theta\, \leq\, \frac{\pi}{2}\label{eq:subsec3:prom1:hanni}
    \end{equation}
    ここで、確率変数$\Theta$は区間$[0, 2\pi)$上の一様分布に従う確率変数であるので、
    $\Theta$に関する確率密度関数$f_{\Theta}(\Theta)$は定数$c\, \in \mathbb{R}$を用いて、以下のように定義される。
    \begin{align}
        f_{\Theta}(\Theta) = 
        \begin{cases}
            c & 0\, \leq\, \Theta\, <\, 2\pi\\
            0 & 0 > \Theta,\, \Theta\, \geq\, 2\pi
        \end{cases}\label{eq:subsec3:prom1:ftheta}
    \end{align}
    よって、式\eqref{eq:subsec3:prom1:ftheta}と確率密度関数の定義より、以下が成り立つ。
    \begin{align*}
        \dint[\Theta]{-\infty}{\infty}{f_{\Theta}(\Theta)} = 1\\
        \Longleftrightarrow\; &\lim_{x \to 2\pi-0}\dint[\Theta]{0}{x}{c} = 1\\
        \Longleftrightarrow\; &\lim_{x \to 2\pi-0}cx = 1\\
        \Longleftrightarrow\; &2c\pi = 1\\
        \Longleftrightarrow\; &c = \frac{1}{2\pi}
    \end{align*}
    よって、式\eqref{eq:subsec3:prom1:ftheta}から$f_{\Theta}(\Theta)$は以下のように定義し直すことができる。
    \begin{align}
        f_{\Theta}(\Theta) = 
        \begin{cases}
            \frac{1}{2\pi} & 0\, \leq\, \Theta\, <\, 2\pi\\
            0 & 0 > \Theta,\, \Theta\, \geq\, 2\pi
        \end{cases}\label{eq:subsec3:prom1:f}
    \end{align}
    従って式\eqref{eq:subsec3:prom1:f}, \eqref{eq:subsec3:prom1:hanni}から求める確率$P$は以下のようになる。
    \begin{align*}
        P &= \dint[\Theta]{0}{\cfrac{\pi}{2}}{f_{\Theta}(\Theta)} = \dint[\Theta]{0}{\cfrac{\pi}{2}}{\frac{1}{2\pi}}\\
        &= \frac{1}{2\pi}\frac{\pi}{2} = \frac{1}{4}
    \end{align*}
    よって、求める確率は$\frac{1}{4}$になる。
    \item 設問\eqref{subsec3:prom1}より、辺$\mathrm{AB}$上に点$\mathrm{Q}$が存在するときは$0\, \leq \, \Theta\, \leq \frac{\pi}{2}$を満たす。
   
    また、この時、題意より確率変数$X$について以下が成り立つ。
    
    $\ell$の方程式: 
    \begin{align}
        X = r\cos\Theta, \, Y = r\sin\Theta\, 
        \therefore X\sin\Theta = Y\cos\Theta\label{eq:subsec3:prom2:prex}\\
    \end{align}
    辺$\mathrm{AB}$の方程式:
    \begin{equation*}
        (y - 1) = \frac{0 - 1}{1 - 0}(x - 0)\, \therefore y = -x + 1\, (0\, \leq \, x \, \leq \, 1)
    \end{equation*}
    よって点$\mathrm{Q}$が$\mathrm{AB}$上にあるので
    \begin{equation}
        Y = -X + 1\, (0\, \leq\,  X\, \leq \, 1)\label{eq:subsec3:prom2:tyokkou}\\
    \end{equation}
    式\eqref{eq:subsec3:prom2:prex}, \eqref{eq:subsec3:prom2:tyokkou}より$0\, \leq\, X\, \leq\, 1$において
    \begin{align}
        X\sin\Theta &= (-X + 1)\cos\Theta\nonumber\\
        \therefore \cos\Theta &= X(\sin\Theta + \cos\Theta)\label{eq:subsec3:prom2:inix}
    \end{align}
    ここで$\sin\Theta + \cos\Theta = \sqrt{2}\sin\left(\Theta + \frac{\pi}{4}\right)$であるため、
    $0\, \leq\, \Theta\, \leq \, \frac{\pi}{2}$における範囲は以下のようになる。
    \begin{align}
        \sqrt{2}\times\frac{1}{\sqrt{2}} \leq\, &\sqrt{2}\sin\left(\Theta + \frac{\pi}{4}\right) \, \leq\, \sqrt{2}\nonumber\\
        \Longleftrightarrow 
        1 \leq\, &\sin\Theta + \cos\Theta \, \leq\, \sqrt{2}\label{eq:subsec3:prom2:hanni}
    \end{align}
    よって、式\eqref{eq:subsec3:prom2:inix}, \eqref{eq:subsec3:prom2:hanni}より、$X$は以下のように表せる。
    \begin{equation}
        X = \frac{\cos\Theta}{\sin\Theta + \cos\Theta} = \frac{\cos\Theta}{\sqrt{2}\sin\left(\Theta + \frac{\pi}{4}\right)}\label{eq:subsec3:prom2:x}
    \end{equation}
    ここで$0\, \leq\, \Theta\, \leq\, \frac{\pi}{2}$において、
    $X$を$\Theta$の関数$F(\Theta)$として考えると以下が成り立つ。
    \begin{align*}
        \diff{F}{\Theta}(\Theta) 
        &= \frac{-\cos\left\{\Theta - \left(\Theta + \frac{\pi}{4}\right)\right\}}{\sqrt{2}\sin^{2}\left(\Theta + \frac{\pi}{4}\right)}\\
        &= -\frac{1}{2\sin^{2}\left(\Theta + \frac{\pi}{4}\right)} < 0
    \end{align*}
    よって、$0\, \leq\, \Theta\, \leq\, \frac{\pi}{2}$において、$X = F(\Theta)$は単調減少することが分かる。
    従って、$0\, \leq\, \Theta\, \leq\, \frac{\pi}{2}$において、ある$\Theta$に対応する$X$の値はただ一つしか存在
    しないので、$X$も$0\, \leq\, X\, \leq\, 1$において一様分布に従う。よって、この区間における確率密度関数$f_{X_{01}}(X)$は以下のようになる。
    \begin{align*}
        f_{X_{01}}(X) = 
        \begin{cases}
            c_{X} & 0\, \leq\, X\, \leq\, 1\\
            0 & X < 0, \, X > 1    
        \end{cases}
    \end{align*}
    よって、確率密度関数の性質より
    \begin{align*}
        \dint[X]{-\infty}{\infty}{f_{X_{01}}(X)} &= 1\\
        \dint[X]{0}{1}{f_{X_{01}}(X)} &= 1\\
        \dint[X]{0}{1}{c_{X}} &= 1\\
        c_{X} = 1
    \end{align*}
    従って、点$\mathrm{Q}$が辺$\mathrm{AB}$上にある条件の下での$X$の期待値$E_{X}$は以下のようになる。
    \begin{align*}
        E_{X} &= \dint[X]{-\infty}{\infty}{Xf_{X_{01}}(X)}\\
        &= \dint[X]{0}{1}{X}\\
        &= \frac{1}{2}
    \end{align*}
    よって、題意は示された。
    \item 以降$\arctan x$の定義域は$-\frac{\pi}{2} < x < \frac{\pi}{2}$であるとする。
    
    点$\mathrm{Q}$が辺$\mathrm{BC}$上にあるので、確率変数$X$に関して以下が成り立つ。
    
    辺$\mathrm{BC}$の方程式:
    \begin{equation*}
        (y - 1) = \frac{1 + 1}{0 + 1}(x - 0)\, \therefore y = 2x + 1
    \end{equation*}
    よって点$\mathrm{Q}$が$\mathrm{BC}$上にあるので$-1\, \leq\, X\, \leq\, 0$において
    \begin{equation}
        Y = 2X + 1\label{eq:subsec3:prom3:tyokkou}
    \end{equation}
    点$\mathrm{Q}$は半直線$\ell$上の点でもあるので、式\eqref{eq:subsec3:prom2:prex}, \eqref{eq:subsec3:prom3:tyokkou}より$-1\, \leq\, X\, \leq\, 0$において
    \begin{align}
        X\sin\Theta &= \left(2X + 1\right)\cos\Theta\nonumber\\
        &\therefore
        \begin{cases}
            \Theta = \frac{\pi}{2} & X = 0\\
            \sin\Theta = \frac{2X + 1}{X}\cos\Theta & -1\, \leq\, X < 0
        \end{cases}\label{eq:subsec3:prom3:theta1}
    \end{align}
    よって、$X \neq 0$の時$\Theta \neq \frac{\pi}{2}$であるので$\cos\Theta \neq 0$
    であり、式\eqref{eq:subsec3:prom3:theta1}より$\Theta$について点$\mathrm{Q}$が辺$\mathrm{BC}$に
    存在する場合、$\frac{\pi}{2} \, \leq\, \Theta\, \leq\, \frac{5\pi}{4}$より以下が成り立つ。
    \begin{align}
        &\begin{cases}
            \Theta = \frac{\pi}{2} & X = 0\\
            \tan\Theta = \frac{2X + 1}{X} & -1\, \leq\, X < 0\\
        \end{cases}\nonumber\\
        \Longleftrightarrow&
        \begin{cases}
            \Theta = \frac{\pi}{2} & X = 0\\
            \Theta = \pi + \arctan\left(\frac{2X + 1}{X}\right) & -1\, \leq\, X < 0\\
        \end{cases}\label{eq:subsec3:prom3:theta2}
    \end{align}
    $\Theta$は区間$[0, 2\pi)$において一様分布に従うので確率密度関数$g(\Theta)$は式
    \eqref{eq:subsec3:prom1:f}であるため、以下が成り立つ。
    \begin{equation*}
        g(\Theta) = \frac{1}{2\pi}
    \end{equation*}
    $\frac{\pi}{2} < \Theta \leq \frac{5\pi}{4}$において、つまり$ -1\, \leq\, X < 0$の時、
    式\eqref{eq:subsec3:prom3:theta2}より確率密度関数$f(x)$は変数変換の公式を用いて以下のようになる
    \begin{align}
        \Theta = h(X) &= \pi + \arctan\left(\frac{2X + 1}{X}\right) \nonumber\\
        \therefore \diff{h}{x}(x) &= \frac{1}{1 + \left(\frac{2x + 1}{x}\right)^2}\frac{2x - (2x + 1)}{x^2}\nonumber\\
        &= \frac{-x^2}{\left\{\left(2x + 1\right)^2 + x^2\right\}x^2}\nonumber\\
        &= \frac{-1}{\left(2x + 1\right)^2 + x^2}\nonumber\\
        \Longleftrightarrow 
        f(x) &= \frac{1}{2\pi}\frac{-1}{\left(2x + 1\right)^2 + x^2}\label{eq:subsec3:prom3:fx:nez}
    \end{align}
    $\Theta = \frac{\pi}{2}$において、つまり$X = 0$の時、
    式\eqref{eq:subsec3:prom3:theta2}より確率密度関数$f(x)$は変数変換の公式を用いて以下のようになる
    \begin{align}
        \Theta = h(0) &= \frac{\pi}{2}\nonumber\\
        \diff{h}{x} &= 0\nonumber\\
        \Longleftrightarrow f(0) &= 0\label{eq:subsec3:prom3:fx:ez}
    \end{align}
    よって、式\eqref{eq:subsec3:prom3:fx:nez},\eqref{eq:subsec3:prom3:fx:ez}より、求める確率密度関数$f(x)$は以下のようになる。
    \begin{equation*}
        f(x) = 
        \begin{cases}
            0 & x = 0\\
            \frac{-1}{2\pi\left\{\left(2x + 1\right)^2 + x^2\right\}} & -1\, \leq\, x < 0
        \end{cases}
    \end{equation*}
    \item 設問\eqref{subsec3:prom3}より辺$\mathrm{BC}$上に点$\mathrm{Q}$
    が存在する時の$X$の期待値$\alpha$は以下のようになる。
    \begin{align*}
        \alpha &= \dint{-\infty}{\infty}{xf(x)}\\
        &= \lim_{t \to -0}\dint{-1}{t}{xf(x)}\\
        &= \lim_{t \to -0}\dint{-1}{t}{\frac{-x}{2\pi\left\{\left(2x + 1\right)^2 + x^2\right\}}}\\
        &= \lim_{t \to -0}\frac{-1}{20\pi}\dint{-1}{t}{\frac{10x + 4 - 4}{\left(2x + 1\right)^2 + x^2}}\\
        &= \lim_{t \to -0}\frac{-1}{20\pi}\left\{\left[\log\left\{\left(2x + 1\right)^2 + x^2\right\}\right]_{-1}^{t} - \dint{-1}{t}{\frac{4}{\left(2x + 1\right)^2 + x^2}}\right\}\\
        &= \lim_{t \to -0}\left(\frac{-1}{20\pi}\left[\log\left\{\left(2t + 1\right)^2 + t^2\right\} - \log 2\right] + \frac{1}{5\pi}\dint{-1}{t}{\frac{1}{\frac{1}{5}\left\{\left(5x + 2\right)^2 + 1\right\}}}\right)\\
        &= \frac{\log 2}{20\pi} + \lim_{t \to -0}\frac{1}{\pi}\dint{-1}{t}{\frac{1}{\left(5x + 2\right)^2 + 1}}
    \end{align*}
    ここで$5x + 2 = \tan u$と置換し, $\beta, \gamma$を$\tan\beta = -3, \tan\gamma = 5t + 2$を満たすとものとしておくと以下のようになる。
    \begin{align*}
        \alpha &= \frac{\log 2}{20\pi} + \lim_{t \to -0}\frac{1}{\pi}\dint[u]{\beta}{\gamma}{\frac{1}{\tan^{2}u+ 1}\frac{1}{5\cos^{2}u}}\\
        &= \frac{\log 2}{20\pi} + \lim_{t \to -0}\frac{1}{5\pi}(\gamma - \beta)\\
        &= \frac{\log 2}{20\pi} + \lim_{t \to -0}\frac{1}{5\pi}\bigl\{(\pi + \arctan(5t + 2)) - (\pi + \arctan(-3))\bigr\}\\
        &= \frac{\log 2}{20\pi} + \frac{1}{5\pi}(\arctan 2 + \arctan 3)
    \end{align*}
    よって、期待値$\alpha$について以下のようになる。
    \begin{equation}
        \alpha = \frac{\log 2}{20\pi} + \frac{1}{5\pi}(\arctan 2 + \arctan 3)\label{eq:subsec3:prom4:ans}
    \end{equation}
    \item まず、点$\mathrm{Q}$が辺$\mathrm{AC}$上にある時の$X$の期待値$\delta$を求める。
    
    設問\eqref{subsec3:prom4}, \eqref{subsec3:prom3}と同様にして確率密度関数を求めてから期待値を求める。

    この時、確率変数$X$について以下のことが成り立つ。
    
    辺$\mathrm{AC}$の方程式:
    \begin{equation*}
        (y - 0) = \frac{0 + 1}{1 + 1}(x - 1)\, \therefore y = \frac{1}{2}(x - 1)
    \end{equation*}
    よって点$\mathrm{Q}$が$\mathrm{AC}$上にあるので$-1\, \leq\,  X\, \leq \, 1$において
    \begin{equation}
        Y = \frac{1}{2}(X - 1)\label{eq:subsec3:prom4:tyokkou}
    \end{equation}
    点$\mathrm{Q}$は半直線$\ell$上の点でもあるので、式\eqref{eq:subsec3:prom2:prex}, 
    \eqref{eq:subsec3:prom4:tyokkou}より$-1\, \leq\,  X\, \leq \, 1$において
    \begin{align}
        X\sin\Theta &= \frac{1}{2}\left(X - 1\right)\cos\Theta\nonumber\\
        &\therefore
        \begin{cases}
            \Theta = \frac{3\pi}{2} & X = 0\\
            \sin\Theta = \frac{X - 1}{2X}\cos\Theta & -1\, \leq\, X < 0,\, 0 < X\, \leq\, 1
        \end{cases}\label{eq:subsec3:prom4:theta1}
    \end{align}
    よって、$X \neq 0$の時$\Theta \neq \frac{3\pi}{2}$であるので$\cos\Theta \neq 0$であり、
    式\eqref{eq:subsec3:prom4:theta1}より$\Theta$について点$\mathrm{Q}$が辺$\mathrm{AC}$
    に存在する場合、つまり、$-1\, \leq\, X\, \leq\, 1$において以下が成り立つ。
    \begin{align}
        &\begin{cases}
            \Theta = \frac{3\pi}{2} & X = 0\\
            \tan\Theta = \frac{X - 1}{2X} & -1\, \leq\, X < 0,\, 0 < X\, \leq\, 1\\
        \end{cases}\nonumber\\
        \Longleftrightarrow&
        \begin{cases}
            \Theta = \frac{3\pi}{2} & X = 0\\
            \Theta = \pi + \arctan\left(\frac{X - 1}{2X}\right) & 0 < X \, \leq\, 1\\
            \Theta = 2\pi + \arctan\left(\frac{X - 1}{2X}\right) & -1\, \leq\, X < 0\\
        \end{cases}\label{eq:subsec3:prom4:theta2}
    \end{align}
    ここで$\Theta$は区間$[0, 2\pi)$において一様分布に従うので確率密度関数$g(\Theta)$は式
    \eqref{eq:subsec3:prom1:f}であるため、以下が成り立つ。
    \begin{align*}
        g(\Theta) &= \frac{1}{2\pi}
    \end{align*}
    よって、$\frac{5\pi}{4}\, \leq\, \Theta < \frac{3\pi}{2}$において、つまり $0 < X\, \leq\, 1$の時、
    式\eqref{eq:subsec3:prom4:theta2}より、確率密度関数$f(x)$は変数変換の公式を用いて以下のようになる。
    \begin{align}
        \Theta = h(X) &= \pi + \arctan\left(\frac{X - 1}{2X}\right) \nonumber\\
        \therefore \diff{h}{x}(x) &= \frac{1}{1 + \left(\frac{x - 1}{2x}\right)^2}\frac{x - (x - 1)}{2x^2}\nonumber\\
        &= \frac{4x^2}{\left\{\left(x - 1\right)^2 + 4x^2\right\}2x^2}\nonumber\\
        &= \frac{2}{\left(x - 1\right)^2 + 4x^2}\nonumber\\
        \Longleftrightarrow f(x) &= \frac{1}{2\pi}\frac{2}{\left(x - 1\right)^2 + 4x^2}\label{eq:subsec3:prom5:fx:gz}
    \end{align}
    $\frac{3\pi}{2} < \Theta < 2\pi$において、つまり$-1\, \leq\, X < 0$の時、
    式\eqref{eq:subsec3:prom4:theta2}より、確率密度関数$f(x)$は変数変換の公式を用いて以下のようになる。
    \begin{equation*}
        \Theta = h(X) = 2\pi + \arctan\left(\frac{X - 1}{2X}\right)
    \end{equation*}
    $0 < X\, \leq\, 1$の時と同様にして
    \begin{align}
        \diff{h}{x}(x) &= \frac{2}{\left(x - 1\right)^2 + 4x^2}\nonumber\\
        \Longleftrightarrow f(x) &= \frac{1}{2\pi}\frac{2}{\left(x - 1\right)^2 + 4x^2}\label{eq:subsec3:prom5:fx:lz}
    \end{align}
    $\Theta = \frac{3\pi}{2}$において、つまり$X = 0$の時、
    式\eqref{eq:subsec3:prom4:theta2}より、確率密度関数$f(x)$は変数変換の公式を用いて以下のようになる。
    \begin{align}
        h(0) = \Theta &= \frac{3\pi}{2}\nonumber\\
        \diff{h}{x} &= 0\nonumber\\
        \Longleftrightarrow f(0) &= 0\label{eq:subsec3:prom5:fx:nez}
    \end{align}
    よって、式\eqref{eq:subsec3:prom5:fx:gz}, \eqref{eq:subsec3:prom5:fx:lz}, \eqref{eq:subsec3:prom5:fx:nez}から確率密度関数は以下のようになる。
    \begin{equation*}
        f(x) = 
        \begin{cases}
            0 & x = 0\\
            \frac{1}{\pi\left\{\left(x - 1\right)^2 + 4x^2\right\}} & -1\, \leq\, x < 0,\, 0 < x\, \leq\, 1 
        \end{cases}
    \end{equation*}
    従って期待値は以下のようになる。
    \begin{align*}
        \delta &= \dint{-\infty}{\infty}{xf(x)}\\
        &= \lim_{t \to -0}\dint{-1}{t}{xf(x)} + \lim_{t \to +0}\dint{t}{1}{xf(x)}\\
        &= \lim_{t \to -0}\dint{-1}{t}{\frac{x}{\pi\left\{(x - 1)^2 + 4x^2\right\}}}
         + \lim_{t \to +0}\dint{t}{1}{\frac{x}{\pi\left\{(x - 1)^2 + 4x^2\right\}}}\\
        &= \lim_{t \to -0}\frac{1}{10\pi}\dint{-1}{t}{\frac{10x - 2 + 2}{(x - 1)^2 + 4x^2}} 
         + \lim_{t \to +0}\frac{1}{10\pi}\dint{t}{1}{\frac{10x - 2 + 2}{(x - 1)^2 + 4x^2}}\\
        &= \lim_{t \to -0}\frac{1}{10\pi}\left\{\left[\log\left\{(x - 1)^2 + 4x^2\right\}\right]_{-1}^{t} + \dint{-1}{t}{\frac{2}{(x - 1)^2 + 4x^2}}\right\}\\
        &\quad + \lim_{t \to +0}\frac{1}{10\pi}\left\{\left[\log\left\{(x - 1)^2 + 4x^2\right\}\right]_{t}^{1} + \dint{t}{1}{\frac{2}{(x - 1)^2 + 4x^2}}\right\}\\
        &= \lim_{t \to -0}\left(\frac{1}{10\pi}\left[\log\left\{(t - 1)^2 + 4t^2\right\} - 3\log 2\right] + \frac{1}{5\pi}\dint{-1}{t}{\frac{1}{\frac{1}{5}\left\{(5x - 1)^2 + 4\right\}}}\right)\\
        &\quad + \lim_{t \to +0}\left(\frac{1}{10\pi}\left[2\log 2 - \log\left\{(t - 1)^2 + 4t^2\right\}\right] + \frac{1}{5\pi}\dint{t}{1}{\frac{1}{\frac{1}{5}\left\{(5x - 1)^2 + 4\right\}}}\right)\\
        &= \frac{-3\log 2}{10\pi} + \frac{2\log 2}{10\pi} + \lim_{t \to -0}\frac{1}{\pi}\dint{-1}{t}{\frac{1}{(5x - 1)^2 + 4}} + \lim_{t \to +0}\frac{1}{\pi}\dint{t}{1}{\frac{1}{(5x - 1)^2 + 4}}\\
        &= \frac{-\log 2}{10\pi} + \lim_{t \to -0}\frac{1}{\pi}\dint{-1}{t}{\frac{1}{(5x - 1)^2 + 4}} + \lim_{t \to +0}\frac{1}{\pi}\dint{t}{1}{\frac{1}{(5x - 1)^2 + 4}}
    \end{align*}
    ここで$5x - 1 = 2\tan u$と置換し, $\beta,\, \gamma_1(< \frac{3\pi}{2}),\, \gamma_2(> \frac{3\pi}{2}),\, \eta$を$\tan\beta = -3, \tan\gamma_1 = \tan\gamma_2 = \frac{5t - 1}{2}, \tan\eta = 2$を満たすとものとしておくと以下のようになる。
    \begin{align*}
        \delta 
        &= \frac{-\log 2}{10\pi} + \lim_{t \to -0}\frac{1}{\pi}\dint[u]{\beta}{\gamma_1}{\frac{1}{4\tan^{2}u + 4}\frac{2}{5\cos^{2}u}}
         + \lim_{t \to +0}\frac{1}{\pi}\dint[u]{\gamma_2}{\eta}{\frac{1}{4\tan^{2}u + 4}\frac{2}{5\cos^{2}u}}\\
        &= \frac{-\log 2}{10\pi} + \lim_{t \to -0}\frac{1}{\pi}\dint[u]{\beta}{\gamma_1}{\frac{1}{10}} 
         + \lim_{t \to +0}\frac{1}{\pi}\dint[u]{\gamma_2}{\eta}{\frac{1}{10}}\\
        &= \frac{-\log 2}{10\pi} + \lim_{t \to -0}\frac{1}{10\pi}(\gamma_1 - \beta) + \lim_{t \to -0}\frac{1}{\pi}(\eta - \gamma_2)\\
        &= \frac{-\log 2}{10\pi} + \frac{1}{10\pi}\left\{\left(\pi + \arctan\left(\frac{-1}{2}\right)\right) - (\pi + \arctan(-3))
        + (2\pi + \arctan 2) - \left(2\pi + \arctan\left(\frac{-1}{2}\right)\right)\right\}\\
        &= \frac{-\log 2}{10\pi} + \frac{1}{10\pi}(\arctan 2 + \arctan 3)
    \end{align*}
    よって、期待値$\delta$について以下のようになる。
    \begin{equation}
        \delta = \frac{-\log 2}{10\pi} + \frac{1}{10\pi}(\arctan 2 + \arctan 3)\label{eq:subsec3:prom5:delta}
    \end{equation}
    よって、設問\eqref{subsec3:prom2},式\eqref{eq:subsec3:prom4:ans}, \eqref{eq:subsec3:prom5:delta}
    より点$\mathrm{Q}$が辺$\mathrm{AB}$にある状態、辺$\mathrm{AC}$にある状態、辺$\mathrm{BC}$にある状態での$X$の期待値を
    それぞれの状態における離散的な確率変数として考えるとそれぞれの出現確率が$\frac{1}{4}, \frac{3}{8}, \frac{3}{8}$であるので
    $X$の期待値$\mu$は以下のようになる。
    \begin{align*}
        \mu &= \frac{1}{4}E_{X} + \frac{3}{8}\alpha + \frac{3}{8}\delta\\
        &= \frac{1}{4}\cdot\frac{1}{2} 
        + \frac{3}{8}\left\{\frac{\log 2}{20\pi} + \frac{1}{5\pi}(\arctan 2 + \arctan 3)\right\}
        + \frac{3}{8}\left\{\frac{-\log 2}{10\pi} + \frac{1}{10\pi}(\arctan 2 + \arctan 3)\right\}\\
        &= \frac{1}{8}\left\{1 - \frac{3\log 2}{20\pi} + \frac{9}{10\pi}(\arctan 2 + \arctan 3)\right\}
    \end{align*}
\end{enumerate}