% \documentclass[dvipdfmx, a4paper, 11pt]{jsarticle}%A4用紙縦、明朝(デフォルト)11ポイント
\documentclass[dvipdfmx,titlepage, 11pt, a4paper]{jsarticle}%A4用紙縦、明朝(デフォルト)11ポイント
\usepackage[top=18truemm,bottom=18truemm,left=18truemm,right=18truemm]{geometry}%余白調整
% \setlength{\textheight}{45\baselineskip}
% \setlength{\textwidth}{46zw}% 46文字/行

\usepackage{template}
%%%====================================================================================================================================
\renewcommand{\thesection}{第\arabic{section}問}
\renewcommand{\thesubsection}{\thesection}
\titleformat*{\section}{\LARGE\mcfamily}%章のタイトルの文字の大きさを通常サイズに設定(明朝体で)
\titlespacing*{\section}{0pt}{*0}{0pt}%章番号の後の空白行の削除
\titleformat*{\subsection}{\Large\mcfamily}%節のタイトルの文字の大きさを通常サイズに設定(明朝体で)
\titlespacing*{\subsection}{0pt}{*0}{0pt}%節番号の後の空白行の削除
\titleformat*{\subsubsection}{\large\mcfamily}%小節のタイトルの文字の大きさを通常サイズに設定(明朝体で)
\titlespacing*{\subsubsection}{0pt}{*0}{0pt}%小節番号の後の空白行の削除

\makeatletter
%章番号付きコード番号
\AtBeginDocument{
  \renewcommand*{\thelstlisting}{\arabic{section}.\arabic{lstlisting}}%
  \@addtoreset{lstlisting}{section}
}

%章番号付き表番号
\renewcommand{\thetable}{%
    \arabic{section}.\arabic{table}%
}
\@addtoreset{table}{section}%

%章番号付き図番号
\renewcommand{\thefigure}{%
	\arabic{section}.\arabic{figure}%
}%
\@addtoreset{figure}{section}%

%章番号付き式番号
\renewcommand{\theequation}{%
	\arabic{section}.\arabic{equation}%
}%
\@addtoreset{equation}{section}%

\renewcommand{\p@enumiii}{}% 箇条書きの参照時の番号の変更(2(親の番号)a(子の番号)→aだけに)
\renewcommand{\p@enumii}{}% 箇条書きの参照時の番号の変更(2a→aだけに)
\makeatother
%%%====================================================================================================================================

%%%====================================================================================================================================
%%ページのレイアウト設定
\pagestyle{fancy}
\renewcommand{\sectionmark}[1]{\markboth{}{\thesection\ #1}}%                   %\rightmarkにセクション名を格納
\renewcommand{\subsectionmark}[1]{\markboth{#1}{\rightmark}}%   %\leftmarkにサブセクション名を格納
%[]は省略可で省略すると{}で指定された内容が偶数ページ奇数のどちらにも適用される。
% \renewcommand{\headrulewidth}{0pt} %ヘッダの罫線を消す 
\fancyfoot{}%                        %clear all footer fields
\lhead{\leftmark}%                   %左側ヘッダの定義[<偶数ページ>]{<奇数ページ>}
\chead{}%                            %中央ヘッダの定義[<偶数ページ>]{<奇数ページ>}
\rhead{\rightmark}%                  %右側ヘッダの定義[<偶数ページ>]{<奇数ページ>}
\lfoot{東大2019年度創造情報学専門解答例}%       %左側フッターの定義[<偶数ページ>]{<奇数ページ>}
\cfoot{\thepage}%                    %中央フッターの定義[<偶数ページ>]{<奇数ページ>}
\rfoot{文殊の知恵}%                  %右側フッターの定義[<偶数ページ>]{<奇数ページ>}
\renewcommand{\headrulewidth}{0.1pt}%%ヘッダの線の太さ 
\renewcommand{\footrulewidth}{0.1pt}%%フッターの線の太さ
%%%====================================================================================================================================

\makeindex%索引用

\renewcommand{\refname}{}%参考文献の文字を非表示にする
\title{\Huge 東大2019年度創造情報学専門解答例\\[10mm]}
\author{{\LARGE 文殊の知恵}\\[1mm]\LARGE 高橋那弥}
\date{}

\begin{document}
\maketitle
\tableofcontents % 目次
\pagenumbering{roman}%目次のページ番号のスタイルをローマ数字にする
\newpage
\setcounter{tocdepth}{3}%章節の深さを3にするsubsubsectionまで
\pagenumbering{arabic}%他のページ番号は通常の数字にする。
\section{}%第1問
\subsection{問題文}
3次元ベクトル$\spalignmat[c]{
    {x_n};
    {y_n};
    {z_n}
}$は式
\begin{equation*}
    \spalignmat[c]{
        {x_{n + 1}};
        {y_{n + 1}};
        {z_{n + 1}}
    }
    = 
    A \spalignmat[c]{
        {x_n};
        {y_n};
        {z_n}
    }\qquad (n \, = \, 0, 1, 2, \cdots)
\end{equation*}
を満たすものとする。但し、$x_0, y_0, z_0, \alpha$は実数とし、
\begin{equation*}
    A = \spalignmat[c]{
        {1 - 2\alpha} {\alpha} {\alpha};
        {\alpha} {1 - \alpha} {0};
        {\alpha} {0} {1 - \alpha}
    }, \qquad 0 < \alpha < \frac{1}{3}
\end{equation*}
とする。以下の問いに答えよ。
\begin{enumerate}[(1)]
  \item $x_n + y_n + z_n$を$x_0, y_0, z_0$を用いて表せ。
  \item 行列$A$の固有値$\lambda_1, \lambda_2, \lambda_3$と、それぞれの固有値に対応する固有ベクトル
        $\mathbold{v}_1, \mathbold{v}_2, \mathbold{v}_3$を求めよ。
  \item 行列$A$を$\lambda_1, \lambda_2, \lambda_3, \mathbold{v}_1, \mathbold{v}_2, \mathbold{v}_3$を用いて表せ。
  \item $\spalignmat[c]{
    {x_n};
    {y_n};
    {z_n}
  }$を$x_0, y_0, z_0, \alpha$を用いて表せ。
  \item $\lim\limits_{n \to \infty} \spalignmat[c]{
      {x_n};
      {y_n};
      {z_n}
  }$を求めよ。
  \item 以下の式
  \begin{equation*}
	  f(x_0, y_0, z_0)\, = \, \frac{
		  (x_n\;y_n\; z_n)
	  \spalignmat[c]{
		  {x_{n + 1}};
		  {y_{n + 1}};
		  {z_{n + 1}}
	  }}{(x_n\;y_n\; z_n)\spalignmat[c]{
		  {x_n};
		  {y_n};
		  {z_n}
	  }}
  \end{equation*}
  を$x_0, y_0, z_0$の関数とみなして、$f(x_0, y_0, z_0)$の最大値及び最小値を求めよ。但し、
  $x_0^{2} + y_0^{2} + z_0^{2}\, \neq \, 0$
\end{enumerate}
\newpage
\subsection{解答}
\begin{enumerate}[(1)]
	\item 題意より以下が成り立つ。
		\begin{align*}
			\spalignmat[c]{
				{x_{n + 1}};
				{y_{n + 1}};
				{z_{n + 1}}
			} 
			& = A
			\spalignmat[c]{
				{x_n};
				{y_n};
				{z_n}
			}\\
			& = \spalignmat[c]{
				{1 - 2\alpha} {\alpha} {\alpha};
				{\alpha} {1 - \alpha} {0};
				{\alpha} {0} {1 - \alpha}
			}
			\spalignmat[c]{
				{x_n};
				{y_n};
				{z_n}
			}\\
			& = \spalignmat[c]{
				{(1 - 2\alpha)x_n + \alpha y_n + \alpha z_n};
				{\alpha x_n + (1 - \alpha)y_n};
				{\alpha x_n + (1 - \alpha)z_n}
			}\\
			\therefore
			x_{n + 1} + y_{n + 1} + z_{n + 1} 
			&=  (1 - 2\alpha)x_n + \alpha y_n + \alpha z_n + \alpha x_n + (1 - \alpha)y_n + \alpha x_n + (1 - \alpha)z_n\\
			&=  x_n + y_n + z_n\\
			\therefore
			x_n + y_n + z_n &= x_0 + y_0 + z_0
		\end{align*}
	\item まずAの固有値を求める。
		\begin{align*}
			\spaligndelims\vert\vert \spalignmat{{\lambda I - A}}
			& = \spaligndelims\vert\vert \spalignmat{
			{\lambda - 1 + 2\alpha} {-\alpha} {-\alpha};
			{-\alpha} {\lambda - 1 + \alpha} {0};
			{-\alpha} {0} {\lambda - 1 + \alpha};
			}\\
			& = \frac{1}{\alpha}\spaligndelims\vert\vert \spalignmat{
			{(\lambda - 1 + 2\alpha)} {-\alpha} {-\alpha};
			{-\alpha} {\lambda - 1 + \alpha} {0};
			{-\alpha^{2} + (\lambda - 1 + 2\alpha)(\lambda - 1 + \alpha)} {-\alpha(\lambda - 1 + \alpha)} {0};
			}\\
			& = (-1)^{1 + 3}(-1)\spaligndelims\vert\vert \spalignmat{
			{-\alpha} {\lambda - 1 + \alpha};
			{-\alpha^{2} + (\lambda - 1 + 2\alpha)(\lambda - 1 + \alpha)} {-\alpha(\lambda - 1 + \alpha)};
			}\\
			& = -\left\{\alpha^{2}(\lambda - 1 + \alpha) - (\lambda - 1 + \alpha)(-\alpha^{2} + (\lambda - 1 + 2\alpha)(\lambda - 1 + \alpha))\right\} \\
			& = -(\lambda - 1 + \alpha)\left\{\alpha^{2} - (-\alpha^{2} + (\lambda - 1 + 2\alpha)(\lambda - 1 + \alpha))\right\}                       \\
			& = -(\lambda - 1 + \alpha)\left\{ 2\alpha^{2} - (\lambda - 1)^2 - 2\alpha^2 - 3\alpha(\lambda - 1))\right\}                               \\
			& = (\lambda - 1 + \alpha)(\lambda - 1)(\lambda - 1 + 3\alpha)
		\end{align*}
		よって、$\lambda = 1 - \alpha, 1, 1 - 3\alpha$より、$\lambda_1 = 1 - \alpha, \lambda_2 = 1, \lambda_3 = 1 - 3\alpha$\\
		従って固有空間$W(\lambda; A) = \{\mathbold{v} | \left(\lambda I - A\right)\mathbold{v} = \mathbold{0}\}$は以下のようになる。
		\begin{align*}
			W(1 - \alpha; A)  
			% & = \left\{\mathbold{v} \left\lvert \spalignmat{
			% {\alpha} {-\alpha} {-\alpha};
			% {-\alpha} {0} {0};
			% {-\alpha} {0} {0}
			% }\mathbold{v} = \mathbold{0}\right. \right\}\\
			% & = \left\{\mathbold{v} \left\lvert \spalignmat{
			% {1} {0} {0};
			% {0} {-1} {-1};
			% {0} {0} {0}
			% }\mathbold{v} = \mathbold{0}\right. \right\}\\
			& = \left\{\mathbold{v} \left\lvert
			\mathbold{v} =\spalignmat{
			{0};
			{s};
			{-s}
			} = s\spalignmat{
			{0};
			{1};
			{-1}
			}\right. \right\}
			\mbox{よって、}\mathbold{v}_1 = s\spalignmat{
			{0};
			{1};
			{-1}
			}\\
			W(1; A)
			% & = \left\{\mathbold{v} \left\lvert \spalignmat{
			% {2\alpha} {-\alpha} {-\alpha};
			% {-\alpha} {\alpha} {0};
			% {-\alpha} {0} {\alpha}
			% }\mathbold{v} = \mathbold{0}\right. \right\}\\
			% & = \left\{\mathbold{v} \left\lvert \spalignmat{
			% {0} {0} {0};
			% {-1} {1} {0};
			% {-1} {0} {1}
			% }\mathbold{v} = \mathbold{0}\right. \right\}\\
			& = \left\{\mathbold{v} \left\lvert
			\mathbold{v} =\spalignmat{
			{s};
			{s};
			{s}
			} = s\spalignmat{
			{1};
			{1};
			{1}
			}\right. \right\}
			\mbox{よって、}\mathbold{v}_2 = t\spalignmat{
			{1};
			{1};
			{1}
			}\\
			W(1 - 3\alpha; A)
			% & = \left\{\mathbold{v} \left\lvert \spalignmat{
			% {-\alpha} {-\alpha} {-\alpha};
			% {-\alpha} {-2\alpha} {0};
			% {-\alpha} {0} {-2\alpha}
			% }\mathbold{v} = \mathbold{0}\right. \right\}\\
			% & = \left\{\mathbold{v} \left\lvert \spalignmat{
			% {1} {2} {0};
			% {0} {1} {-1};
			% {0} {0} {0}
			% }\mathbold{v} = \mathbold{0}\right. \right\}\\
			& = \left\{\mathbold{v} \left\lvert
			\mathbold{v} =\spalignmat{
			{-2s};
			{s};
			{s}
			} = s\spalignmat{
			{-2};
			{1};
			{1}
			}\right. \right\}
			\mbox{よって、}\mathbold{v}_3 = u\spalignmat{
			{-2};
			{1};
			{1}
			}
		\end{align*}
		よって、$\lambda_1 = 1 - \alpha, \lambda_2 = 1, \lambda_3 = 1 - 3\alpha, 
		\mathbold{v}_1 = s\spalignmat[c]{
			   {0};
			   {1};
			   {-1}
		},
		\mathbold{v}_2 = t\spalignmat[c]{
			   {1};
			   {1};
			   {1}
		},
		\mathbold{v}_3 = u\spalignmat[c]{
			   {-2};
			   {1};
			   {1}
		}$\\
		但し、$s, t, u$は0でない任意の実数とする。
	\item  題意より、$A$は対称行列であるため、直交行列で対角化できる。また、異なる固有値の基底ベクトルは互いに直交する。
		よって、以下は正規直交基底ベクトル集合$W$となる。
		\begin{align*}
			W = \left\{\cfrac{\mathbold{v}_1}{\left\lvert \mathbold{v}_1\right\rvert}, \, \cfrac{\mathbold{v}_2}{\left\lvert \mathbold{v}_2\right\rvert}, \, \cfrac{\mathbold{v}_3}{\left\lvert \mathbold{v}_3\right\rvert}\right\}
		\end{align*}
		よって、この集合の要素を並べたものは直交行列となり、その直交行列$P$を
		$P = \left(\cfrac{\mathbold{v}_1}{\left\lvert\mathbold{v}_1 \right\rvert}\, \cfrac{\mathbold{v}_2}{\left\lvert\mathbold{v}_2 \right\rvert}\, \cfrac{\mathbold{v}_3}{\left\lvert\mathbold{v}_3 \right\rvert}\right)$とおくと、
		$P$は直交行列より、$P^{-1} = P^{\top} =\spalignmat[c]{
			{\cfrac{\mathbold{v}_1{}^{\top}}{\abs{\mathbold{v}_1}}};
			{\cfrac{\mathbold{v}_2{}^{\top}}{\abs{\mathbold{v}_2}}};
			{\cfrac{\mathbold{v}_3{}^{\top}}{\abs{\mathbold{v}_3}}}
			}$となる。よって、以下が成り立つ。
		\begin{align*}
			A  & = P\spalignmat[c]{
			{\lambda_1} {0} {0};
			{0} {\lambda_2} {0};
			{0} {0} {\lambda_3}
			}P^{-1}\\
			\Leftrightarrow
			A  & = P\spalignmat[c]{
			{\lambda_1} {0} {0};
			{0} {\lambda_2} {0};
			{0} {0} {\lambda_3}
			}P^{\top}\\
			& = \left(\cfrac{\mathbold{v}_1}{\abs{\mathbold{v}_1}}\, \cfrac{\mathbold{v}_2}{\abs{\mathbold{v}_2}}\, \cfrac{\mathbold{v}_3}{\abs{\mathbold{v}_3}}\right)
			\spalignmat[c]{
			{\lambda_1} {0} {0};
			{0} {\lambda_2} {0};
			{0} {0} {\lambda_3}
			}\spalignmat[c]{
			{\cfrac{\mathbold{v}_1{}^{\top}}{\abs{\mathbold{v}_1}}};
			{\cfrac{\mathbold{v}_2{}^{\top}}{\abs{\mathbold{v}_2}}};
			{\cfrac{\mathbold{v}_3{}^{\top}}{\abs{\mathbold{v}_3}}}
			}\\
			& = \left(
				\lambda_1\cfrac{\mathbold{v}_1}{\abs{\mathbold{v}_1}}\;
				\lambda_2\cfrac{\mathbold{v}_2}{\abs{\mathbold{v}_2}}\;
				\lambda_3\cfrac{\mathbold{v}_3}{\abs{\mathbold{v}_3}}
			\right)
			\spalignmat[c]{
			{\cfrac{\mathbold{v}_1{}^{\top}}{\abs{\mathbold{v}_1}}};
			{\cfrac{\mathbold{v}_2{}^{\top}}{\abs{\mathbold{v}_2}}};
			{\cfrac{\mathbold{v}_3{}^{\top}}{\abs{\mathbold{v}_3}}}
			}\\
			% & = \spalignmat[c]{
			% 	{0} {\lambda_2} {-2\lambda_3};
			% 	{\lambda_1} {\lambda_2} {\lambda_3};
			% 	{-\lambda_1} {\lambda_2} {\lambda_3}
			% }\spalignmat[c]{
			% 	{0} {1} {-1};
			% 	{1} {1} {1};
			% 	{-2} {1} {1}
			% }\\
			% & = \spalignmat[c]{
			% 	{\lambda_2 + 4\lambda_3} {\lambda_2 - 2\lambda_3} {\lambda_2 - 2\lambda_3};
			% 	{\lambda_2 - 2\lambda_3} {\lambda_1 + \lambda_2 + \lambda_3} {-\lambda_1 + \lambda_2 + \lambda_3};
			% 	{\lambda_2 - 2\lambda_3} {-\lambda_1 + \lambda_2 + \lambda_3} {\lambda_1 + \lambda_2 + \lambda_3};
			% }\\
			% & = \left(
			% 		0\cdot\lambda_1\cfrac{\mathbold{v}_1}{\abs{\mathbold{v}_1}} 
			% 		\, + 1\cdot\lambda_2\cfrac{\mathbold{v}_2}{\abs{\mathbold{v}_2}}
			% 		\, - 2\cdot\lambda_3\cfrac{\mathbold{v}_3}{\abs{\mathbold{v}_3}}
			% 	\;
			% 		1\cdot\lambda_1\cfrac{\mathbold{v}_1}{\abs{\mathbold{v}_1}} 
			% 		\,+ 1\cdot\lambda_2\cfrac{\mathbold{v}_2}{\abs{\mathbold{v}_2}}
			% 		\,+ 1\cdot\lambda_3\cfrac{\mathbold{v}_3}{\abs{\mathbold{v}_3}}
			% \right.\\
			% &\qquad\left.
			% 		-1\cdot\lambda_1\cfrac{\mathbold{v}_1}{\abs{\mathbold{v}_1}} 
			% 		\, + 1\cdot\lambda_2\cfrac{\mathbold{v}_2}{\abs{\mathbold{v}_2}}
			% 		\, + 1\cdot\lambda_3\cfrac{\mathbold{v}_3}{\abs{\mathbold{v}_3}}
			% \right)\\
			% & = \spalignmat[c]{
			% 	{
			% 		\cfrac{\lambda_2}{\sqrt{3}}\cfrac{\mathbold{v}_2}{\abs{\mathbold{v}_2}}
			% 		\, - \cfrac{2\lambda_3}{\sqrt{6}}\cfrac{\mathbold{v}_3}{\abs{\mathbold{v}_3}}
			% 	}
			% 	{
			% 		\cfrac{\lambda_1}{\sqrt{2}}\cfrac{\mathbold{v}_1}{\abs{\mathbold{v}_1}} 
			% 		\,+ \cfrac{\lambda_2}{\sqrt{3}}\cfrac{\mathbold{v}_2}{\abs{\mathbold{v}_2}}
			% 		\,+ \cfrac{\lambda_3}{\sqrt{6}}\cfrac{\mathbold{v}_3}{\abs{\mathbold{v}_3}}
			% 	}
			% 	{
			% 		-\cfrac{\lambda_1}{\sqrt{2}}\cfrac{\mathbold{v}_1}{\abs{\mathbold{v}_1}} 
			% 		\, + \cfrac{\lambda_2}{\sqrt{3}}\cfrac{\mathbold{v}_2}{\abs{\mathbold{v}_2}}
			% 		\, + \cfrac{\lambda_3}{\sqrt{6}}\cfrac{\mathbold{v}_3}{\abs{\mathbold{v}_3}}
			% 	}
			% }
		\end{align*}
	\item 題意より、以下が成り立つ。
		\begin{align}
			\spalignmat[c]{
				{x_n};
				{y_n};
				{z_n}
			} = A^{n} \spalignmat[c]{
				{x_0};
				{y_0};
				{z_0}
			}\label{prom1:subprom4:eq1}
		\end{align}
		但し、$A^{n}$は$A$を$A$に対して右から$n$回かけたことを意味する。また、以下も同様の意味を表す。\\
		ここで、(3)より、以下が成り立つ。
		\begin{align*}
			P^{\top}AP 
			& = \spalignmat[c]{
				{\lambda_1} {0} {0};
				{0}	{\lambda_2} {0};
				{0} {0} {\lambda_3}
			}\\
			\Longleftrightarrow 
			(P^{\top}AP)^{n} 
			& = \spalignmat[c]{
				{\lambda_1} {0} {0};
				{0}	{\lambda_2} {0};
				{0} {0} {\lambda_3}
			}^{n}\\
			\Longleftrightarrow 
			P^{\top}A^{n}P 
			& = \spalignmat[c]{
				{\lambda_1{}^{n}} {0} {0};
				{0}	{\lambda_2{}^{n}} {0};
				{0} {0} {\lambda_3{}^{n}}
			}\\
			\Longleftrightarrow 
			A^{n} 
			& = P\spalignmat[c]{
				{\lambda_1{}^{n}} {0} {0};
				{0}	{\lambda_2{}^{n}} {0};
				{0} {0} {\lambda_3{}^{n}}
			}P^{\top}
		\end{align*}
		\begin{align*}
			\therefore 
			A^{n}
			&= \spalignmat[c]{
				{
					\cfrac{\lambda_2{}^{n}}{\sqrt{3}}\cfrac{\mathbold{v}_2}{\abs{\mathbold{v}_2}}
					\, - \cfrac{2\lambda_3{}^{n}}{\sqrt{6}}\cfrac{\mathbold{v}_3}{\abs{\mathbold{v}_3}}
				}
				{
					\cfrac{\lambda_1{}^{n}}{\sqrt{2}}\cfrac{\mathbold{v}_1}{\abs{\mathbold{v}_1}} 
					\,+ \cfrac{\lambda_2{}^{n}}{\sqrt{3}}\cfrac{\mathbold{v}_2}{\abs{\mathbold{v}_2}}
					\,+ \cfrac{\lambda_3{}^{n}}{\sqrt{6}}\cfrac{\mathbold{v}_3}{\abs{\mathbold{v}_3}}
				}
				{
					-\cfrac{\lambda_1{}^{n}}{\sqrt{2}}\cfrac{\mathbold{v}_1}{\abs{\mathbold{v}_1}} 
					\, + \cfrac{\lambda_2{}^{n}}{\sqrt{3}}\cfrac{\mathbold{v}_2}{\abs{\mathbold{v}_2}}
					\, + \cfrac{\lambda_3{}^{n}}{\sqrt{6}}\cfrac{\mathbold{v}_3}{\abs{\mathbold{v}_3}}
				}
			}
		\end{align*}
		ここで
		$\mathbold{u}_{1_{n}} =  
		\cfrac{\lambda_2{}^{n}}{\sqrt{3}}\cfrac{\mathbold{v}_2}{\abs{\mathbold{v}_2}}
		\, - \cfrac{2\lambda_3{}^{n}}{\sqrt{6}}\cfrac{\mathbold{v}_3}{\abs{\mathbold{v}_3}}$,
		$\mathbold{u}_{2_{n}} = 
		\cfrac{\lambda_1{}^{n}}{\sqrt{2}}\cfrac{\mathbold{v}_1}{\abs{\mathbold{v}_1}} 
					\,+ \cfrac{\lambda_2{}^{n}}{\sqrt{3}}\cfrac{\mathbold{v}_2}{\abs{\mathbold{v}_2}}
					\,+ \cfrac{\lambda_3{}^{n}}{\sqrt{6}}\cfrac{\mathbold{v}_3}{\abs{\mathbold{v}_3}}$,
		$\mathbold{u}_{3_{n}} = 
		-\cfrac{\lambda_1{}^{n}}{\sqrt{2}}\cfrac{\mathbold{v}_1}{\abs{\mathbold{v}_1}} 
					\, + \cfrac{\lambda_2{}^{n}}{\sqrt{3}}\cfrac{\mathbold{v}_2}{\abs{\mathbold{v}_2}}
					\, + \cfrac{\lambda_3{}^{n}}{\sqrt{6}}\cfrac{\mathbold{v}_3}{\abs{\mathbold{v}_3}}$と置くと、以下のようになる。
		\begin{align}
			A^{n} = \spalignmat[c]{
				{\mathbold{u}_{1_n}} {\mathbold{u}_{2_n}} {\mathbold{u}_{3_n}}
			}
			\label{prom1:subprom4:eq2}
		\end{align}
		よって、式\eqref{prom1:subprom4:eq1}, \eqref{prom1:subprom4:eq2}より、題意は以下のようになる。
		\begin{align*}
			\spalignmat[c]{
				{x_n};
				{y_n};
				{z_n}
			}
			& = \spalignmat[c]{
				{\mathbold{u}_{1_n}} {\mathbold{u}_{2_n}} {\mathbold{u}_{3_n}}
			}
			\spalignmat[c]{
				{x_0};
				{y_0};
				{z_0}
			}\\
			&= x_0\mathbold{u}_{1_n} + y_0\mathbold{u}_{2_n} + z_0\mathbold{u}_{3_n}\\
			&= x_0\left(\cfrac{\lambda_2{}^{n}}{\sqrt{3}}\cfrac{\mathbold{v}_2}{\abs{\mathbold{v}_2}}
			\, - \cfrac{2\lambda_3{}^{n}}{\sqrt{6}}\cfrac{\mathbold{v}_3}{\abs{\mathbold{v}_3}}\right)
			\; + y_0\left(\cfrac{\lambda_1{}^{n}}{\sqrt{2}}\cfrac{\mathbold{v}_1}{\abs{\mathbold{v}_1}} 
			\,+ \cfrac{\lambda_2{}^{n}}{\sqrt{3}}\cfrac{\mathbold{v}_2}{\abs{\mathbold{v}_2}}
			\,+ \cfrac{\lambda_3{}^{n}}{\sqrt{6}}\cfrac{\mathbold{v}_3}{\abs{\mathbold{v}_3}}\right) 
			\\ & + z_0\left(-\cfrac{\lambda_1{}^{n}}{\sqrt{2}}\cfrac{\mathbold{v}_1}{\abs{\mathbold{v}_1}} 
			\, + \cfrac{\lambda_2{}^{n}}{\sqrt{3}}\cfrac{\mathbold{v}_2}{\abs{\mathbold{v}_2}}
			\, + \cfrac{\lambda_3{}^{n}}{\sqrt{6}}\cfrac{\mathbold{v}_3}{\abs{\mathbold{v}_3}}\right)\\
			&= \cfrac{\lambda_1{}^{n}}{\sqrt{2}}(y_0 - z_0)\cfrac{\mathbold{v}_1}{\abs{\mathbold{v}_1}} 
			+ \cfrac{\lambda_2{}^{n}}{\sqrt{3}}(x_0 + y_0 + z_0)\cfrac{\mathbold{v}_2}{\abs{\mathbold{v}_2}}
			+ \cfrac{\lambda_3{}^{n}}{\sqrt{6}}(-2x_0 + y_0 + z_0)\cfrac{\mathbold{v}_3}{\abs{\mathbold{v}_3}}\\
			&= \cfrac{\lambda_1{}^{n}}{\sqrt{2}}(y_0 - z_0)\spalignmat[c]{
				{0};
				{\frac{1}{\sqrt{2}}};
				{-\frac{1}{\sqrt{2}}}
			}
			+ \cfrac{\lambda_2{}^{n}}{\sqrt{3}}(x_0 + y_0 + z_0)\spalignmat[c]{
				{\frac{1}{\sqrt{3}}};
				{\frac{1}{\sqrt{3}}};
				{\frac{1}{\sqrt{3}}}
			}
			+ \cfrac{\lambda_3{}^{n}}{\sqrt{6}}(-2x_0 + y_0 + z_0)\spalignmat[c]{
				{-\frac{2}{\sqrt{6}}};
				{\frac{1}{\sqrt{6}}};
				{\frac{1}{\sqrt{6}}}
			}\\
			&= \spalignmat[c]{
				{\frac{\lambda_2{}^{n}(x_0 + y_0 + z_0)}{3} - \frac{2\lambda_3{}^{n}(-2x_0 + y_0 + z_0)}{6}};
				{\frac{\lambda_1{}^{n}(y_0 - z_0)}{2} + \frac{\lambda_2{}^{n}(x_0 + y_0 + z_0)}{3} + \frac{\lambda_3{}^{n}(-2x_0 + y_0 + z_0)}{6}};
				{-\frac{\lambda_1{}^{n}(y_0 - z_0)}{2} + \frac{\lambda_2{}^{n}(x_0 + y_0 + z_0)}{3} + \frac{\lambda_3{}^{n}(-2x_0 + y_0 + z_0)}{6}}
			}\\
			&= \frac{1}{6}\spalignmat[c]{
				{2x_0\left\{1 + 2(1 - 3\alpha)^{n}\right\} + 2y_0\left\{1 - (1 - 3\alpha)^{n}\right\} + 2z_0\left\{1 - (1 - 3\alpha)^{n}\right\}};
				{2x_0\left\{1 - (1 - 3\alpha)^{n}\right\} + y_0\left\{3(1 - \alpha)^{n} + 2 + (1 - 3\alpha)^{n}\right\} + z_0\left\{- 3(1 - \alpha)^{n} + 2 + (1 - 3\alpha)^{n}\right\}};
				{2x_0\left\{1 - (1 - 3\alpha)^{n}\right\} + y_0\left\{-3(1 - \alpha)^{n} + 2 + (1 - 3\alpha)^{n}\right\} + z_0\left\{3(1 - \alpha)^{n} + 2 + (1 - 3\alpha)^{n}\right\}}
			}
		\end{align*}
		\item (4)より、題意は以下のようになる。
			\begin{align*}
				&\lim_{n \to \infty} \spalignmat[c]{
					{x_n};
					{y_n};
					{z_n}
				}\\
				&= \lim_{n \to \infty} \frac{1}{6}\spalignmat[c]{
					{2x_0\left\{1 + 2(1 - 3\alpha)^{n}\right\} + 2y_0\left\{1 - (1 - 3\alpha)^{n}\right\} + 2z_0\left\{1 - (1 - 3\alpha)^{n}\right\}};
					{2x_0\left\{1 - (1 - 3\alpha)^{n}\right\} + y_0\left\{3(1 - \alpha)^{n} + 2 + (1 - 3\alpha)^{n}\right\} + z_0\left\{- 3(1 - \alpha)^{n} + 2 + (1 - 3\alpha)^{n}\right\}};
					{2x_0\left\{1 - (1 - 3\alpha)^{n}\right\} + y_0\left\{-3(1 - \alpha)^{n} + 2 + (1 - 3\alpha)^{n}\right\} + z_0\left\{3(1 - \alpha)^{n} + 2 + (1 - 3\alpha)^{n}\right\}}
				}
			\end{align*}
			ここで、題意より$0 < \alpha < \frac{1}{3}$より、$\frac{2}{3} < 1 - \alpha < 1$, $0 < 1 - 3\alpha < 1$である。
			よって、
			\begin{align*}
				\lim_{n \to \infty} \spalignmat[c]{
					{x_n};
					{y_n};
					{z_n}
				} &= \frac{1}{6}\spalignmat[c]{
					{2x_0 + 2y_0 + 2z_0};
					{2x_0 + 2y_0 + 2z_0};
					{2x_0 + 2y_0 + 2z_0}
				}\\
				&= \frac{1}{3}\spalignmat[c]{
					{x_0 + y_0 + z_0};
					{x_0 + y_0 + z_0};
					{x_0 + y_0 + z_0}
				}
			\end{align*}
			となる。
	\item まず$\mathbold{w}_n = (x_n\;y_n\; z_n)^{\top}, \Lambda = \spalignmat[c]{
		{\lambda_1} {0} {0};
		{0} {\lambda_2} {0};
		{0} {0} {\lambda_3}
		}$とおくと$(3)$と題意より以下が成り立つ。
		\begin{align*}
			f(x_0, y_0, z_0) &= \frac{\mathbold{w}_n{}^{\top} A \mathbold{w}_n}{\mathbold{w}_n{}^{\top}\mathbold{w}_n}\\
			&= \frac{\mathbold{w}_n{}^{\top} P \Lambda P^{-1} \mathbold{w}_n}{\mathbold{w}_n{}^{\top}\mathbold{w}_n}\\
			&= \frac{\left(P^{-1}\mathbold{w}_n\right)^{\top} \Lambda P^{-1} \mathbold{w}_n}{\mathbold{w}_n{}^{\top}\mathbold{w}_n}
		\end{align*}
		よって、ここで$\mathbold{w}_n' = P^{-1}\mathbold{w}_n = (x_n'\;y_n'\; z_n')^{\top}$とおくと以下が成り立つ。
		\begin{align*}
			f(x_0, y_0, z_0) &= \frac{\mathbold{w}_n'^{\top} \Lambda \mathbold{w}_n'}{\left(P\mathbold{w}_n'\right)^{\top}P\mathbold{w}_n'}\\
			&= \frac{\mathbold{w}_n'^{\top} \Lambda \mathbold{w}_n'}{\mathbold{w}_n'{}^{\top}\mathbold{w}_n'}\\
			&= \frac{\lambda_1 x_n'^2 + \lambda_2 y_n'^2 + \lambda_3 z_n'^2}{x_n'^2 + y_n'^2 + z_n'^2}\\
			&= \frac{x_n'^2 + y_n'^2 + z_n'^2 - \alpha(x_n'^2 + 3z_n'^2)}{x_n'^2 + y_n'^2 + z_n'^2}\\
			&= 1 - \alpha\frac{x_n'^2 + 3z_n'^2}{x_n'^2 + y_n'^2 + z_n'^2}\\
		\end{align*}
		よって、$g(x_0, y_0, z_0) = \frac{x_n'^2 + 3z_n'^2}{x_n'^2 + y_n'^2 + z_n'^2}$とおくと、$f(x_0, y_0, z_0)$は
		$g(x_0, y_0, z_0)$が最大の時、最小となり、$g(x_0, y_0, z_0)$が最小の時、最大となる。よって、$g(x_0, y_0, z_0)$の最大、最小を考える。
		\begin{align*}
			g(x_0, y_0, z_0) \leq 0
		\end{align*}
		が成り立ち、等号成立条件は$x_n'^2 = z_n'^2 = 0$つまり$x_n' = z_n' = 0$の時である。\\
		よって、この値を満たす$\mathbold{w}_0$が存在することを以下に示す。まず以下が成り立つ。
		\begin{align*}
			\mathbold{w}_{n + 1}' &= P^{-1}\mathbold{w}_{n + 1}\\
			&= P^{-1}A\mathbold{w}_n = P^{-1}P\Lambda P^{-1}\mathbold{w}_n\\
			&= \Lambda \mathbold{w}_n'\\
			\therefore \mathbold{w}_n' &= \Lambda^n \mathbold{w}_0' = \Lambda^{n} P^{-1}\mathbold{w}_0\\
			\spalignmat{
				{x_n'};
				{y_n'};
				{z_n'}
			} &= 
			\spalignmat{
				{\lambda_1^n\left(\frac{y_0}{\sqrt{2}} - \frac{z_0}{\sqrt{2}}\right)};
				{\lambda_2^n\left(\frac{x_0}{\sqrt{3}} + \frac{y_0}{\sqrt{3}} + \frac{z_0}{\sqrt{3}}\right)};
				{\lambda_3^n\left(-\frac{2x_0}{\sqrt{6}} + \frac{y_0}{\sqrt{6}} + \frac{z_0}{\sqrt{6}}\right)}
			}\\
			\therefore x_n' = 0より、\qquad y_0 &= z_0\\
			z_n' = 0より、\qquad y_0 + z_0 &= 2x_0\\
			\therefore x_0 = y_0 &= z_0 = a \qquad (a \in \mathbb{R},\; a\neq 0)
		\end{align*}
		よって、$x_0 = y_0 = z_0 = a$の時、$g(x_0, y_0, z_0)$は最小値$0$を取るので、$f(x_0, y_0, z_0)$は最大値$1 - \alpha$を取る。\\
		次に$g(x_0, y_0, z_0)$の最大値について考える。\\
		$y_n'^2 \leq 0$より以下が成り立つ。
		\begin{align*}
			g(x_0, y_0, z_0) &\leq \frac{x_n'^2 + 3z_n'^2}{x_n'^2 + z_n'^2} = 1 + \frac{2z_n'^2}{x_n'^2 + z_n'^2}\\
			&= 1 + \frac{2}{\displaystyle \frac{x_n'^2}{z_n'^2} + 1}
		\end{align*}
		よって、この不等号の等号成立条件は以下のようになる。
		\begin{align*}
			まず、y_n' = 0より、x_0 + y_0 + z_0 &= 0かつ\\
			\frac{x_n'^2}{z_n'^2} \leq 0の時、\frac{x_n'^2}{z_n'^2} &= 0の時、等号が成り立つ。\\
			よって、x_n' = 0よりy_0 =& z_0\\
			よって、b \in \mathbb{R}, b\neq 0を満たすbを用いて、
			&\begin{cases}
				x_0 = -2b\\
				y_0 = z_0 = b\\
			\end{cases}となる。
		\end{align*}
		よって、$x_0 = -2b, y_0 = z_0 = b$の時、$g(x_0, y_0, z_0)$は最大値$3$をとる。
		よって、$f(x_0, y_0, z_0)$は最小値$1 - 3\alpha$を取る。
		従って求める解答は以下のようになる。
		\begin{equation*}
			\begin{cases}
				最大値 & 1 - \alpha\\
				最小値 & 1 - 3\alpha\\
			\end{cases}
		\end{equation*}
\end{enumerate}

\newpage
\section{}%第2問
\subsection{問題文}
実数値関数$u(x, t)$が$0\, \leq \, x\, \leq\, 1, \; t\, \geq \, 0$で定義されている。
ここで$x$と$t$は互いに独立である。偏微分方程式
\begin{equation*}
    \pdiff{u}{t} = \pdiff[2]{u}{x} \qquad (*)
\end{equation*}
の解を次の条件
\begin{align*}
    \mbox{境界条件:} & \quad u(0, t) = u(1, t) = 0\\
    \mbox{初期条件:} & \quad u(x, 0) = x - x^{2}
\end{align*}
のもとで求める。但し、定数関数$u(x, t) = 0$は明らかに解であるから、それ以外の解を考える。
以下の問いに答えよ。
\begin{enumerate}[(1)]
    \item 次の式を計算せよ。ここで, $n, m$はともに正の整数とする。
        \begin{equation*}
            \dint{0}{1}{\sin (n\pi x)\; \sin (m\pi x)}
        \end{equation*}\label{subsec:prom2:subprom1}
    \item $x$のみの関数$\xi (x)$及び$t$のみの関数$\tau (t)$を用いて、$u(x, t) = \xi (x)\tau (t)$と置けるとする。
        任意の定数$C$を用いて、$\xi$および$\tau$が満たす常微分方程式をそれそれ表せ。関数$f(x)$と関数$g(t)$が任意の
        $x$と$t$について$f(x) = g(t)$を満たす場合は、$f(x)$と$g(t)$が定数関数となることを用いてもよい。\label{subsec:prom2:subprom2}
    \item 設問\eqref{subsec:prom2:subprom2}の常微分方程式を解け。次に、境界条件を満たす偏微分方程式$(*)$の解の一つが次の式で表される$u_n(x, t)$
        で与えられることを示し、$\alpha, \beta$を正の整数$n$を用いて表せ。
        \begin{equation*}
            u_n (x, t) = e^{\alpha t}\sin (\beta x)
        \end{equation*}\label{subsec:prom2:subprom3}
    \item 境界条件と初期条件を満たす偏微分方程式$(*)$の解は$u_n (x, t)$の線形結合として次の式で表される。$c_n$を求めよ。設問\eqref{subsec:prom2:subprom1}の結果を
        用いてもよい。
        \begin{equation*}
            u(x, t) = \sum\limits_{n = 1}^{\infty} c_n u_n (x, t)
        \end{equation*}\label{subsec:prom2:subprom4}
\end{enumerate}
\newpage
\subsection{解答}
\begin{enumerate}[(1)]
    \item 題意の式より以下が成り立つ。
        \begin{align*}
            \mbox{(与式)} 
            & = \frac{1}{2}\dint{0}{1}{\cos (n\pi x - m\pi x) - \cos (n\pi x + m\pi x)}\\
            & = 
            \begin{cases}
                \frac{1}{2}\dint{0}{1}{1 - \cos (2n\pi x)} & n = m\\
                \frac{1}{2}\left[\frac{\sin ((n - m)\pi x)}{(n - m)\pi}\right]_{0}^{1} - \frac{1}{2}\left[\frac{\sin ((n + m)\pi x)}{(n + m)\pi}\right]_{0}^{1} & n \neq m\\
            \end{cases}\\
            & = 
            \begin{cases}
                \frac{1}{2} - \frac{1}{2}\left[\frac{\sin (2n\pi x)}{2n\pi}\right]_{0}^{1} & n = m\\
                0 & n \neq m\\
            \end{cases}\\
            & = 
            \begin{cases}
                \frac{1}{2} & n = m\\
                0 & n \neq m\\
            \end{cases}
        \end{align*}
    \item 偏微分方程式$(*)$より$u(x, t) = \xi (x)\tau (t)$と表せるとすると以下が成り立つ。
        \begin{align*}
            (*) \Longleftrightarrow
            \pdiff{\{\xi (x)\tau (t)\}}{t} & = \pdiff[2]{\{\xi (x)\tau (t)\}}{x}\\
            \Longleftrightarrow
            \xi (x)\diff{\tau (t)}{t} & = \tau (t)\diff[2]{\xi (x)}{x}
        \end{align*}
        ここで、$\xi (x)\tau(t) \neq 0$より$\xi (x) \neq 0, \tau (t) \neq 0$よって、以下のようになる。
        \begin{align*}
            \frac{\diff[2]{\xi (x)}{x}}{\xi (x)} &= \frac{\diff{\tau (t)}{t}}{\tau (t)}\\
            \therefore &
            \begin{cases}
                \frac{\diff[2]{\xi (x)}{x}}{\xi (x)} = C\\
                \frac{\diff{\tau (t)}{t}}{\tau (t)} = C\\
            \end{cases}\\
            \Longleftrightarrow &
            \begin{cases}
                C\xi (x) = \diff[2]{\xi (x)}{x}\\
                C\tau (t) = \diff{\tau (t)}{t}\\
            \end{cases}\\
        \end{align*}
    \item $(2)$よりそれぞれ解くと以下のようになる。\\
    $\tau (t)$に関する常微分方程式を解く。$\tau (t) \neq 0$より以下が成り立つ。
        \begin{align*}
            C &= \frac{1}{\tau (t)}\diff{\tau (t)}{t}
        \end{align*}
        よって両辺$t$で積分して、
        \begin{align*}
            \dint[t]{}{}{C} &= \dint[t]{}{}{\frac{1}{\tau (t)}\diff{\tau (t)}{t}}\\
            \dint[t]{}{}{C} &= \dint[\tau]{}{}{\frac{1}{\tau}} \qquad(分かりやすくするため(t)を省略)\\
            Ct + C_3 &= \log |\tau (t)|\\
            |\tau (t)| &= e^{Ct + C_3}\\
            \tau (t) &= e^{Ct + C_3}
        \end{align*}
        よって、一般解は以下のようになる。
        \begin{equation*}
            \tau (t) = e^{Ct + C_3}
        \end{equation*}
        ここで常微分方程式から$t\to \infty$の時、$\tau (t)$は発散しないが、$C > 0$とすると、
        この一般解の式から$\tau (t)$は発散するので矛盾してしまう。よって、$C < 0$となるので、正の実数$k$を用いて、$C = -k^2$とおく。\\
    次に$\xi (x) = C_1 e^{\lambda x}$とおくと、$(2)$の常微分方程式と、$C = -k^2$より
        \begin{align*}
            -k^2 C_1 e^{\lambda x} &= \lambda^2 C_1 e^{\lambda x}\\
            -k^2 &= \lambda^2 \\
            \therefore \lambda &= \pm \imag k
        \end{align*}
    よって、斉次の微分方程式より独立な2つの解の和も解となるので、$\xi(x)$の一般解は以下のようになる。
        \begin{align*}
            \xi (x) &= C_1 e^{\imag kx} + C_2 e^{-\imag kx}\\
            \xi (x) &= (C_1 + C_2)\cos (kx) + \imag(C_1 - C_2)\sin (kx)
        \end{align*}
        よって、$C_1 + C_2 = A \in \mathbb{R}, \imag(C_1 - C_2) = B\in\mathbb{R}$とおくと以下のようになる。
        \begin{align*}
            \xi (x) &= A\cos (kx) + B\sin (kx)
        \end{align*}
        次に境界条件を満たす偏微分方程式$(*)$の解の一つに題意の式$u_n(x, t)$が存在することを示す。\\
        $(2)$より$u(x, t) = \xi (x)\tau (t) = e^{-k^2 t + C_3}(A\cos (kx) + B\sin (kx))$となる。\\
        よって、この時初期条件より
        \begin{align*}
            \tau (0) &= 1\\
            e^{C_3} &= 1\\
            \therefore C_3 &= 0\\
            \therefore \tau (t) &= e^{-k^2t}
        \end{align*}
        境界条件より、
        \begin{align*}
            &\begin{cases}
                \xi(0) = 0\\
                \xi(1) = 0\\
            \end{cases}\\
            &\begin{cases}
                A\cos (0) + B\sin (0) = 0\\
                A\cos (k) + B\sin (k) = 0\\
            \end{cases}\\
            &\begin{cases}
                A = 0\\
                B\sin (k) = 0\\
            \end{cases}\\
        \end{align*}
        よって、$\xi (x) \neq 0$より$B \neq 0$より、
        \begin{align*}
            \sin (k) &= 0\\
            \therefore k = n\pi
        \end{align*}
        \begin{equation*}
            \xi (x) = B\sin (n\pi x)
        \end{equation*}
        となる。よって、この時、$(2)$より、$u(x, t) = \xi (x)\tau (t)$は偏微分方程式$(*)$を満たすので、以下の式はこの偏微分方程式の解の一つである。
        \begin{equation*}
            u(x, t) = Be^{-n^2\pi^2t}\sin(n\pi x)
        \end{equation*}
        よって、この式の$B = 1, n\pi = \beta, -n^2\pi^2 = \alpha$とおくと、
        \begin{align*}
            e^{\alpha t}\sin(\beta x) = u_n(x, t)
        \end{align*}
        となるので、題意は示された。また、$\alpha = -n^2\pi^2, \beta = n\pi$となる。
    \item 題意のように偏微分方程式$(*)$は線形であるため、その解は$u_n(x, t)$の線形結合で表される。よって、$(3)$より以下のようになる。
    \begin{align*}
        初期条件から、u(x, 0) &= x - x^2 = \sum_{n = 1}^{\infty}c_n\sin(n\pi x)\\
        \dint{0}{1}{(x - x^2)\sin(m\pi x)} &= \dint{0}{1}{\left(\sum_{n = 1}^{\infty}c_n\sin(n\pi x)\right)\sin (m\pi x)}
    \end{align*}
    よって、それぞれ計算する。
    \begin{align*}
        m = 0の時、
        (左辺) = 0&\\
        m \neq 0の時、\qquad
        \dint{0}{1}{x\sin(m\pi x)} &= \left[x\frac{-\cos(m\pi x)}{m\pi}\right]_{0}^{1} + \frac{1}{m\pi}\dint{0}{1}{\cos(m\pi x)}\\
        &= \frac{(-1)^{m + 1}}{m\pi} + \frac{1}{m^2\pi^2}\bigl[\sin (m\pi x)\bigr]_{0}^{1} = \frac{(-1)^{m + 1}}{m\pi}\\
        \dint{0}{1}{x\cos(m\pi x)} &= \left[x\frac{\sin(m\pi x)}{m\pi}\right]_{0}^{1} - \frac{1}{m\pi}\dint{0}{1}{\sin(m\pi x)}\\
        &= -\frac{1}{m^2\pi^2}\bigl[-\cos (m\pi x)\bigr]_{0}^{1} = \frac{(-1)^{m} - 1}{m^2\pi^2}\\
        \dint{0}{1}{x^2\sin(m\pi x)} &= \left[x^2\frac{-\cos(m\pi x)}{m\pi}\right]_{0}^{1} + \frac{2}{m\pi}\dint{0}{1}{x\cos(m\pi x)}\\
        &= \frac{(-1)^{m + 1}}{m\pi} + \frac{2}{m\pi}\frac{(-1)^{m} - 1}{m^2\pi^2}\\
        (左辺) &= \dint{0}{1}{x\sin(m\pi x)} - \dint{0}{1}{x^2\sin(m\pi x)}\\
        &= \frac{(-1)^{m + 1}}{m\pi} - \left(\frac{(-1)^{m + 1}}{m\pi} + \frac{2}{m\pi}\frac{(-1)^{m} - 1}{m^2\pi^2}\right)\\
        &= \frac{2 - 2(-1)^{m}}{m^3\pi^3}
    \end{align*}
    $(1)$より右辺に関しては以下のようになる。
    \begin{align*}
        (右辺) &= \sum_{n = 1}^{\infty}c_n \dint{0}{1}{\sin(n\pi x)\sin(m\pi x)}\\
        &= \frac{1}{2}c_m
    \end{align*}
    従って、以下が成り立つ。
    \begin{align*}
        \frac{2 - 2(-1)^{m}}{m^3\pi^3} &= \frac{1}{2}c_m\\
        c_m &= \frac{4\left\{1 - (-1)^{m}\right\}}{m^3\pi^3}\\
    \end{align*}
    従って求める解答は以下のようになる。
    \begin{equation*}
        c_n = \frac{4\left\{1 - (-1)^{n}\right\}}{n^3\pi^3}
    \end{equation*}
\end{enumerate}

\newpage
\section{}%第3問
\subsection{問題文}
\begin{enumerate}[(1)]
    \item 連続確率変数$T$の確率密度関数$f(t)$が$\lambda$を正の定数として
        \begin{equation*}
            f(t) = 
            \begin{cases}
                \lambda e^{-\lambda t} & (t\geq 0)\\    
                0 & (t < 0)\\    
            \end{cases}
        \end{equation*}
        で表されるとき、$T$はパラメータ$\lambda$の指数分布に従うという。この確率変数の平均値を求めよ。
        またこの指数分布の確率分布関数$F(t) = P (T\leq t)$を求めよ。なお、$P(X)$は事象$X$が起こる確率である。\label{subsec:prom3:subprom1}
    \item 設問\eqref{subsec:prom3:subprom1}の分布が無記憶であること、すなわち任意の$s > 0, t > 0$に対して
        \begin{equation*}
            P(T > s + t | T > s) = P(T > t)
        \end{equation*}
        が成立することを示せ。なお、$P(X|Y)$は事象$Y$が起こった条件のもとで事象$X$が起こる確率である。\label{subsec:prom3:subprom2}
    \item 問題の解答を始めてから解答を終えるまでの時間を解答所要時間と呼ぶことにする。ある問題に対して
        $n$人の学生の解答所要時間が全て同じパラメータ$\lambda_0$の指数分布に従うものとする。$n$人が同時に解答を始めた時、
        最も早く解答を終える学生の解答所要時間の確率分布関数と平均値を示せ。ただし、各学生の解答所要時間はそれぞれ独立であるとする。\label{subsec:prom3:subprom3}
    \item 学生A, Bの解答所要時間がパラメータ$\lambda_A, \lambda_B$の指数分布にそれぞれ従うものとする。
        この二人が同時に解答を開始した時に、学生Aのほうが学生Bより先に解答を終える確率を求めよ。\label{subsec:prom3:subprom4}
    \item 優秀な学生である秀夫君と、他10名の学生に問題を同時に解かせる。各学生の解答所要時間は指数分布に従うものとし、また
    秀夫君以外の各学生の平均解答所要時間は、全て秀夫君の平均解答所要時間の10倍であるとする。秀夫君が1番目に解答を終える確率、
    及び4番目に解答を終える確率をそれぞれ求めよ。\label{subsec:prom3:subprom5}
\end{enumerate}
\newpage
\subsection{解答}
\begin{enumerate}[(1)]
    \item 平均値$E[T]$はこの確率変数の期待値$E[T]$を求めることと同義であるため、以下のようになる。
        \begin{align*}
            E[T] 
            &= \dint[t]{-\infty}{\infty}{tf(t)}\\
            &= \dint[t]{0}{\infty}{\lambda te^{-\lambda t}}\\
            &= \lambda\left[t\frac{e^{-\lambda t}}{-\lambda}\right]_{0}^{\infty} + \dint[t]{0}{\infty}{e^{-\lambda t}}
            \left(= \left[\frac{e^{-\lambda t}}{-\lambda}\right]_{0}^{\infty}\right)\\
            &= \frac{1}{\lambda}\\
        \end{align*}
        また確率分布関数は以下のようになる。
        \begin{align*}
            F(t) 
            &= \dint{0}{t}{f(x)}\\
            &= \lambda\left[\frac{e^{-\lambda x}}{-\lambda}\right]_{0}^{t}\\
            &= 1 - e^{-\lambda t}
        \end{align*}
    \item \eqref{subsec:prom3:subprom1}より題意の式について以下が成り立つ。
        \begin{align*}
            (左辺) 
            &= \frac{P(T > s + t \cap T > s)}{P(T > s)} \left(= \frac{P(T > s + t)}{P(T > s)}\right)\\
            &= \frac{1 - F(s + t)}{1 - F(s)}
            = \frac{e^{\displaystyle -\lambda(s + t)}}{e^{\displaystyle -\lambda s}}\\
            &= e^{-\lambda t}\\
            &= 1 - F(t) = P(T > t)\\
            &= (右辺)
        \end{align*}
        従って題意は示された。
    \item 最も早く解答を終える学生が$t$秒までに解答を終える確率は学生全員が$t$秒の時点で解答を終えてない事象の余事象が
    求める確率であるので、確率分布関数は以下のようになる。
        \begin{align*}
            P(T' \leq t) &= 1 - (P(T > t))^{n}\\
            \therefore P(T \leq t) &= 1 - e^{- \lambda n t}
        \end{align*}
        従って確率密度関数$f(t)$、平均値$E[T']$は以下のようになる。
        \begin{align*}
            f(t) &= \diff{P(T \leq t)}{t}\\
            &= \lambda n e^{- \lambda n t}\\
            \therefore E[T'] &= \dint[t]{0}{\infty}{tf(t)}\\
            &= n E[T] = \frac{n}{\lambda}\\
        \end{align*}
    \item 題意の確率$P$は学生$A$が$t$秒で解答を終えた時、学生$B$は$t$秒より後に解答を終える確率を積分したものであるため以下のようになる。
        \begin{align*}
            P &= \dint[t]{0}{\infty}{\lambda_Ae^{-\lambda_A t}e^{-\lambda_B t}}\\
            &= \lambda_A \dint[t]{0}{\infty}{e^{-(\lambda_A + \lambda_B)t}}\\
            &= \frac{\lambda_A}{\lambda_A + \lambda_B} \bigl[-e^{-(\lambda_A + \lambda_B)t}\bigr]_{0}^{\infty} 
            = \frac{\lambda_A}{\lambda_A + \lambda_B} 
        \end{align*}
    \item 秀夫君のパラメータを$\lambda_s$、他の学生のパラメータを$\lambda_1$とおくと秀夫君の平均値は
    他の学生の平均値の$\frac{1}{10}$倍より以下が成り立つ。
        \begin{align*}
            \frac{1}{\lambda_s} &= \frac{1}{10\lambda_1}\\
            \therefore \lambda_s &= 10\lambda_1
        \end{align*}
        従って、秀夫君が一番目に解答を終える確率$P_1$は以下のようになる。
        \begin{align*}
            P_1 &= \dint[t]{0}{\infty}{10\lambda_1 e^{-10\lambda_1 t}\left(e^{-\lambda_1 t}\right)^{10}}\\
            &= 10\lambda_1 \dint[t]{0}{\infty}{e^{-20\lambda_1 t}}\\
            &= \frac{1}{2}
        \end{align*}
        次に4番目に解答を終える確率$P_4$は以下のようになる。
        \begin{align*}
            P_4 &= \dint[t]{0}{\infty}{\left(1 - e^{-\lambda_1 t}\right)^{3}10\lambda_1 e^{-10\lambda_1 t}\left(e^{-\lambda_1 t}\right)^{7}}\\
            &= 10\lambda_1 \dint[t]{0}{\infty}{\left(1 - e^{-3\lambda_1 t} + 3e^{-2\lambda_1 t} - 3e^{-\lambda_1 t}\right)e^{-17\lambda_1 t}}\\
            &= 10\lambda_1 \dint[t]{0}{\infty}{e^{-17\lambda_1 t} - e^{-20\lambda_1 t} + 3e^{-19\lambda_1 t} - 3e^{-18\lambda_1 t}}\\
            &= 10\left(\frac{1}{17} - \frac{1}{20} + \frac{3}{19} - \frac{3}{18}\right)\\
            &= 10\left(\frac{60 - 19}{20\cdot 19} + \frac{18 - 51}{18\cdot 17}\right)\\
            &= \frac{41}{38} - \frac{55}{51}\\
            &= \frac{41\cdot 51 - 55\cdot 38}{38 \cdot 51}\\
            &= \frac{2091 - 2090}{1938}\\
            &= \frac{1}{1938}\\
        \end{align*}
\end{enumerate}
% \newpage
\index{ティック@tikz}
% \printindex
\end{document}