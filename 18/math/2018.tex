\documentclass[dvipdfmx,titlepage, 11pt, a4paper]{jsarticle}%A4用紙縦,明朝(デフォルト)11ポイント
\usepackage[top=18truemm,bottom=18truemm,left=18truemm,right=18truemm]{geometry}%余白調整
\usepackage{template}
%%%====================================================================================================================================
\renewcommand{\thesection}{第\arabic{section}問}
\renewcommand{\thesubsection}{\thesection}
\titleformat*{\section}{\LARGE\mcfamily}%章のタイトルの文字の大きさを通常サイズに設定(明朝体で)
\titlespacing*{\section}{0pt}{*0}{0pt}%章番号の後の空白行の削除
\titleformat*{\subsection}{\Large\mcfamily}%節のタイトルの文字の大きさを通常サイズに設定(明朝体で)
\titlespacing*{\subsection}{0pt}{*0}{0pt}%節番号の後の空白行の削除
\titleformat*{\subsubsection}{\large\mcfamily}%小節のタイトルの文字の大きさを通常サイズに設定(明朝体で)
\titlespacing*{\subsubsection}{0pt}{*0}{0pt}%小節番号の後の空白行の削除

\makeatletter
%章番号付きコード番号
\AtBeginDocument{
  \renewcommand*{\thelstlisting}{\arabic{section}.\arabic{lstlisting}}%
  \@addtoreset{lstlisting}{section}
}

%章番号付き表番号
\renewcommand{\thetable}{%
    \arabic{section}.\arabic{table}%
}
\@addtoreset{table}{section}%

%章番号付き図番号
\renewcommand{\thefigure}{%
	\arabic{section}.\arabic{figure}%
}%
\@addtoreset{figure}{section}%

%章番号付き式番号
\renewcommand{\theequation}{%
	\arabic{section}.\arabic{equation}%
}%
\@addtoreset{equation}{section}%

\renewcommand{\p@enumiii}{}% 箇条書きの参照時の番号の変更(2(親の番号)a(子の番号)→aだけに)
\renewcommand{\p@enumii}{}% 箇条書きの参照時の番号の変更(2a→aだけに)
\makeatother
%%%====================================================================================================================================

%%%====================================================================================================================================
%%ページのレイアウト設定
\pagestyle{fancy}
\renewcommand{\sectionmark}[1]{\markboth{}{\thesection\ #1}}%                   %\rightmarkにセクション名を格納
\renewcommand{\subsectionmark}[1]{\markboth{#1}{\rightmark}}%   %\leftmarkにサブセクション名を格納
%[]は省略可で省略すると{}で指定された内容が偶数ページ奇数のどちらにも適用される.
% \renewcommand{\headrulewidth}{0pt} %ヘッダの罫線を消す 
\fancyfoot{}%                        %clear all footer fields
\lhead{\leftmark}%                   %左側ヘッダの定義[<偶数ページ>]{<奇数ページ>}
\chead{}%                            %中央ヘッダの定義[<偶数ページ>]{<奇数ページ>}
\rhead{\rightmark}%                  %右側ヘッダの定義[<偶数ページ>]{<奇数ページ>}
\lfoot{東大2018年度数学解答例}%       %左側フッターの定義[<偶数ページ>]{<奇数ページ>}
\cfoot{\thepage}%                    %中央フッターの定義[<偶数ページ>]{<奇数ページ>}
\rfoot{文殊の知恵}%                  %右側フッターの定義[<偶数ページ>]{<奇数ページ>}
\renewcommand{\headrulewidth}{0.1pt}%%ヘッダの線の太さ 
\renewcommand{\footrulewidth}{0.1pt}%%フッターの線の太さ
%%%====================================================================================================================================

\makeindex%索引用

\renewcommand{\refname}{}%参考文献の文字を非表示にする
\title{\Huge 東大2018年度数学解答例\\[10mm]}
\author{{\LARGE 文殊の知恵}\\[1mm]\LARGE 中田昌輝\\(加筆:高橋那弥)}
\date{}

\begin{document}
\newcommand{\dix}{\,{\rm d}}
\maketitle
\tableofcontents % 目次
\pagenumbering{roman}%目次のページ番号のスタイルをローマ数字にする
\newpage
\setcounter{tocdepth}{3}%章節の深さを3にするsubsubsectionまで
\pagenumbering{arabic}%他のページ番号は通常の数字にする.
\section{}%第1問
\subsection{問題文}
次の連立一次方程式を解く問題を考える.
\begin{eqnarray*}
	\bm{A}\bm{x} = \bm{b}
\end{eqnarray*}
ここで, $\bm{A}\in \mathcal{R}^{m\times n},\,b\in \mathcal{R}^{m}$は与えられた定数の行列とベクトルであり, $\bm{x}\in \mathcal{R}^{n}$は未知ベクトルである. 以下の問いに答えよ.
\begin{enumerate}[(1)]
	\setlength{\itemsep}{10pt}
	\item $\overline{\bm{A}}=(\bm{A}\,|\,\bm{b})$のように, 行列$\bm{A}$の最後の列の後ろに1列追加した$m\times (n+1)$行列を作る. 例えば, $\bm{A}=\begin{pmatrix}1&0&-1\\1&1&0\\0&1&1\end{pmatrix},\,\bm{b}=\begin{pmatrix}2\\4\\2\end{pmatrix}$の場合には, $\overline{\bm{A}}=\begin{pmatrix}1&0&-1&2\\1&1&0&4\\0&1&1&2\end{pmatrix}$となる. この例の$\overline{\bm{A}}$の第$i$列ベクトルを$\bm{a}_{i}\ (i=1,\,2,\,3,\,4)$とする.
	      \begin{enumerate}[(i)]
		      \item $\bm{a}_{1},\,\bm{a}_{2},\,\bm{a}_{3}$のうち線形独立なベクトルの最大個数を求めよ.
		      \item $\bm{a}_{4}$が$\bm{a}_{1},\,\bm{a}_{2},\,\bm{a}_{3}$の線形和で表されることを, $\bm{a}_{4}=x_{1}\bm{a}_{1}+x_{2}\bm{a}_{2}+\bm{a_{3}}$となるスカラー$x_{1},\,x_{2}$を求めることで示せ.
		      \item $\bm{a}_{1},\,\bm{a}_{2},\,\bm{a}_{3},\,\bm{a}_{4}$のうち線形独立なベクトルの最大個数を求めよ.
	      \end{enumerate}
	\item 任意の$m,\,n,\,\bm{A},\,\bm{b}$に対して, ${\rm rank}\overline{\bm{A}}={\rm rank}\bm{A}$のとき連立一次方程式の解が存在することを示せ.
	\item ${\rm rank}\overline{\bm{A}}>{\rm rank}\bm{A}$ならば解は存在しない. $m>n,\,{\rm rank}\bm{A}=n$で, ${\rm rank}\overline{\bm{A}}>{\rm rank}\bm{A}$のとき, 連立一次方程式の右辺と左辺の差のノルムの2乗$\|\bm{b}-\bm{Ax}\|^{2}$を最小にする$\bm{x}$を求めよ.
	\item $m<n,\,{\rm rank}\bm{A}=m$のとき, どのような$\bm{b}$に対しても連立一次方程式を満たす解が複数存在する. 解のうちで$\|\bm{x}\|^{2}$を最小にする$\bm{x}$を, 連立一次方程式の制約条件として, ラグランジュ乗数法を用いて求めよ.
	\item 任意の$m,\,n,\,\bm{A}$に対して, 以下の4つの式を満たす$\bm{P}\in\mathcal{R}^{n\times m}$が唯一に決まることを示せ.
	      \begin{eqnarray*}
		      \bm{APA}=\bm{A}\\
		      \bm{PAP}=\bm{P}\\
		      (\bm{AP})^{T}=\bm{AP}\\
		      (\bm{PA})^{T}=\bm{PA}
	      \end{eqnarray*}
	\item (3)で求めた$\bm{x}$と(4)で求めた$\bm{x}$が, いずれも$\bm{x}=\bm{Pb}$の形で表せることを示せ.
\end{enumerate}

\newpage

\subsection{解答例}
\begin{enumerate}[(1)]
	\setlength{\itemsep}{10pt}
	\item
	      \begin{enumerate}[(i)]
		      \setlength{\itemsep}{10pt}
		      \item $\bm{a}_{1},\,\bm{a}_{2}$が線形従属の関係にある場合, 実数$c\ (\neq 0,\,\in \mathbb{R})$を用いて
		            \begin{eqnarray*}
			            \bm{a}_{1} = c\,\bm{a}_{2}
		            \end{eqnarray*}
		            と表すことができる. しかし, $\bm{a}_{1}=\begin{pmatrix}1\\1\\0\end{pmatrix},\ \bm{a}_{2}=\begin{pmatrix}0\\1\\1\end{pmatrix}$であることから, これを満たす実数$c$は存在しないので, $\bm{a}_{1},\,\bm{a}_{2}$は線形独立なベクトルである. 次に, $\bm{a}_{3}$が$\bm{a}_{1},\,\bm{a}_{2}$の線形従属の関係にある場合, 実数$c_{1},\,c_{2}\ (\neq 0,\,\in \mathbb{R})$を用いて
		            \begin{eqnarray*}
			            \bm{a}_{3}=c_{1}\bm{a}_{1}+c_{2}\bm{a}_{2}
		            \end{eqnarray*}
		            と表すことが出来る. $\bm{a}_{3}=\begin{pmatrix}-1\\0\\1\end{pmatrix}$であるので, $c_{1}=-1,\,c_{2}=1$とすればこの式は満たされる. ゆえに$\bm{a}_{3}$は線形従属の関係にあるので, 線形独立なベクトルの最大個数は2個である.
		      \item $\bm{a}_{4}$について$\bm{a}_{1},\,\bm{a}_{2},\,\bm{a}_{3}$の線形和で表されることを$x_{1},\,x_{2}$を求めることで示す.
		            \begin{eqnarray*}
			            && \bm{a}_{4}=x_{1}\bm{a}_{1}+x_{2}\bm{a}_{2}+\bm{a}_{3}\\
			            \Longleftrightarrow\ && \begin{pmatrix}2\\4\\2\end{pmatrix}=x_{1}\begin{pmatrix}1\\1\\0\end{pmatrix}+x_{2}\begin{pmatrix}0\\1\\1\end{pmatrix}+\begin{pmatrix}-1\\0\\1\end{pmatrix}\\
			            \Longleftrightarrow\ &&\left\{\begin{array}{l}
				            2=x_{1}-1     \\
				            4=x_{1}+x_{2} \\
				            2=x_{2}+1
			            \end{array}
			            \right.\\
			            \Longleftrightarrow\ && \left\{\begin{array}{l}
				            x_{1}=3 \\
				            x_{2}=1
			            \end{array}
			            \right.
		            \end{eqnarray*}
		            したがって, $x_{1}=3,\,x_{2}=1$というスカラーの組が求まったので, $\bm{a}_{4}$は$\bm{a}_{1},\,\bm{a}_{2},\,\bm{a}_{3}$の線形和で表される.
		      \item $\bm{a}_{4}$は$\bm{a}_{1},\,\bm{a}_{2},\,\bm{a}_{3}$の線形和で表されることから$\bm{a}_{4}$は$\bm{a}_{1},\,\bm{a}_{2},\,\bm{a}_{3}$と線形従属な関係であるので, 線形独立なベクトルの最大個数は2個である.
	      \end{enumerate}
	\item 行列$\bm{A}$の第$i$列ベクトルを$\bm{a}_{i}\ (i=1,2,...,n)$とする. ${\rm rank}\overline{\bm{A}}={\rm rank}\bm{A}$のとき, 実数$c_{i}\ (\neq 0,\,\in \mathbb{R})$を用いて
	      \begin{eqnarray*}
		      \bm{b}=\sum_{i=1}^{n}c_{i}\bm{a}_{i}
	      \end{eqnarray*}
	      と表すことができる. この$c_{i}$を1から順に$n$まで縦に並べた$\bm{c}=\begin{pmatrix}c_{1}\\c_{2}\\\vdots\\c_{n}\end{pmatrix}$をつくると
	      \begin{eqnarray*}
		      \bm{Ac}=\bm{b}
	      \end{eqnarray*}
		  となるので, $\bm{c}$は連立方程式の解$\bm{x}$となる. ゆえに題意は示された.\\[1cm]
		  (高橋解)\\
		  ${\rm rank}\overline{\bm{A}}={\rm rank}\bm{A}$のとき,行列$\bm{A}$の階段行列を$\bm{A'}$,行列$\overline{\bm{A}}$の階段行列を$\overline{\bm{A'}}$と置き,
		  ${\rm rank}\overline{\bm{A}}={\rm rank}\bm{A} = r(\leq n)$と置く.
		  この時,行列$\bm{A'}$の第$(i, j)$成分を$a_{ij}{}^{'}$,行列$\overline{\bm{A'}}$の第$(n + 1, j)$成分を$b_{j}{}^{'}$, この成分を縦に並べたベクトルを$\bm{b'}$とおく.
		  また,$j$行において$\alpha_j = \underset{i\, \in\, \{i |j\, \leq\, i\, \leq\, n\, \wedge\, a_{ij}{}^{'} \neq 0\}}{\min }i$とおく.
		  この時,以下が成り立つ.
		  \begin{eqnarray*}
			a_{ij}{}^{'} = b_{j}{}^{'} = 0 \; (r\, <\, j\, \leq\, m)\\
			1\,\leq\,\alpha_{1}\,<\,\cdots\,<\,\alpha_{j - 1}\, <\, \alpha_{j}\,<\,\cdots\,<\,\alpha_{r}\,\leq\,n\\
			a_{ij}{}^{'} = 0 \; (1\, \leq\, i\, <\, \alpha_j\, \wedge\, 1\, \leq\, j\, \leq\, r)\\
			\therefore \sum_{i = \alpha_j}^{n}c_{i}a_{ij}{}^{'} = b_{j}{}^{'} \; (1\, \leq\, j\, \leq\, r)となるc_{i}が存在する.
		  \end{eqnarray*}
		  従って,$c_{i}$をを1から順に$n$まで縦に並べた$\bm{c}=\begin{pmatrix}c_{1}\\c_{2}\\\vdots\\c_{n}\end{pmatrix}$をつくると
			\begin{eqnarray}
				\bm{A'c} = \bm{b'}
			\end{eqnarray}
			であり,
		  \begin{eqnarray*}
			  \bm{Ax} = \bm{b}\\
			  \Longleftrightarrow \bm{A'x} = \bm{b'}
		  \end{eqnarray*}
		  であるので,$\bm{c}$は元の連立方程式の解$\bm{x}$である.よって題意は示された.
	\item 以下のように変形することができる.
	      \begin{eqnarray*}
		      \|\bm{b}-\bm{Ax}\|^{2}&=&(\bm{b}-\bm{Ax})^{T}(\bm{b}-\bm{Ax})\\
		      &=&(\bm{b}^{T}-\bm{x}^{T}\bm{A}^{T})(\bm{b}-\bm{Ax})\\
		      &=&\bm{x}^{T}\bm{A}^{T}\bm{A}\bm{x}-2\bm{x}^{T}\bm{A}^{T}\bm{b}+\bm{b}^{T}\bm{b}
	      \end{eqnarray*}
	      これを$\bm{x}$で微分すると
	      \begin{eqnarray*}
		      \frac{\partial \|\bm{b-Ax}\|^{2}}{\partial \bm{x}} = 2\bm{A}^{T}\bm{A}\bm{x}-2\bm{A}^{T}\bm{b}
	      \end{eqnarray*}
	      となる. ゆえに最小となる必要条件は
	      \begin{eqnarray*}
		      \bm{A}^{T}\bm{A}\bm{x}=\bm{A}^{T}\bm{b}
	      \end{eqnarray*}
	      である. ${\rm rank}\bm{A}=n$かつ${\rm rank}\overline{\bm{A}}>{\rm rank}\bm{A}$のとき, ${\rm rank}(\bm{A}^{T}\bm{A})=n$となる. ゆえに$\bm{A}$は正則であるので, 逆行列が存在して
	      \begin{eqnarray*}
		      &&(\bm{A}^{T}\bm{A})^{-1}\bm{A}^{T}\bm{A}\bm{x}=(\bm{A}^{T}\bm{A})^{-1}\bm{A}^{T}\bm{b}\\
		      \Longleftrightarrow\ && \bm{x}=(\bm{A}^{T}\bm{A})^{-1}\bm{A}^{T}\bm{b}
	      \end{eqnarray*}
		  となる.
	\item (高橋解)
		連立一次方程式を制約式とするラグランジュの未定乗数法よりラグランジュ関数を$L(\bm{x}, \bm{\lambda})$,ラグランジュ乗数$\bm{\lambda}\in\mathbb{R}^{m\times 1}$とおく
		\begin{eqnarray*}
			L(\bm{x}, \bm{\lambda}) = \bm{x}^{\top}\bm{x} - \bm{\lambda}^{\top}(\bm{Ax} - \bm{b})
		\end{eqnarray*}
		ここで,$\|\bm{x}\|^{2} = \bm{x}^{\top}\bm{x}\, \geq\, 0$より,上記の式が極値を取る時,$\|\bm{x}\|^{2}$は最小となる.
		よって,偏微分すると以下のようになる.
		\begin{eqnarray*}
			\pdiff{L(\bm{x}, \bm{\lambda})}{\bm{x}} &=& 2\bm{x} - \bm{A}^{\top}\bm{\lambda} = \bm{0}\\
			\pdiff{L(\bm{x}, \bm{\lambda})}{\bm{\lambda}} &=& -\bm{Ax} + \bm{b} = \bm{0}\\
			\bm{x} = \frac{1}{2}\bm{A}^{\top}\bm{\lambda}を代入して\\
			\frac{1}{2}\bm{A}\bm{A}^{\top}\bm{\lambda} = \bm{b}\;
			&\therefore&
			\bm{\lambda} = 2(\bm{AA^{\top}})^{-1}\bm{b}\\
			&\therefore&
			\bm{x} = \bm{A}^{\top}(\bm{AA^{\top}})^{-1}\bm{b}
		\end{eqnarray*}
	\item 
	\item (3), (4)の$\bm{x}$が$\bm{x} = \bm{Pb}$と表せるとすると(5)の条件を満たす必要がある.また,(5)より
	(5)の条件を満たすものは一つだけ存在するため,条件を満たすならば(3), (4)は同じ行列で表せる.
	(3)より,$\bm{P_3} = (\bm{A}^{\top}\bm{A})^{-1}\bm{A}^{\top}$と置くと,(5)の条件より,
	\begin{eqnarray*}
		\bm{AP_{3}A} = \bm{A}(\bm{A}^{\top}\bm{A})^{-1}\bm{A}^{\top}\bm{A} = \bm{A}\\
		\bm{P_{3}AP_{3}} = (\bm{A}^{\top}\bm{A})^{-1}\bm{A}^{\top}\bm{A}(\bm{A}^{\top}\bm{A})^{-1}\bm{A}^{\top} = (\bm{A}^{\top}\bm{A})^{-1}\bm{A}^{\top} = \bm{P_3}\\
		(\bm{AP_{3}})^{\top} = (\bm{A}(\bm{A}^{\top}\bm{A})^{-1}\bm{A}^{\top})^{\top} = \bm{A}(\bm{A}^{\top}\bm{A})^{-1}\bm{A}^{\top} = \bm{A}\bm{P_3}
		\quad \because (\bm{A}^{-1})^{\top} = (\bm{A}^{\top})^{-1}\\
		(\bm{P_{3}A})^{\top} = ((\bm{A}^{\top}\bm{A})^{-1}\bm{A}^{\top}\bm{A})^{\top} = (\bm{A}^{\top}\bm{A})^{-1}\bm{A}^{\top}\bm{A} = \bm{P_3}\bm{A}
	\end{eqnarray*}
\end{enumerate}
\newpage
\section{}%第2問
\subsection{問題文}
関数$f_{1}$を$[0,1]$上で定義される正値の定数関数とし, $f_{1}(x)=c$とおく. また, 正の実数$p,q$を$1/p+1/q=1$を満たすものとする. これらに対し, $[0,1]$上で定義される関数の列$\{f_{n}\}$を
\begin{eqnarray*}
	f_{n+1}(x)=p\int_{0}^{x}(f_{n}(t))^{1/q}\dix t
\end{eqnarray*}
で定める. 以下の問いに答えよ.
\begin{enumerate}[(1)]
	\setlength{\itemsep}{10pt}
	\item $a_{1}=0,\,c_{1}=c$かつ
	      \begin{eqnarray*}
		      && a_{n+1}=q^{-1}a_{n}+1\hspace{15pt} (n=1,2,...),\\
		      && c_{n+1}=\frac{p\,(c_{n})^{1/q}}{a_{n+1}}\hspace{15pt} (n=1,2,...)
	      \end{eqnarray*}
	      で定まる実数列$\{a_{n}\}$と$\{c_{n}\}$を用いて$f_{n}(x)=c_{n}x^{a_{n}}$と表されることを示せ.
	\item $n\geq 2$に対し$[0,1]$上で定義される関数$g_{n}$を$g_{n}(x)=x^{a_{n}}-x^{p}$とおく. $n\geq 2$に対し$a_{n}\geq 1$となることに注意して, $g_{n}$がある点$x=x_{n}$で最大値をとることを示し, この$x_{n}$を求めよ.
	\item 任意の$x\in [0,1]$に対して$\displaystyle \lim_{n\to \infty}g_{n}(x)=0$となることを示せ.
	\item $d_{n}=(c_{n})^{q^{n}}$とおく. $d_{n+1}/d_{n}$が$n\to \infty$のとき有限な正の値に収束することを示せ.\\
	      なお, $\displaystyle \lim_{t\to \infty}(1-1/t)^{t}=1/{\rm e}$となることは用いて良い.
	\item $\displaystyle \lim_{n\to \infty}c_{n}$の値を求めよ.
	\item 任意の$x\in[0,1]$に対して$\displaystyle \lim_{n\to \infty}f_{n}(x)=x^{p}$となることを示せ.
\end{enumerate}
\newpage
\subsection{解答例}
\begin{enumerate}[(1)]
	\setlength{\itemsep}{10pt}
	\item 数学的帰納法によって示す.\\
	      \begin{enumerate}[(i)]
		      \item $n=1$のとき,
		            \begin{eqnarray*}
			            f_{1}(x)=c=c\cdot x^{0}=c_{1}x^{a_{0}}
		            \end{eqnarray*}
		            となるので, 成立する.
		      \item $n=k$のとき$f_{n}(x)=c_{n}x^{a_{n}}$であると仮定すると, $n=k+1$のときは
		            \begin{eqnarray*}
			            f_{k+1}(x)&=&p\int_{0}^{x}((f_{k}(t))^{1/q}\dix t\\
			            &=&p\int_{0}^{x}(c_{k}x^{a_{k}})^{1/q}\dix t\\
			            &=&p\left[c_{k}^{1/q}\cdot \frac{q}{a_{k}+q}x^{\frac{a_{k}+q}{q}}\right]_{0}^{x}\\
			            &=&p\,c_{k}^{1/q}\cdot \frac{1}{q^{-1}a_{k}+1}x^{q^{-1}a_{k}+1}\\
			            &=&\frac{p\,c_{k}^{1/q}}{a_{k+1}}x^{a_{k+1}}\\
			            &=&c_{k+1}x^{a_{k+1}}
		            \end{eqnarray*}
		            となり, 成立する.
	      \end{enumerate}
	      (i),(ii)より, 数学的帰納法より, 題意は示された.
	\item $n\geq 2$のとき,
	      \begin{eqnarray*}
		      g_{n}'(x)=a_{n}x^{a_{n}-1}-px^{p-1}
	      \end{eqnarray*}
	      ここで, $a_{n}$の一般項を求める. 特性方程式$\alpha = q^{-1}\alpha+1$を解くことによって, $\alpha = p$となる. ゆえに
	      \begin{eqnarray*}
		      &&a_{n}-p=q^{-1}\left(a_{n}-p\right)\\
		      &&a_{n}=p\left(1-q^{1-n}\right) < p
	      \end{eqnarray*}
	      ゆえに, $g'_{n}(x)=0$となるときは
	      \begin{eqnarray*}
		      &&g_{n}'(x)=0\\
		      \Longleftrightarrow\ && a_{n}x^{a_{n}-1}-p\,x^{p-1}=0\\
		      \Longleftrightarrow\ && x^{a_{n}-1}\Bigl(a_{n}-p\,x^{p-a_{n}}\Bigr)=0\\
		      \Longleftrightarrow\ && x=0,\left(\frac{a_{n}}{p}\right)^{\frac{1}{p-a_{n}}}
	      \end{eqnarray*}
	      $g_{n}'(1)=a_{n}-p<0$より増減表は以下のように書くことができる.
	      \begin{center}
		      \begin{tabular}{|c|c||c|c|c|c|}\hline
			      $x$         & 0 & $\cdots$   & $\displaystyle \left(\frac{a_{n}}{p}\right)^{\frac{1}{p-a_{n}}}$ & $\cdots$   & 1         \\ \hline
			      $g_{n}'(x)$ & 0 & $+$        & 0                                                                & $-$        & $a_{n}-p$ \\ \hline
			      $g_{n}(x)$  & 0 & $\nearrow$ &                                                                  & $\searrow$ & 0         \\ \hline
		      \end{tabular}
	      \end{center}
	      よって, $x_{n}=\displaystyle \left(\frac{a_{n}}{p}\right)^{\frac{1}{p-a_{n}}}$のとき最大値をとることが示された.
	\item まず,
	      \begin{eqnarray*}
		      \lim_{n\to \infty}a_{n}=\lim_{n\to \infty}\left\{p(1-q^{1-n})\right\}=p
	      \end{eqnarray*}
	      より,
	      \begin{eqnarray*}
		      \lim_{n\to \infty}g_{n}'(x) &=& \lim_{n\to \infty}a_{n}x^{a_{n}-1}-px^{p-1}\\
		      &=& px^{p-1}-px^{p-1}\\
		      &=&0
	      \end{eqnarray*}
		  これより関数$g_{n}'(x)$は区間$[0,1]$において増減がなく, $n\to \infty$において$g_{n}(0)=g_{n}(1)=0$であるから, 任意の$x\in[0,1]$に対しても$n\to \infty$において$g_{n}(x)=0$となる. よって, 題意は示された.\\[1cm]
		  (高橋解1)
		  \begin{eqnarray*}
			q^{-1} < 1より\\
			\lim_{n\to \infty}a_{n} = \lim_{n\to \infty}p(1 - q^{1 - n}) = p\\
			\therefore \lim_{n\to \infty}g_{n}(x) = \lim_{n\to \infty}x^{a_n} - x^{p} = 0
		  \end{eqnarray*}
		  (高橋解2)
		  \begin{eqnarray*}
			0\, \leq\, x\, \leq\, 1において(2)より\\
			0\, \leq\, g_n(x)\, \leq\, g_n(x_n) = \left(\frac{a_n}{p}\right)^{\frac{a_n}{p - a_n}} - \left(\frac{a_n}{p}\right)^{\frac{p}{p - a_n}}\\
			\lim_{n\to \infty}g_n(x_n) = \lim_{n\to \infty}\left(\frac{a_n}{p}\right)^{\frac{a_n}{p - a_n}} - \left(\frac{a_n}{p}\right)^{\frac{p}{p - a_n}} = 1 - 1 = 0\\
			よって、挟み撃ちの定理より
			\lim_{n\to \infty}g_{n}(x) = \lim_{n\to \infty}x^{a_n} - x^{p} = 0
		  \end{eqnarray*}
	\item $q$の大きさについて
	      \begin{eqnarray*}
		      \frac{1}{p}+\frac{1}{q}=1&\Longrightarrow& \frac{1}{q}<1\\
		      &\Longleftrightarrow& q>1
	      \end{eqnarray*}
	      となる. $d_{n+1}/d_{n}$について整理すると
	      \begin{eqnarray*}
		      \frac{d_{n+1}}{d_{n}}&=&\frac{\displaystyle \left(\frac{p(c_{n})^{1/q}}{a_{n+1}}\right)^{q^{n+1}}}{(c_{n})^{q^{n}}}\\
		      &=&\frac{\displaystyle \frac{p^{q^{n+1}}(c_{n})^{q^{n}}}{(a_{n+1})^{q^{n+1}}}}{(c_{n})^{q^{n}}}\\
		      &=&\frac{p^{q^{n+1}}}{(a_{n+1})^{q^{n+1}}}\\
		      &=&\left(\frac{p}{p(1-q^{-n})}\right)^{q_{n+1}}\\
		      &=&\left(\frac{1}{1-q^{-n}}\right)^{q^{n+1}}\\
		      &=&\left(\frac{1}{1-\frac{1}{q^{n}}}\right)^{q^{n}}\cdot \left(\frac{1}{1-\frac{1}{q^{n}}}\right)
	      \end{eqnarray*}
	      $q>1$より, $n\to \infty$のとき$q^{n}\to \infty$であることを用いて
	      \begin{eqnarray*}
		      \lim_{n\to \infty}\frac{d_{n+1}}{d_{n}}&=&\lim_{n\to \infty}\left(\frac{1}{1-\frac{1}{q^{n}}}\right)^{q^{n}}\cdot \left(\frac{1}{1-\frac{1}{q^{n}}}\right)\\
		      &=& \frac{1}{\rm e}
		  \end{eqnarray*}
		  (高橋解)
		  \begin{eqnarray*}
			  \frac{d_{n + 1}}{d_{n}}&=&\frac{\displaystyle \left(\frac{p(c_{n})^{1/q}}{a_{n+1}}\right)^{q^{n+1}}}{(c_{n})^{q^{n}}}\\
			  &=&\frac{\displaystyle \left(\frac{p(c_{n})^{1/q}}{p(1 - q^{-n})}\right)^{q^{n+1}}}{(c_{n})^{q^{n}}}\\
			  &=&\frac{\displaystyle \left(\frac{(c_{n})^{1/q}}{(1 - q^{-n})}\right)^{q^{n+1}}}{(c_{n})^{q^{n}}}\\
			  &=&\frac{\left((c_{n})^{1/q}\right)^{q^{n+1}}}{(c_{n})^{q^{n}}(1 - q^{-n})^{q^{n + 1}}}\\
			  &=&\frac{(c_{n})^{q^{n}}}{(c_{n})^{q^{n}}(1 - q^{-n})^{q^{n + 1}}}\\
			  &=&\frac{1}{(1 - q^{-n})^{q^{n + 1}}}\\
			  &=&\left(1 - \frac{1}{q^{n}}\right)^{-q^{n + 1}}
		  \end{eqnarray*}
		  従って,
		  \begin{eqnarray*}
			  \lim_{n\to \infty} \frac{d_{n + 1}}{d_{n}} &=& \lim_{n\to \infty}\left(1 - \frac{1}{q^{n}}\right)^{-q^{n + 1}}\\
			  &=& \lim_{n\to \infty}\left\{\left(1 - \frac{1}{q^{n}}\right)^{q^{n}}\right\}^{-q}\\
			  &=& e^{q}
			\end{eqnarray*}
	\item $(4)$より,$\,b_n = {\displaystyle  \frac{d_n + 1}{d_n}} = {\displaystyle \left(1 - \frac{1}{q^{n}}\right)^{-q^{n + 1}}}$と置くと,以下が成り立つ.
	\begin{eqnarray*}
		\frac{b_{n + 1}}{b_n} &=& \frac{{\displaystyle \left(1 - \frac{1}{q^{n + 1}}\right)^{-q^{n + 2}}}}{{\displaystyle \left(1 - \frac{1}{q^{n}}\right)^{-q^{n + 1}}}}\\
		&=& \frac{{\displaystyle \left(1 - \frac{1}{q^{n}}\right)^{q^{n + 1}}}}{{\displaystyle \left(1 - \frac{1}{q^{n + 1}}\right)^{q^{n + 2}}}}\\
	\end{eqnarray*}
	ここで,
	\begin{eqnarray*}
		\frac{1}{q} < 1より,\left(\frac{1}{q}\right)^{n} < 1, \;\left(\frac{1}{q}\right)^{n + 1} < \left(\frac{1}{q}\right)^{n}\\
		\therefore 1 - \left(\frac{1}{q}\right)^{n} < 1, \; 1 - \left(\frac{1}{q}\right)^{n} < 1 - \left(\frac{1}{q}\right)^{n + 1}\\
		\lim_{n \to \infty}(1 - 1/n)^{n} = \lim_{n \to -\infty}(1 + 1/n)^{-n} = \frac{1}{e}\\
		f(x) = (1 - 1/x)^x\\
		\log f(x) = x\log(1 - 1/x) \\
		\frac{f'(x)}{f(x)} = \log(1 - 1/x) + x\frac{1}{x^2}\frac{x}{x - 1} = \log(1 - 1/x) + \frac{1}{x - 1}\\
		g(x) =  \log(1 - 1/x) + \frac{1}{x - 1}\\
		g'(x) = \frac{1}{x(x - 1)} - (x - 1)^{-2} = \frac{x - 1 - x}{x(x - 1)^2} = \frac{-1}{x(x - 1)^2}\\
		g(x) > g(2) = -\log 2 + 1 > 0
	\end{eqnarray*}
	\begin{eqnarray*}
		f(x) = (1 - 1/x)^{-qx}\\
		\log f(x) = -qx\log (1 - 1/x)\\
		\frac{f'(x)}{f(x)} = -q\log(1 - 1/x) - qx\frac{1}{x^2}\frac{x}{x - 1} = -q\left(\log (1 - 1/x) + \frac{1}{x - 1}\right)\\
		g(x) > 0より単調減少
	\end{eqnarray*}
	\begin{eqnarray*}
		% c_{n + 1} = \frac{p(c_n)^{1/q}}{p(1 - q^{-n})}	= \frac{c_n{}^{1/q}}{1 - q^{-n}}
		% \frac{d_{n + 1}}{d_n} < e^{q}\\
		% d_{n + 1} < d_n e^{q}\\
		% \therefore d_{n + 1} < d_2 e^{(n - 1)q}\\
		% c_{n + 1}^{q^{n + 1}} < \left(\frac{c^{1/q}}{1 - q^{-n}}\right)^{q} e^{nq}\\
		% c_{n + 1} < c^{\frac{1}{q^{n}}} e^{\frac{n}{q^{n}}}\\
		\frac{d_{n + 1}}{d_n} < 1\, \Leftrightarrow \, d_{n + 1} < d_n\\
		d_{n + 1} < d_2\\
		(c_{n + 1})^{q^{n + 1}} < \left(\frac{c^{1/q}}{1 - q^{-2}}\right)^{q^{2}}\\
		c_{n + 1} < \left(\frac{c^{1/q}}{1 - q^{-2}}\right)^{\displaystyle \frac{q^2}{q^{n + 1}}}
	\end{eqnarray*}
	\begin{eqnarray*}
		c_{n + 1} > c_{n}{}^{1/q}\\
		\therefore c_{n + 1} > c_{1}{}^{(1/q)^{n - 1}} = c^{(1/q)^{n - 1}}\\
		\therefore c^{(1/q)^{n - 1}} < c_{n + 1} < \left(\frac{c^{1/q}}{1 - q^{-2}}\right)^{\displaystyle \frac{q^2}{q^{n + 1}}}
	\end{eqnarray*}
	\item $(2), (3)$より
	\begin{eqnarray*}
		\lim_{n \to \infty}a_n = p, \lim_{n \to \infty}c_n = 1より\\
		\lim_{n \to \infty}f_n(x) = \lim_{n \to \infty}c_nx^{a_n} = x^p
	\end{eqnarray*}
\end{enumerate}
\newpage
\section{}%第3問
\subsection{問題文}
赤いカードが2枚と白いカードが1枚入った袋および複素数$z_{n},\,w_{n}\ (n=0,1,2,...)$について考える. まず, 袋から1枚のカードを取り出し袋に戻す. このとき取り出されたカードの色に応じて$z_{k+1}\ (k=0,1,2,...)$を以下のルールで生成する.
\begin{eqnarray*}
	z_{k+1}=\left\{\begin{array}{l}iz_{k}\hspace{30pt} 赤いカードが取り出された場合\\-iz_{k}\hspace{22pt} 白いカードが取り出された場合\end{array}\right.
\end{eqnarray*}
次に, 袋からもう一度1枚のカードを取り出し袋に戻す. このとき取り出したカードの色に応じて$w_{k+1}$を以下のルールで生成する.
\begin{eqnarray*}
	w_{k+1}=\left\{\begin{array}{l}-iw_{k}\hspace{22pt} 赤いカードが取り出された場合\\iw_{k}\hspace{30pt} 白いカードが取り出された場合\end{array}\right.
\end{eqnarray*}
ここで, 各カードは独立に等確率で取り出されるものとする. また初期状態を$z_{0}=1,\,w_{0}=1$とする. すなわち, $z_{n},\,w_{n}$は, $z_{0}=1,\,w_{0}=1$の状態から始め, 上記の一連の二つの操作を$n$回繰り返した後の値である. なお, ここでは$i$は虚数単位とする.\\[0.5cm]
\hspace{10pt} 以下の問いに答えよ.

\begin{enumerate}[(1)]
	\setlength{\itemindent}{10pt}
	      \setlength{\itemsep}{10pt}
	\item $n$が奇数のとき${\rm Re}(z_{n})=0$, 偶数のとき${\rm Im}(z_{n})=0$であることを示せ. ただし, ${\rm Re}(z),\,{\rm Im}(z)$はそれぞれ$z$の実部, 虚部を表すものとする.
	\item $z_{n}=1$である確率を$P_{n},\ z_{n}=i$である確率を$Q_{n}$とする. $P_{n},\ Q_{n}$についての漸化式を立てよ.
	\item $z_{n}=1,\ z_{n}=i,\ z_{n}=-1,\ z_{n}=-i$である確率をそれぞれ求めよ.
	\item $z_{n}$の期待値が$(i/3)^{n}$であることを示せ.
	\item $z_{n}=w_{n}$である確率を求めよ.
	\item $z_{n}+w_{n}$の期待値を求めよ.
	\item $z_{n}w_{n}$の期待値を求めよ.
\end{enumerate}
\newpage

\subsection{解答例}
\begin{enumerate}[(1)]
	\setlength{\itemsep}{10pt}
	\item 赤いカードが2回連続で出されたら
	      \begin{eqnarray*}
		      z_{k+2}=-z_{k}^{2}
	      \end{eqnarray*}
	      となり, これは白いカードが2回連続で出された場合も同様である. 赤いカードが取り出され, 白いカードが取り出された場合
	      \begin{eqnarray*}
		      z_{k+2}=z_{k}^{2}
	      \end{eqnarray*}
	      となり, これは白いカードが取り出され, 赤いカードが取り出された場合も同様である. ゆえに,
	      \begin{eqnarray*}
		      z_{k+2}=\left\{
		      \begin{array}{l}
			      -z_{k}^{2}\hspace{22pt} 同じカードが2回連続で取り出された場合 \\
			      z_{k}^{2}\hspace{30pt} 異なるカードが取り出された場合
		      \end{array}
		      \right.
	      \end{eqnarray*}
		  ここで, $n=0$のとき$z_{0}=1$であるので, $z_{2}$は$1$または$-1$の値をとる. 帰納的に$n=2k$のとき$z_{n}$は1または$-1$の値をとるので, ${\rm Im}(z_{n})=0$となる. 一方で, $n=1$のとき, $z_{1}$は$i$または$-i$である.帰納的に$n=2k+1$のとき$z_{n}$は$i$または$-i$の値をとるので, ${\rm Re}(z_{n})=0$となる. よって, 題意は示された.
	\item $(1)$より,$n + 1$回目に$1$となるのは$n$回目に$i$で$n + 1$回目に白いカードを引く時と$n$回目に$-i$で$n + 1$回目に赤いカードを引くときである.
		  また,$n + 1$回目に$i$となるのは$n$回目に$-1$で$n + 1$回目に白いカードを引く時と$n$回目に$1$で$n + 1$回目に赤いカードを引くときである.
		  よって,以下のようになる.
		  \begin{eqnarray*}
			  P_{n + 1} = \frac{1}{3}Q_{n} + \frac{2}{3}(1 - Q_{n}) = \frac{2}{3} - \frac{1}{3}Q_{n}\\
			  Q_{n + 1} = \frac{1}{3}(1 - P_{n}) + \frac{2}{3}P_{n} = \frac{1}{3} + \frac{1}{3}P_{n}\\
		  \end{eqnarray*}
		  \item (2)より,
		  \begin{eqnarray*}
			P_{n + 2} &=& \frac{5}{9} - \frac{1}{9}P_{n}\\
			Q_{n + 2} &=& \frac{5}{9} - \frac{1}{9}Q_{n}
		  \end{eqnarray*}
		  $n = 2k + 1$の時
		  \begin{eqnarray*}
			Q_{2k + 3} &=& \frac{5}{9} - \frac{1}{9}Q_{2k + 1}\\
			変形して,\qquad
			Q_{2k + 3} - \frac{1}{2}&=& -\frac{1}{9}\left(Q_{2k + 1} - \frac{1}{2}\right)\\
			\therefore Q_{2k + 1} - \frac{1}{2} &=& \left(-\frac{1}{9}\right)^{k}\left(Q_{1} - \frac{1}{2}\right)\\
			\therefore Q_{2k + 1} &=& \left(-\frac{1}{9}\right)^{k}\frac{1}{6} + \frac{1}{2}\\
		  \end{eqnarray*}
		  $n = 2k$の時,漸化式は等しいので,同様にして
		  \begin{eqnarray*}
			P_{2k} &=& \left(-\frac{1}{9}\right)^{k}\left(P_{0} - \frac{1}{2}\right) + \frac{1}{2}\\
			&=& \left(-\frac{1}{9}\right)^{k}\left(\frac{1}{2}\right) + \frac{1}{2}
		  \end{eqnarray*}
		  よって,
		  \begin{eqnarray*}
			  P_{n} 
			  &=& \begin{cases}
				\left(-\frac{1}{9}\right)^{\frac{n}{2}}\left(\frac{1}{2}\right) + \frac{1}{2}  
				= \frac{1}{2}\left\{\left(\frac{i}{3}\right)^{n} + 1\right\} & n = 2k\\
				0 & n = 2k + 1\\
			  \end{cases}\\
			  Q_{n} 
			  &=& \begin{cases}
				\left(-\frac{1}{9}\right)^{\frac{n - 1}{2}}\frac{1}{6} + \frac{1}{2}
				= \left(\frac{i}{3}\right)^{n - 1}\frac{1}{6} + \frac{1}{2} = \frac{1}{2}\left\{\left(\frac{i}{3}\right)^{n - 1}\frac{1}{3} + 1\right\}
				= \frac{1}{2}\left\{-i\left(\frac{i}{3}\right)^{n} + 1\right\} & n = 2k + 1\\
				0 & n = 2k\\
			  \end{cases}\\
		  \end{eqnarray*}
		  従って,題意の回答は以下のようになる.
		  \begin{eqnarray}
			  \begin{cases} 
				\begin{cases}
				  P_n = \frac{1}{2}\left\{\left(\frac{i}{3}\right)^{n} + 1\right\} & n = 2k\\
				  0 & n = 2k + 1\\
				\end{cases} & z_n = 1の確率\\
				\begin{cases}
				  1 - P_n = 1 - \frac{1}{2}\left\{\left(\frac{i}{3}\right)^{n} + 1\right\} 
				  = \frac{1}{2}\left\{-\left(\frac{i}{3}\right)^{n} + 1\right\} & n = 2k\\
				  0 & n = 2k + 1\\
				\end{cases} & z_n = -1の確率\\
				\begin{cases}
				  0 & n = 2k\\
				  Q_n = \frac{1}{2}\left\{-i\left(\frac{i}{3}\right)^{n} + 1\right\} & n = 2k + 1\\
				\end{cases} & z_n = iの確率\\
				\begin{cases}
				  0 & n = 2k\\
				  1 - Q_n = 1 - \frac{1}{2}\left\{-i\left(\frac{i}{3}\right)^{n} + 1\right\}
				  = \frac{1}{2}\left\{i\left(\frac{i}{3}\right)^{n} + 1\right\} & n = 2k + 1\\
				\end{cases} & z_n = -iの確率\\
			  \end{cases} 
		  \end{eqnarray}
	\item $(3)$より$z_n$の期待値は以下のようになる.
	\begin{eqnarray*}
		n = 2kのとき期待値はP_n + (-1)(1 - P_n) = 2P_n - 1 = \left(\frac{i}{3}\right)^{n} + 1 - 1 = \left(\frac{i}{3}\right)^{n}\\
		n = 2k + 1のとき期待値は iQ_n + (-i)(1 - Q_n) = 2iQ_n - i = \left(\frac{i}{3}\right)^{n} + i - i = \left(\frac{i}{3}\right)^{n}
	\end{eqnarray*}
	\item $w_n$についても同様に確率$P_n', Q_n'$をおくと,漸化式は以下のようになる.
	\begin{eqnarray*}
		P_{n + 1}' &=& \frac{2}{3}Q_{n}' + \frac{1}{3}(1 - Q_{n}')\\
		 &=& \frac{1}{3} + \frac{1}{3}Q_{n}'\\
		Q_{n + 1}' &=& \frac{2}{3}(1 - P_{n}') + \frac{1}{3}P_{n}'\\
		 &=& \frac{2}{3} - \frac{1}{3}P_{n}'\\
		\therefore P_{n + 2}' &=& \frac{5}{9} - \frac{1}{9}P_{n}'\\
		Q_{n + 2}' &=& \frac{5}{9} - \frac{1}{9}Q_{n}'
	\end{eqnarray*}
	よって,$P_0' = 1, Q_1' = \frac{1}{3}$より
	\begin{eqnarray*}
		&&P_n' = P_n\\
		&&Q_n' = \frac{1}{2}(i\frac{i}{3}^{n} + 1) = 1 - Q_n\\
	\end{eqnarray*}
	従って,$z_n = w_n$が成り立つ確率は以下のようになる.\\
	$n = 2k$の時
	\begin{eqnarray*}
		P_n P_n' + (1 - P_n)(1 - P_n') &=& 2 - 2P_n + 2P_n^2\\
		&=& 1 + \frac{1}{2}\left\{\frac{i}{3}^{n} + 1\right\}^{2} - \left\{\frac{i}{3}^{n} + 1\right\}\\
		&=& \frac{1}{2}\left(\frac{i}{3}^{2n} + 1\right)\\
	\end{eqnarray*}
	$n = 2k + 1$の時
	\begin{eqnarray*}
	Q_n Q_n' + (1 - Q_n)(1 - Q_n') &=& 2Q_n(1 - Q_n)\\
	 &=& \frac{2}{2}\left(-i\frac{i}{3}^{n} + 1\right)\frac{1}{2}\left(i\frac{i}{3}^{n} + 1\right)\\
	 &=& \frac{1}{2}\left(1 + \frac{i}{3}^{2n}\right) \\
	\end{eqnarray*}
	よって,$z_n = w_n$となる確率は
	\begin{equation*}
		\frac{1}{2}\left(1 + \frac{i}{3}^{2n}\right)
	\end{equation*}
	\item $w_n$の期待値は$z_n$の期待値と同様にして,
	\begin{eqnarray*}
		n = 2kの時\frac{i}{3}^{n}\\
		n = 2k + 1の時-\frac{i}{3}^{n}
	\end{eqnarray*}
	$z_n + w_n$の期待値は
	\begin{eqnarray*}
		n = 2kの時2\frac{i}{3}^{n}\\
		n = 2k + 1の時0\\
	\end{eqnarray*}
	\item $z_nw_n$の期待値は
	\begin{eqnarray*}
		n = 2kの時\frac{i}{3}^{2n}\\
		n = 2k + 1の時-\frac{i}{3}^{2n}
	\end{eqnarray*}
\end{enumerate}
\index{tikz@ティック}
\end{document}