\subsection{問題文}
実数値関数$u(x, t)$が$0\, \leq \, x\, \leq\, 1, \; t\, \geq \, 0$で定義されている。
ここで$x$と$t$は互いに独立である。偏微分方程式
\begin{equation*}
    \pdiff{u}{t} = \pdiff[2]{u}{x} \qquad (*)
\end{equation*}
の解を次の条件
\begin{align*}
    \mbox{境界条件:} & \quad u(0, t) = u(1, t) = 0\\
    \mbox{初期条件:} & \quad u(x, 0) = x - x^{2}
\end{align*}
のもとで求める。但し、定数関数$u(x, t) = 0$は明らかに解であるから、それ以外の解を考える。
以下の問いに答えよ。
\begin{enumerate}[(1)]
    \item 次の式を計算せよ。ここで, $n, m$はともに正の整数とする。
        \begin{equation*}
            \dint{0}{1}{\sin (n\pi x)\; \sin (m\pi x)}
        \end{equation*}\label{subsec:prom2:subprom1}
    \item $x$のみの関数$\xi (x)$及び$t$のみの関数$\tau (t)$を用いて、$u(x, t) = \xi (x)\tau (t)$と置けるとする。
        任意の定数$C$を用いて、$\xi$および$\tau$が満たす常微分方程式をそれそれ表せ。関数$f(x)$と関数$g(t)$が任意の
        $x$と$t$について$f(x) = g(t)$を満たす場合は、$f(x)$と$g(t)$が定数関数となることを用いてもよい。\label{subsec:prom2:subprom2}
    \item 設問\eqref{subsec:prom2:subprom2}の常微分方程式を解け。次に、境界条件を満たす偏微分方程式$(*)$の解の一つが次の式で表される$u_n(x, t)$
        で与えられることを示し、$\alpha, \beta$を正の整数$n$を用いて表せ。
        \begin{equation*}
            u_n (x, t) = e^{\alpha t}\sin (\beta x)
        \end{equation*}\label{subsec:prom2:subprom3}
    \item 境界条件と初期条件を満たす偏微分方程式$(*)$の解は$u_n (x, t)$の線形結合として次の式で表される。$c_n$を求めよ。設問\eqref{subsec:prom2:subprom1}の結果を
        用いてもよい。
        \begin{equation*}
            u(x, t) = \sum\limits_{n = 1}^{\infty} c_n u_n (x, t)
        \end{equation*}\label{subsec:prom2:subprom4}
\end{enumerate}
\newpage
\subsection{解答}
\begin{enumerate}[(1)]
    \item 題意の式より以下が成り立つ。
        \begin{align*}
            \mbox{(与式)} 
            & = \frac{1}{2}\dint{0}{1}{\cos (n\pi x - m\pi x) - \cos (n\pi x + m\pi x)}\\
            & = 
            \begin{cases}
                \frac{1}{2}\dint{0}{1}{1 - \cos (2n\pi x)} & n = m\\
                \frac{1}{2}\left[\frac{\sin ((n - m)\pi x)}{(n - m)\pi}\right]_{0}^{1} - \frac{1}{2}\left[\frac{\sin ((n + m)\pi x)}{(n + m)\pi}\right]_{0}^{1} & n \neq m\\
            \end{cases}\\
            & = 
            \begin{cases}
                \frac{1}{2} - \frac{1}{2}\left[\frac{\sin (2n\pi x)}{2n\pi}\right]_{0}^{1} & n = m\\
                0 & n \neq m\\
            \end{cases}\\
            & = 
            \begin{cases}
                \frac{1}{2} & n = m\\
                0 & n \neq m\\
            \end{cases}
        \end{align*}
    \item 偏微分方程式$(*)$より$u(x, t) = \xi (x)\tau (t)$と表せるとすると以下が成り立つ。
        \begin{align*}
            (*) \Longleftrightarrow
            \pdiff{\{\xi (x)\tau (t)\}}{t} & = \pdiff[2]{\{\xi (x)\tau (t)\}}{x}\\
            \Longleftrightarrow
            \xi (x)\diff{\tau (t)}{t} & = \tau (t)\diff[2]{\xi (x)}{x}
        \end{align*}
        ここで、$\xi (x)\tau(t) \neq 0$より$\xi (x) \neq 0, \tau (t) \neq 0$よって、以下のようになる。
        \begin{align*}
            \frac{\diff[2]{\xi (x)}{x}}{\xi (x)} &= \frac{\diff{\tau (t)}{t}}{\tau (t)}\\
            \therefore &
            \begin{cases}
                \frac{\diff[2]{\xi (x)}{x}}{\xi (x)} = C\\
                \frac{\diff{\tau (t)}{t}}{\tau (t)} = C\\
            \end{cases}\\
            \Longleftrightarrow &
            \begin{cases}
                C\xi (x) = \diff[2]{\xi (x)}{x}\\
                C\tau (t) = \diff{\tau (t)}{t}\\
            \end{cases}\\
        \end{align*}
    \item $(2)$よりそれぞれ解くと以下のようになる。\\
    $\tau (t)$に関する常微分方程式を解く。$\tau (t) \neq 0$より以下が成り立つ。
        \begin{align*}
            C &= \frac{1}{\tau (t)}\diff{\tau (t)}{t}
        \end{align*}
        よって両辺$t$で積分して、
        \begin{align*}
            \dint[t]{}{}{C} &= \dint[t]{}{}{\frac{1}{\tau (t)}\diff{\tau (t)}{t}}\\
            \dint[t]{}{}{C} &= \dint[\tau]{}{}{\frac{1}{\tau}} \qquad(分かりやすくするため(t)を省略)\\
            Ct + C_3 &= \log |\tau (t)|\\
            |\tau (t)| &= e^{Ct + C_3}\\
            \tau (t) &= e^{Ct + C_3}
        \end{align*}
        よって、一般解は以下のようになる。
        \begin{equation*}
            \tau (t) = e^{Ct + C_3}
        \end{equation*}
        ここで常微分方程式から$t\to \infty$の時、$\tau (t)$は発散しないが、$C > 0$とすると、
        この一般解の式から$\tau (t)$は発散するので矛盾してしまう。よって、$C < 0$となるので、正の実数$k$を用いて、$C = -k^2$とおく。\\
    次に$\xi (x) = C_1 e^{\lambda x}$とおくと、$(2)$の常微分方程式と、$C = -k^2$より
        \begin{align*}
            -k^2 C_1 e^{\lambda x} &= \lambda^2 C_1 e^{\lambda x}\\
            -k^2 &= \lambda^2 \\
            \therefore \lambda &= \pm \imag k
        \end{align*}
    よって、斉次の微分方程式より独立な2つの解の和も解となるので、$\xi(x)$の一般解は以下のようになる。
        \begin{align*}
            \xi (x) &= C_1 e^{\imag kx} + C_2 e^{-\imag kx}\\
            \xi (x) &= (C_1 + C_2)\cos (kx) + \imag(C_1 - C_2)\sin (kx)
        \end{align*}
        よって、$C_1 + C_2 = A \in \mathbb{R}, \imag(C_1 - C_2) = B\in\mathbb{R}$とおくと以下のようになる。
        \begin{align*}
            \xi (x) &= A\cos (kx) + B\sin (kx)
        \end{align*}
        次に境界条件を満たす偏微分方程式$(*)$の解の一つに題意の式$u_n(x, t)$が存在することを示す。\\
        $(2)$より$u(x, t) = \xi (x)\tau (t) = e^{-k^2 t + C_3}(A\cos (kx) + B\sin (kx))$となる。\\
        よって、この時初期条件より
        \begin{align*}
            \tau (0) &= 1\\
            e^{C_3} &= 1\\
            \therefore C_3 &= 0\\
            \therefore \tau (t) &= e^{-k^2t}
        \end{align*}
        境界条件より、
        \begin{align*}
            &\begin{cases}
                \xi(0) = 0\\
                \xi(1) = 0\\
            \end{cases}\\
            &\begin{cases}
                A\cos (0) + B\sin (0) = 0\\
                A\cos (k) + B\sin (k) = 0\\
            \end{cases}\\
            &\begin{cases}
                A = 0\\
                B\sin (k) = 0\\
            \end{cases}\\
        \end{align*}
        よって、$\xi (x) \neq 0$より$B \neq 0$より、
        \begin{align*}
            \sin (k) &= 0\\
            \therefore k = n\pi
        \end{align*}
        \begin{equation*}
            \xi (x) = B\sin (n\pi x)
        \end{equation*}
        となる。よって、この時、$(2)$より、$u(x, t) = \xi (x)\tau (t)$は偏微分方程式$(*)$を満たすので、以下の式はこの偏微分方程式の解の一つである。
        \begin{equation*}
            u(x, t) = Be^{-n^2\pi^2t}\sin(n\pi x)
        \end{equation*}
        よって、この式の$B = 1, n\pi = \beta, -n^2\pi^2 = \alpha$とおくと、
        \begin{align*}
            e^{\alpha t}\sin(\beta x) = u_n(x, t)
        \end{align*}
        となるので、題意は示された。また、$\alpha = -n^2\pi^2, \beta = n\pi$となる。
    \item 題意のように偏微分方程式$(*)$は線形であるため、その解は$u_n(x, t)$の線形結合で表される。よって、$(3)$より以下のようになる。
    \begin{align*}
        初期条件から、u(x, 0) &= x - x^2 = \sum_{n = 1}^{\infty}c_n\sin(n\pi x)\\
        \dint{0}{1}{(x - x^2)\sin(m\pi x)} &= \dint{0}{1}{\left(\sum_{n = 1}^{\infty}c_n\sin(n\pi x)\right)\sin (m\pi x)}
    \end{align*}
    よって、それぞれ計算する。
    \begin{align*}
        m = 0の時、
        (左辺) = 0&\\
        m \neq 0の時、\qquad
        \dint{0}{1}{x\sin(m\pi x)} &= \left[x\frac{-\cos(m\pi x)}{m\pi}\right]_{0}^{1} + \frac{1}{m\pi}\dint{0}{1}{\cos(m\pi x)}\\
        &= \frac{(-1)^{m + 1}}{m\pi} + \frac{1}{m^2\pi^2}\bigl[\sin (m\pi x)\bigr]_{0}^{1} = \frac{(-1)^{m + 1}}{m\pi}\\
        \dint{0}{1}{x\cos(m\pi x)} &= \left[x\frac{\sin(m\pi x)}{m\pi}\right]_{0}^{1} - \frac{1}{m\pi}\dint{0}{1}{\sin(m\pi x)}\\
        &= -\frac{1}{m^2\pi^2}\bigl[-\cos (m\pi x)\bigr]_{0}^{1} = \frac{(-1)^{m} - 1}{m^2\pi^2}\\
        \dint{0}{1}{x^2\sin(m\pi x)} &= \left[x^2\frac{-\cos(m\pi x)}{m\pi}\right]_{0}^{1} + \frac{2}{m\pi}\dint{0}{1}{x\cos(m\pi x)}\\
        &= \frac{(-1)^{m + 1}}{m\pi} + \frac{2}{m\pi}\frac{(-1)^{m} - 1}{m^2\pi^2}\\
        (左辺) &= \dint{0}{1}{x\sin(m\pi x)} - \dint{0}{1}{x^2\sin(m\pi x)}\\
        &= \frac{(-1)^{m + 1}}{m\pi} - \left(\frac{(-1)^{m + 1}}{m\pi} + \frac{2}{m\pi}\frac{(-1)^{m} - 1}{m^2\pi^2}\right)\\
        &= \frac{2 - 2(-1)^{m}}{m^3\pi^3}
    \end{align*}
    $(1)$より右辺に関しては以下のようになる。
    \begin{align*}
        (右辺) &= \sum_{n = 1}^{\infty}c_n \dint{0}{1}{\sin(n\pi x)\sin(m\pi x)}\\
        &= \frac{1}{2}c_m
    \end{align*}
    従って、以下が成り立つ。
    \begin{align*}
        \frac{2 - 2(-1)^{m}}{m^3\pi^3} &= \frac{1}{2}c_m\\
        c_m &= \frac{4\left\{1 - (-1)^{m}\right\}}{m^3\pi^3}\\
    \end{align*}
    従って求める解答は以下のようになる。
    \begin{equation*}
        c_n = \frac{4\left\{1 - (-1)^{n}\right\}}{n^3\pi^3}
    \end{equation*}
\end{enumerate}
