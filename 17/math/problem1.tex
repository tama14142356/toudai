\subsection{問題文}
3次元ベクトル$\spalignmat[c]{
    {x_n};
    {y_n};
    {z_n}
}$は式
\begin{equation*}
    \spalignmat[c]{
        {x_{n + 1}};
        {y_{n + 1}};
        {z_{n + 1}}
    }
    = 
    A \spalignmat[c]{
        {x_n};
        {y_n};
        {z_n}
    }\qquad (n \, = \, 0, 1, 2, \cdots)
\end{equation*}
を満たすものとする。但し、$x_0, y_0, z_0, \alpha$は実数とし、
\begin{equation*}
    A = \spalignmat[c]{
        {1 - 2\alpha} {\alpha} {\alpha};
        {\alpha} {1 - \alpha} {0};
        {\alpha} {0} {1 - \alpha}
    }, \qquad 0 < \alpha < \frac{1}{3}
\end{equation*}
とする。以下の問いに答えよ。
\begin{enumerate}[(1)]
  \item $x_n + y_n + z_n$を$x_0, y_0, z_0$を用いて表せ。
  \item 行列$A$の固有値$\lambda_1, \lambda_2, \lambda_3$と、それぞれの固有値に対応する固有ベクトル
        $\mathbold{v}_1, \mathbold{v}_2, \mathbold{v}_3$を求めよ。
  \item 行列$A$を$\lambda_1, \lambda_2, \lambda_3, \mathbold{v}_1, \mathbold{v}_2, \mathbold{v}_3$を用いて表せ。
  \item $\spalignmat[c]{
    {x_n};
    {y_n};
    {z_n}
  }$を$x_0, y_0, z_0, \alpha$を用いて表せ。
  \item $\lim\limits_{n \to \infty} \spalignmat[c]{
      {x_n};
      {y_n};
      {z_n}
  }$を求めよ。
  \item 以下の式
  \begin{equation*}
	  f(x_0, y_0, z_0)\, = \, \frac{
		  (x_n\;y_n\; z_n)
	  \spalignmat[c]{
		  {x_{n + 1}};
		  {y_{n + 1}};
		  {z_{n + 1}}
	  }}{(x_n\;y_n\; z_n)\spalignmat[c]{
		  {x_n};
		  {y_n};
		  {z_n}
	  }}
  \end{equation*}
  を$x_0, y_0, z_0$の関数とみなして、$f(x_0, y_0, z_0)$の最大値及び最小値を求めよ。但し、
  $x_0^{2} + y_0^{2} + z_0^{2}\, \neq \, 0$
\end{enumerate}
\newpage
\subsection{解答}
\begin{enumerate}[(1)]
	\item 題意より以下が成り立つ。
		\begin{align*}
			\spalignmat[c]{
				{x_{n + 1}};
				{y_{n + 1}};
				{z_{n + 1}}
			} 
			& = A
			\spalignmat[c]{
				{x_n};
				{y_n};
				{z_n}
			}\\
			& = \spalignmat[c]{
				{1 - 2\alpha} {\alpha} {\alpha};
				{\alpha} {1 - \alpha} {0};
				{\alpha} {0} {1 - \alpha}
			}
			\spalignmat[c]{
				{x_n};
				{y_n};
				{z_n}
			}\\
			& = \spalignmat[c]{
				{(1 - 2\alpha)x_n + \alpha y_n + \alpha z_n};
				{\alpha x_n + (1 - \alpha)y_n};
				{\alpha x_n + (1 - \alpha)z_n}
			}\\
			\therefore
			x_{n + 1} + y_{n + 1} + z_{n + 1} 
			&=  (1 - 2\alpha)x_n + \alpha y_n + \alpha z_n + \alpha x_n + (1 - \alpha)y_n + \alpha x_n + (1 - \alpha)z_n\\
			&=  x_n + y_n + z_n\\
			\therefore
			x_n + y_n + z_n &= x_0 + y_0 + z_0
		\end{align*}
	\item まずAの固有値を求める。
		\begin{align*}
			\spaligndelims\vert\vert \spalignmat{{\lambda I - A}}
			& = \spaligndelims\vert\vert \spalignmat{
			{\lambda - 1 + 2\alpha} {-\alpha} {-\alpha};
			{-\alpha} {\lambda - 1 + \alpha} {0};
			{-\alpha} {0} {\lambda - 1 + \alpha};
			}\\
			& = \frac{1}{\alpha}\spaligndelims\vert\vert \spalignmat{
			{(\lambda - 1 + 2\alpha)} {-\alpha} {-\alpha};
			{-\alpha} {\lambda - 1 + \alpha} {0};
			{-\alpha^{2} + (\lambda - 1 + 2\alpha)(\lambda - 1 + \alpha)} {-\alpha(\lambda - 1 + \alpha)} {0};
			}\\
			& = (-1)^{1 + 3}(-1)\spaligndelims\vert\vert \spalignmat{
			{-\alpha} {\lambda - 1 + \alpha};
			{-\alpha^{2} + (\lambda - 1 + 2\alpha)(\lambda - 1 + \alpha)} {-\alpha(\lambda - 1 + \alpha)};
			}\\
			& = -\left\{\alpha^{2}(\lambda - 1 + \alpha) - (\lambda - 1 + \alpha)(-\alpha^{2} + (\lambda - 1 + 2\alpha)(\lambda - 1 + \alpha))\right\} \\
			& = -(\lambda - 1 + \alpha)\left\{\alpha^{2} - (-\alpha^{2} + (\lambda - 1 + 2\alpha)(\lambda - 1 + \alpha))\right\}                       \\
			& = -(\lambda - 1 + \alpha)\left\{ 2\alpha^{2} - (\lambda - 1)^2 - 2\alpha^2 - 3\alpha(\lambda - 1))\right\}                               \\
			& = (\lambda - 1 + \alpha)(\lambda - 1)(\lambda - 1 + 3\alpha)
		\end{align*}
		よって、$\lambda = \alpha - 1, 1, 1 - 3\alpha$より、$\lambda_1 = - 1 + \alpha, \lambda_2 = 1, \lambda_3 = 1 - 3\alpha$\\
		従って固有空間$W(\lambda; A) = \{\mathbold{v} | \left(\lambda I - A\right)\mathbold{v} = \mathbold{0}\}$は以下のようになる。
		\begin{align*}
			W(\alpha - 1; A)  
			% & = \left\{\mathbold{v} \left\lvert \spalignmat{
			% {\alpha} {-\alpha} {-\alpha};
			% {-\alpha} {0} {0};
			% {-\alpha} {0} {0}
			% }\mathbold{v} = \mathbold{0}\right. \right\}\\
			% & = \left\{\mathbold{v} \left\lvert \spalignmat{
			% {1} {0} {0};
			% {0} {-1} {-1};
			% {0} {0} {0}
			% }\mathbold{v} = \mathbold{0}\right. \right\}\\
			& = \left\{\mathbold{v} \left\lvert
			\mathbold{v} =\spalignmat{
			{0};
			{s};
			{-s}
			} = s\spalignmat{
			{0};
			{1};
			{-1}
			}\right. \right\}
			\mbox{よって、}\mathbold{v}_1 = s\spalignmat{
			{0};
			{1};
			{-1}
			}\\
			W(1; A)
			% & = \left\{\mathbold{v} \left\lvert \spalignmat{
			% {2\alpha} {-\alpha} {-\alpha};
			% {-\alpha} {\alpha} {0};
			% {-\alpha} {0} {\alpha}
			% }\mathbold{v} = \mathbold{0}\right. \right\}\\
			% & = \left\{\mathbold{v} \left\lvert \spalignmat{
			% {0} {0} {0};
			% {-1} {1} {0};
			% {-1} {0} {1}
			% }\mathbold{v} = \mathbold{0}\right. \right\}\\
			& = \left\{\mathbold{v} \left\lvert
			\mathbold{v} =\spalignmat{
			{s};
			{s};
			{s}
			} = s\spalignmat{
			{1};
			{1};
			{1}
			}\right. \right\}
			\mbox{よって、}\mathbold{v}_2 = t\spalignmat{
			{1};
			{1};
			{1}
			}\\
			W(1 - 3\alpha; A)
			% & = \left\{\mathbold{v} \left\lvert \spalignmat{
			% {-\alpha} {-\alpha} {-\alpha};
			% {-\alpha} {-2\alpha} {0};
			% {-\alpha} {0} {-2\alpha}
			% }\mathbold{v} = \mathbold{0}\right. \right\}\\
			% & = \left\{\mathbold{v} \left\lvert \spalignmat{
			% {1} {2} {0};
			% {0} {1} {-1};
			% {0} {0} {0}
			% }\mathbold{v} = \mathbold{0}\right. \right\}\\
			& = \left\{\mathbold{v} \left\lvert
			\mathbold{v} =\spalignmat{
			{-2s};
			{s};
			{s}
			} = s\spalignmat{
			{-2};
			{1};
			{1}
			}\right. \right\}
			\mbox{よって、}\mathbold{v}_3 = u\spalignmat{
			{-2};
			{1};
			{1}
			}
		\end{align*}
		よって、$\lambda_1 = -1 + \alpha, \lambda_2 = 1, \lambda_3 = 1 - 3\alpha, 
		\mathbold{v}_1 = s\spalignmat[c]{
			   {0};
			   {1};
			   {-1}
		},
		\mathbold{v}_2 = t\spalignmat[c]{
			   {1};
			   {1};
			   {1}
		},
		\mathbold{v}_3 = u\spalignmat[c]{
			   {-2};
			   {1};
			   {1}
		}$\\
		但し、$s, t, u$は0でない任意の実数とする。
	\item  題意より、$A$は対称行列であるため、直交行列で対角化できる。また、異なる固有値の基底ベクトルは互いに直交する。
		よって、以下は正規直交基底ベクトル集合$W$となる。
		\begin{align*}
			W = \left\{\cfrac{\mathbold{v}_1}{\left\lvert \mathbold{v}_1\right\rvert}, \, \cfrac{\mathbold{v}_2}{\left\lvert \mathbold{v}_2\right\rvert}, \, \cfrac{\mathbold{v}_3}{\left\lvert \mathbold{v}_3\right\rvert}\right\}
		\end{align*}
		よって、この集合の要素を並べたものは直交行列となり、その直交行列$P$を
		$P = \left(\cfrac{\mathbold{v}_1}{\left\lvert\mathbold{v}_1 \right\rvert}\, \cfrac{\mathbold{v}_2}{\left\lvert\mathbold{v}_2 \right\rvert}\, \cfrac{\mathbold{v}_3}{\left\lvert\mathbold{v}_3 \right\rvert}\right)$とおくと、
		$P$は直交行列より、$P^{-1} = P^{\top} =\spalignmat[c]{
			{\cfrac{\mathbold{v}_1{}^{\top}}{\abs{\mathbold{v}_1}}};
			{\cfrac{\mathbold{v}_2{}^{\top}}{\abs{\mathbold{v}_2}}};
			{\cfrac{\mathbold{v}_3{}^{\top}}{\abs{\mathbold{v}_3}}}
			}$となる。よって、以下が成り立つ。
		\begin{align*}
			% AP 
			% & = P\spalignmat[c]{
			% {\lambda_1} {0} {0};
			% {0} {\lambda_2} {0};
			% {0} {0} {\lambda_3}
			% }\\
			% \Leftrightarrow
			A  & = P\spalignmat[c]{
			{\lambda_1} {0} {0};
			{0} {\lambda_2} {0};
			{0} {0} {\lambda_3}
			}P^{-1}\\
			\Leftrightarrow
			A  & = P\spalignmat[c]{
			{\lambda_1} {0} {0};
			{0} {\lambda_2} {0};
			{0} {0} {\lambda_3}
			}P^{\top}\\
			& = \left(\cfrac{\mathbold{v}_1}{\abs{\mathbold{v}_1}}\, \cfrac{\mathbold{v}_2}{\abs{\mathbold{v}_2}}\, \cfrac{\mathbold{v}_3}{\abs{\mathbold{v}_3}}\right)
			\spalignmat[c]{
			{\lambda_1} {0} {0};
			{0} {\lambda_2} {0};
			{0} {0} {\lambda_3}
			}\spalignmat[c]{
			{\cfrac{\mathbold{v}_1{}^{\top}}{\abs{\mathbold{v}_1}}};
			{\cfrac{\mathbold{v}_2{}^{\top}}{\abs{\mathbold{v}_2}}};
			{\cfrac{\mathbold{v}_3{}^{\top}}{\abs{\mathbold{v}_3}}}
			}\\
			& = \left(
				\lambda_1\cfrac{\mathbold{v}_1}{\abs{\mathbold{v}_1}}\;
				\lambda_2\cfrac{\mathbold{v}_2}{\abs{\mathbold{v}_2}}\;
				\lambda_3\cfrac{\mathbold{v}_3}{\abs{\mathbold{v}_3}}
			\right)
			\spalignmat[c]{
			{\cfrac{\mathbold{v}_1{}^{\top}}{\abs{\mathbold{v}_1}}};
			{\cfrac{\mathbold{v}_2{}^{\top}}{\abs{\mathbold{v}_2}}};
			{\cfrac{\mathbold{v}_3{}^{\top}}{\abs{\mathbold{v}_3}}}
			}\\
			% & = \spalignmat[c]{
			% 	{0} {\lambda_2} {-2\lambda_3};
			% 	{\lambda_1} {\lambda_2} {\lambda_3};
			% 	{-\lambda_1} {\lambda_2} {\lambda_3}
			% }\spalignmat[c]{
			% 	{0} {1} {-1};
			% 	{1} {1} {1};
			% 	{-2} {1} {1}
			% }\\
			% & = \spalignmat[c]{
			% 	{\lambda_2 + 4\lambda_3} {\lambda_2 - 2\lambda_3} {\lambda_2 - 2\lambda_3};
			% 	{\lambda_2 - 2\lambda_3} {\lambda_1 + \lambda_2 + \lambda_3} {-\lambda_1 + \lambda_2 + \lambda_3};
			% 	{\lambda_2 - 2\lambda_3} {-\lambda_1 + \lambda_2 + \lambda_3} {\lambda_1 + \lambda_2 + \lambda_3};
			% }\\
			% & = \left(
			% 		0\cdot\lambda_1\cfrac{\mathbold{v}_1}{\abs{\mathbold{v}_1}} 
			% 		\, + 1\cdot\lambda_2\cfrac{\mathbold{v}_2}{\abs{\mathbold{v}_2}}
			% 		\, - 2\cdot\lambda_3\cfrac{\mathbold{v}_3}{\abs{\mathbold{v}_3}}
			% 	\;
			% 		1\cdot\lambda_1\cfrac{\mathbold{v}_1}{\abs{\mathbold{v}_1}} 
			% 		\,+ 1\cdot\lambda_2\cfrac{\mathbold{v}_2}{\abs{\mathbold{v}_2}}
			% 		\,+ 1\cdot\lambda_3\cfrac{\mathbold{v}_3}{\abs{\mathbold{v}_3}}
			% \right.\\
			% &\qquad\left.
			% 		-1\cdot\lambda_1\cfrac{\mathbold{v}_1}{\abs{\mathbold{v}_1}} 
			% 		\, + 1\cdot\lambda_2\cfrac{\mathbold{v}_2}{\abs{\mathbold{v}_2}}
			% 		\, + 1\cdot\lambda_3\cfrac{\mathbold{v}_3}{\abs{\mathbold{v}_3}}
			% \right)\\
			& = \spalignmat[c]{
				{
					\cfrac{\lambda_2}{\sqrt{3}}\cfrac{\mathbold{v}_2}{\abs{\mathbold{v}_2}}
					\, - \cfrac{2\lambda_3}{\sqrt{6}}\cfrac{\mathbold{v}_3}{\abs{\mathbold{v}_3}}
				}
				{
					\cfrac{\lambda_1}{\sqrt{2}}\cfrac{\mathbold{v}_1}{\abs{\mathbold{v}_1}} 
					\,+ \cfrac{\lambda_2}{\sqrt{3}}\cfrac{\mathbold{v}_2}{\abs{\mathbold{v}_2}}
					\,+ \cfrac{\lambda_3}{\sqrt{6}}\cfrac{\mathbold{v}_3}{\abs{\mathbold{v}_3}}
				}
				{
					-\cfrac{\lambda_1}{\sqrt{2}}\cfrac{\mathbold{v}_1}{\abs{\mathbold{v}_1}} 
					\, + \cfrac{\lambda_2}{\sqrt{3}}\cfrac{\mathbold{v}_2}{\abs{\mathbold{v}_2}}
					\, + \cfrac{\lambda_3}{\sqrt{6}}\cfrac{\mathbold{v}_3}{\abs{\mathbold{v}_3}}
				}
			}
		\end{align*}
	\item 題意より、以下が成り立つ。
		\begin{align}
			\spalignmat[c]{
				{x_n};
				{y_n};
				{z_n}
			} = A^{n} \spalignmat[c]{
				{x_0};
				{y_0};
				{z_0}
			}\label{prom1:subprom4:eq1}
		\end{align}
		但し、$A^{n}$は$A$を$A$に対して右から$n$回かけたことを意味する。また、以下も同様の意味を表す。\\
		ここで、(3)より、以下が成り立つ。
		\begin{align*}
			P^{\top}AP 
			& = \spalignmat[c]{
				{\lambda_1} {0} {0};
				{0}	{\lambda_2} {0};
				{0} {0} {\lambda_3}
			}\\
			\Longleftrightarrow 
			(P^{\top}AP)^{n} 
			& = \spalignmat[c]{
				{\lambda_1} {0} {0};
				{0}	{\lambda_2} {0};
				{0} {0} {\lambda_3}
			}^{n}\\
			\Longleftrightarrow 
			P^{\top}A^{n}P 
			& = \spalignmat[c]{
				{\lambda_1{}^{n}} {0} {0};
				{0}	{\lambda_2{}^{n}} {0};
				{0} {0} {\lambda_3{}^{n}}
			}\\
			\Longleftrightarrow 
			A^{n} 
			& = P\spalignmat[c]{
				{\lambda_1{}^{n}} {0} {0};
				{0}	{\lambda_2{}^{n}} {0};
				{0} {0} {\lambda_3{}^{n}}
			}P^{\top}
		\end{align*}
		\begin{align*}
			\therefore 
			A^{n}
			&= \spalignmat[c]{
				{
					\cfrac{\lambda_2{}^{n}}{\sqrt{3}}\cfrac{\mathbold{v}_2}{\abs{\mathbold{v}_2}}
					\, - \cfrac{2\lambda_3{}^{n}}{\sqrt{6}}\cfrac{\mathbold{v}_3}{\abs{\mathbold{v}_3}}
				}
				{
					\cfrac{\lambda_1{}^{n}}{\sqrt{2}}\cfrac{\mathbold{v}_1}{\abs{\mathbold{v}_1}} 
					\,+ \cfrac{\lambda_2{}^{n}}{\sqrt{3}}\cfrac{\mathbold{v}_2}{\abs{\mathbold{v}_2}}
					\,+ \cfrac{\lambda_3{}^{n}}{\sqrt{6}}\cfrac{\mathbold{v}_3}{\abs{\mathbold{v}_3}}
				}
				{
					-\cfrac{\lambda_1{}^{n}}{\sqrt{2}}\cfrac{\mathbold{v}_1}{\abs{\mathbold{v}_1}} 
					\, + \cfrac{\lambda_2{}^{n}}{\sqrt{3}}\cfrac{\mathbold{v}_2}{\abs{\mathbold{v}_2}}
					\, + \cfrac{\lambda_3{}^{n}}{\sqrt{6}}\cfrac{\mathbold{v}_3}{\abs{\mathbold{v}_3}}
				}
			}
		\end{align*}
		ここで
		$\mathbold{u}_{1_{n}} =  
		\cfrac{\lambda_2{}^{n}}{\sqrt{3}}\cfrac{\mathbold{v}_2}{\abs{\mathbold{v}_2}}
		\, - \cfrac{2\lambda_3{}^{n}}{\sqrt{6}}\cfrac{\mathbold{v}_3}{\abs{\mathbold{v}_3}}$,
		$\mathbold{u}_{2_{n}} = 
		\cfrac{\lambda_1{}^{n}}{\sqrt{2}}\cfrac{\mathbold{v}_1}{\abs{\mathbold{v}_1}} 
					\,+ \cfrac{\lambda_2{}^{n}}{\sqrt{3}}\cfrac{\mathbold{v}_2}{\abs{\mathbold{v}_2}}
					\,+ \cfrac{\lambda_3{}^{n}}{\sqrt{6}}\cfrac{\mathbold{v}_3}{\abs{\mathbold{v}_3}}$,
		$\mathbold{u}_{3_{n}} = 
		-\cfrac{\lambda_1{}^{n}}{\sqrt{2}}\cfrac{\mathbold{v}_1}{\abs{\mathbold{v}_1}} 
					\, + \cfrac{\lambda_2{}^{n}}{\sqrt{3}}\cfrac{\mathbold{v}_2}{\abs{\mathbold{v}_2}}
					\, + \cfrac{\lambda_3{}^{n}}{\sqrt{6}}\cfrac{\mathbold{v}_3}{\abs{\mathbold{v}_3}}$と置くと、以下のようになる。
		\begin{align}
			A^{n} = \spalignmat[c]{
				{\mathbold{u}_{1_n}} {\mathbold{u}_{2_n}} {\mathbold{u}_{3_n}}
			}
			\label{prom1:subprom4:eq2}
		\end{align}
		よって、式\eqref{prom1:subprom4:eq1}, \eqref{prom1:subprom4:eq2}より、題意は以下のようになる。
		\begin{align*}
			\spalignmat[c]{
				{x_n};
				{y_n};
				{z_n}
			}
			& = \spalignmat[c]{
				{\mathbold{u}_{1_n}} {\mathbold{u}_{2_n}} {\mathbold{u}_{3_n}}
			}
			\spalignmat[c]{
				{x_0};
				{y_0};
				{z_0}
			}\\
			&= x_0\mathbold{u}_{1_n} + y_0\mathbold{u}_{2_n} + z_0\mathbold{u}_{3_n}\\
			&= x_0\left(\cfrac{\lambda_2{}^{n}}{\sqrt{3}}\cfrac{\mathbold{v}_2}{\abs{\mathbold{v}_2}}
			\, - \cfrac{2\lambda_3{}^{n}}{\sqrt{6}}\cfrac{\mathbold{v}_3}{\abs{\mathbold{v}_3}}\right)
			\; + y_0\left(\cfrac{\lambda_1{}^{n}}{\sqrt{2}}\cfrac{\mathbold{v}_1}{\abs{\mathbold{v}_1}} 
			\,+ \cfrac{\lambda_2{}^{n}}{\sqrt{3}}\cfrac{\mathbold{v}_2}{\abs{\mathbold{v}_2}}
			\,+ \cfrac{\lambda_3{}^{n}}{\sqrt{6}}\cfrac{\mathbold{v}_3}{\abs{\mathbold{v}_3}}\right) 
			\\ & + z_0\left(-\cfrac{\lambda_1{}^{n}}{\sqrt{2}}\cfrac{\mathbold{v}_1}{\abs{\mathbold{v}_1}} 
			\, + \cfrac{\lambda_2{}^{n}}{\sqrt{3}}\cfrac{\mathbold{v}_2}{\abs{\mathbold{v}_2}}
			\, + \cfrac{\lambda_3{}^{n}}{\sqrt{6}}\cfrac{\mathbold{v}_3}{\abs{\mathbold{v}_3}}\right)\\
			&= \cfrac{\lambda_1{}^{n}}{\sqrt{2}}(y_0 - z_0)\cfrac{\mathbold{v}_1}{\abs{\mathbold{v}_1}} 
			+ \cfrac{\lambda_2{}^{n}}{\sqrt{3}}(x_0 + y_0 + z_0)\cfrac{\mathbold{v}_2}{\abs{\mathbold{v}_2}}
			+ \cfrac{\lambda_3{}^{n}}{\sqrt{6}}(-2x_0 + y_0 + z_0)\cfrac{\mathbold{v}_3}{\abs{\mathbold{v}_3}}\\
			&= \cfrac{\lambda_1{}^{n}}{\sqrt{2}}(y_0 - z_0)\spalignmat[c]{
				{0};
				{\frac{1}{\sqrt{2}}};
				{-\frac{1}{\sqrt{2}}}
			}
			+ \cfrac{\lambda_2{}^{n}}{\sqrt{3}}(x_0 + y_0 + z_0)\spalignmat[c]{
				{\frac{1}{\sqrt{3}}};
				{\frac{1}{\sqrt{3}}};
				{\frac{1}{\sqrt{3}}}
			}
			+ \cfrac{\lambda_3{}^{n}}{\sqrt{6}}(-2x_0 + y_0 + z_0)\spalignmat[c]{
				{-\frac{2}{\sqrt{6}}};
				{\frac{1}{\sqrt{6}}};
				{\frac{1}{\sqrt{6}}}
			}\\
			&= \spalignmat[c]{
				{\frac{\lambda_2{}^{n}(x_0 + y_0 + z_0)}{3} - \frac{2\lambda_3{}^{n}(-2x_0 + y_0 + z_0)}{6}};
				{\frac{\lambda_1{}^{n}(y_0 - z_0)}{2} + \frac{\lambda_2{}^{n}(x_0 + y_0 + z_0)}{3} + \frac{\lambda_3{}^{n}(-2x_0 + y_0 + z_0)}{6}};
				{-\frac{\lambda_1{}^{n}(y_0 - z_0)}{2} + \frac{\lambda_2{}^{n}(x_0 + y_0 + z_0)}{3} + \frac{\lambda_3{}^{n}(-2x_0 + y_0 + z_0)}{6}}
			}\\
			&= \frac{1}{6}\spalignmat[c]{
				{2x_0\left\{1 + 2(1 - 3\alpha)^{n}\right\} + 2y_0\left\{1 - (1 - 3\alpha)^{n}\right\} + 2z_0\left\{1 - (1 - 3\alpha)^{n}\right\}};
				{2x_0\left\{1 - (1 - 3\alpha)^{n}\right\} + y_0\left\{3(-1 + \alpha)^{n} + 2 + (1 - 3\alpha)^{n}\right\} + z_0\left\{- 3(-1 + \alpha)^{n} + 2 + (1 - 3\alpha)^{n}\right\}};
				{2x_0\left\{1 - (1 - 3\alpha)^{n}\right\} + y_0\left\{-3(-1 + \alpha)^{n} + 2 + (1 - 3\alpha)^{n}\right\} + z_0\left\{3(-1 + \alpha)^{n} + 2 + (1 - 3\alpha)^{n}\right\}}
			}
		\end{align*}
		\item (4)より、題意は以下のようになる。
			\begin{align*}
				&\lim_{n \to \infty} \spalignmat[c]{
					{x_n};
					{y_n};
					{z_n}
				}\\
				&= \lim_{n \to \infty} \frac{1}{6}\spalignmat[c]{
					{2x_0\left\{1 + 2(1 - 3\alpha)^{n}\right\} + 2y_0\left\{1 - (1 - 3\alpha)^{n}\right\} + 2z_0\left\{1 - (1 - 3\alpha)^{n}\right\}};
					{2x_0\left\{1 - (1 - 3\alpha)^{n}\right\} + y_0\left\{3(-1 + \alpha)^{n} + 2 + (1 - 3\alpha)^{n}\right\} + z_0\left\{- 3(-1 + \alpha)^{n} + 2 + (1 - 3\alpha)^{n}\right\}};
					{2x_0\left\{1 - (1 - 3\alpha)^{n}\right\} + y_0\left\{-3(-1 + \alpha)^{n} + 2 + (1 - 3\alpha)^{n}\right\} + z_0\left\{3(-1 + \alpha)^{n} + 2 + (1 - 3\alpha)^{n}\right\}}
				}
			\end{align*}
			ここで、題意より$0 < \alpha < \frac{1}{3}$より、$-1 < -1 + \alpha < -\frac{2}{3}$, $0 < 1 - 3\alpha < 1$である。
			よって、
			\begin{align*}
				\lim_{n \to \infty} \spalignmat[c]{
					{x_n};
					{y_n};
					{z_n}
				} &= \frac{1}{6}\spalignmat[c]{
					{2x_0 + 2y_0 + 2z_0};
					{2x_0 + 2y_0 + 2z_0};
					{2x_0 + 2y_0 + 2z_0}
				}\\
				&= \frac{1}{3}\spalignmat[c]{
					{x_0 + y_0 + z_0};
					{x_0 + y_0 + z_0};
					{x_0 + y_0 + z_0}
				}
			\end{align*}
			となる。
		\item 
\end{enumerate}
